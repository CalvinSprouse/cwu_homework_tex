% LaTeX for GEOG101 Assignment 1
% declare document
\documentclass[a4paper, 12pt]{article}

% import packages
\usepackage{titling}
\usepackage{datetime}
\usepackage{authblk}
\usepackage{geometry}
\usepackage{setspace}
\usepackage{hyperref}
\usepackage{csquotes}
\usepackage[style=apa, backend=biber]{biblatex}
\usepackage[english]{babel}
\usepackage[bottom]{footmisc}

% import bibliography
\addbibresource{citations.bib}

% reduce page margins
\geometry{left=3cm, right=3cm}

% title information
\title{Baseball is a Dirt Road to God}
\author{Calvin Sprouse}
\affil{RELS 101 World Religions}
\date{2024 May 29}

% set spacing
\doublespacing%

\begin{document}
\maketitle

Religion is often associated with the sacred and the divine, but it also encompasses a wide range of human experiences. Such a broad topic is elusive to a simple definition that includes the common notion of religion and excludes all others. In ``Baseball as a Road to God'' John Sexton explores the idea that baseball is a religion like any other and in doing so challenges the need to exclude not so common notions of religion. Using the framework of Ninian Smart and Sarah Pike it is clear that baseball expresses characteristics of a religion. While the road is rough at times, the exploration of baseball as a religion delivers one to a greater understanding of the religious world.

Religion comes in a myriad of forms often bearing little surface level resemblance to another. To capture these differences and similarities Ninian Smart defines seven aspects or dimensions which taken together create a structure for analyzing religion. The first dimension, the practical and ritual dimension, encompasses the regular behavior of adherents including acts of prayer, meditation, yoga, ritual sacrifice, etc.~\autocite{vaughn_anthology_2024}. A game in baseball begins with rituals. A recitation of the national anthem and ceremonial first pitch are regular occurrences which mark the beginning of a sacred experience for fans. The second dimension, experiential and emotional, is ontological to baseball and could be described as the reason for baseballs existence. The sport draws hundreds of thousands of people out of the comfort of their homes and into crowded stadiums just so their cries of faith in their team may be heard. The narrative and mythic dimension describes the stories of a faith both historical and mystic which are handed down by adherents through generations~\autocite{vaughn_anthology_2024}. Baseball hands down stories of curses and events so miraculous they evade explanation~\autocite{sexton_baseball_2013}. Though they are born in history the stories of baseball take on a deeper more profound meaning through recitation. It is through these stories that one ``looks for God in exceptional events''~\autocite{sexton_baseball_2013}. In the doctrinal and philosophical dimension baseball again demonstrates its qualities as a religion. Baseball may be defined by rules of the game but the consistency in variation expresses an interpretive human element: ``according ot the rule book, the strike zone stretches from the midpoint between the top of the batter's shoulders and the top of the uniform pants to just below the knees \dots\ but umpires have always added a dose of discretion''~\autocite{sexton_baseball_2013}. That this dose of discretion is known about and even fought for by fans is evidence of an underlying philosophy in baseball. But the rules themselves express baseball in the ethical and legal dimensions of Ninian Smart, ``the law which a tradition or subtradition incorporates into its fabric''~\autocite{vaughn_anthology_2024}. Baseball is permeated with rules of play which players abide by or are punished by higher powers. Even off the field players may be suspended for moral violations which would damage the teams image. The teams themselves represent he social and institutional dimension: ``every religious movement is embodied in a group of people, and that is very often rather formally organized''~\autocite{vaughn_anthology_2024}. Teams are the pantheons to which fans devote themselves to. These pantheons make their homes in cities which are often adorned in decor and praise put up by the believers, the fans. Which leads finally to the material dimension, ``buildings, works of art, and other creations''~\autocite{vaughn_anthology_2024}. This is expressed in the stadiums, jerseys, team memorabilia and collectibles which permeate fans homes. The lucky hats or shirts of some fans take on a meaning that transcends the fibers that compose the garment, exalting the material to a nearly religious status. Through these seven dimensions baseball makes a case for its religious characteristics. Baseball evokes an emergent religious experience without a mention of a God, a creation story, a death philosophy, or any traditional notion of what a religion provides.

Despite a clear expression of the seven dimensions of religion according to Ninian Smart, baseball will not begin to compete with any major or minor religions for members. What baseball does then, in being considered a religion, is speak to Sarah Pikes comment that ``the deep need to categorize some religions as normal, and others as deviant, has distorted both academic study and popular understanding of religion''~\autocite{vaughn_anthology_2024}. In the intellectual pursuit of considering baseball a religion one fights with ones own preconceived notions of religion. That there must be some death philosophy, creation story, explicit prayer to preform, or God to worship are hallmarks of major religions that subvert a holistic understanding. It is these hurdles which make baseball a rough road to God.

% bibliography
\pagebreak
\printbibliography[title={Works Cited}]

\end{document}
