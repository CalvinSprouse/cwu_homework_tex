% LaTeX Document made from HomeworkTemplate
% Last Updated: 2023 november 17

% document style header
\documentclass[a4paper, 12pt]{../../config/homework}

% import default packages
\usepackage{../../config/defpackages}

% import fun symbol packages (and coffee)
% \usepackage{.config/sympack}

% import custom math commands
\usepackage{../../config/domath}

% end preamble
\begin{document}

% document title
\noindent
\begin{tabularx}{\textwidth}{>{\centering\arraybackslash}X>{\centering\arraybackslash}X>{\centering\arraybackslash}X}
Calvin Sprouse & MATH314 Lab 7 & Due 2023 Nov 29 1700\\\hline
\end{tabularx}

% homework problems begin
\noindent \textit{Part I:} Preform a chi-square test to check whether there is a statistically significant difference in low birth weight from nonsmoking mothers and smoking mothers. Make sure to state your hypotheses, include appropriate results from the test, and clearly state your conclusions.
\\Our null and alternative hypothesis are
\begin{enumerate}
\item[$H_0:$] there is no statistically significant difference in low birth weight from nonsmoking mothers and smoking mothers,
\item[$H_1:$] there is a statistically significant difference in low birth weight from nonsmoking mothers and smoking mothers.
\end{enumerate}
The $p$-val for this chi-square test, as determine by R, is $\pval = 0.27$.
Since $\alpha=0.05$, $\pval>\alpha$ thus we fail to reject the null and must conclude that there is no statistically significant difference in low birth weight from nonsmoking mothers and smoking mothers.

\vspace{2\singlelineheight} \noindent
\textit{Part II:} Give a 90\% confidence interval for the weight gained by mother during pregnancy and interpret it in context.
\\A 90\% confidence interval for the weight gained by mother during pregnancy is given by
\[\bar{x} \pm t^* \frac{s}{\sqrt{n}}.\]
From R, this calculation gives the interval
\[\left(29.74, 30.91\right).\]
We can state that we are 90\% certain that the true value for the average weight gained by a mother during pregnancy falls within $29.74$ and $30.91$ pounds.

\pagebreak \noindent
\textit{Part III:} An old article wrote that the typical length of pregnancies is 39 weeks. Can we claim that the typical length of pregnancies is now less than 39 weeks? Use the \textbf{nc} data set to conduct a hypothesis test to verify the claim. As always, state your hypotheses, include appropriate results from the test, and clearly state your conclusions.
\\Our null and alternative hypothesis are
\[H_0:\mu = 39, \quad H_1:\mu < 39.\]
From R, the $p$-val for this hypothesis test is given by
\[\pval = \expnumber{7.31}{-13}.\]
For $\alpha = 0.05$, $\pval < \alpha$ thus we accept the alternative hypothesis. There is enough evidence to conclude that the typical length of pregnancies is less than 39 weeks.

\vspace{2\singlelineheight} \noindent
\textit{Bonus Question (i.e. Extra Credit):} Conduct a hypothesis test evaluating whether the average weight gained by younger mothers is different that then average weight gained by mature mothers.
\\Let $\mu_\text{y}$ be the average weight gain of younger mothers and $\mu_\text{m}$ the average weight gain of mature mothers. Our null and alternative hypothesis are given by
\[H_0:\mu_\text{y} = \mu_\text{m}, \quad H_1 : \mu_\text{y} \ne \mu_\text{m}.\]
The $p$-val for this hypothesis is given by R as
\[\pval = 0.17.\]
For $\alpha = 0.05$, $\pval > \alpha$ so we fail to reject the null. There is not enough evidence to indicate there is a difference in weight gain between young mothers and mature mothers.
\end{document}
