% document style header
\documentclass[a4paper, 12pt]{config/homework}

% import default packages
\usepackage{config/defpackages}

% import custom math commands
\usepackage{config/domath}

% end preamble
\begin{document}

% document title
\noindent
\begin{tabularx}{\textwidth}{>{\centering\arraybackslash}X>{\centering\arraybackslash}X>{\centering\arraybackslash}X}
Calvin Sprouse & MATH407 HW1 & Due 2024 Jan 11\\
\midrule
\end{tabularx}

% homework problems begin
\begin{enumerate}
\item Let \(Y\) be the set of all living people, and let the binary relation be ``is married to'' (assuming monogamy throughout society).
\begin{enumerate}[label=(\alph*)]

\item Identify the meaning of cases (1) through (4) as found in slide 5/14.
\begin{enumerate}[label=Case (\arabic*).]
\item \(\left(xRy, yRx\right)\). \(x\) is married to \(y\) and \(y\) is married to \(x\).
\item \(\left(xRy, \text{not}\ yRx\right)\). \(x\) is married to \(y\) and \(y\) is not married to \(x\).
\item \(\left(\text{not}\ xRy, yRx\right)\). \(x\) is not married to \(y\) and \(y\) is married to \(x\).
\item \(\left(\text{not}\ xRy, \text{not}\ yRx\right)\). \(x\) is not married to \(y\) and \(y\) is not married to \(x\).
\end{enumerate}

\item For each property p1 through p9, as found in slide 6/14, state whether or not the property is satisfied and why.
\begin{enumerate}[label=p\arabic*.]
\item Reflexive. \(x\) is married to \(x\). This is not satisfied; society does not recognize marriage to the self.
\item Irreflexive. \(x\) is not married to \(x\). This is satisfied; society does not recognize marriage to the self.
\item Symmetric. If \(x\) is married to \(y\) then \(y\) is married to \(x\). This is satisfied; two people are mutually in marriage to one another.
\item Asymmetric. If \(x\) is married to \(y\) then \(y\) is not married to \(x\). This is not satisfied; two people are mutually in marriage to one another as a pair.
\item Antisymmetric. If \(x\) is married to \(y\) and \(y\) is married to \(x\) then \(x\) is \(y\). This is not satisfied; a marriage exists between two unique people.
\item Transitive. If \(x\) is married to \(y\) and \(z\) is married to \(x\) then \(x\) is married to \(z\). This is not satisfied; a person cannot be married to two people so person \(x\) cannot be married to both \(y\) and \(z\) by assumption of monogamy.
\item Negatively transitive. If \(x\) is not married to \(y\) and \(y\) is not married to \(z\) then \(x\) is not married to \(z\). This is not satisfied; if \(x\) were married to \(z\) it could still be said that \(x\) is not married to \(y\) and \(y\) is not married to \(z\).
\item Connected. \(x\) is married to \(y\) and \(y\) is married to \(x\) for all people \(x\) and \(y\). This is not satisfied; there are people who are not married.
\item Weakly connected. If \(x\) and \(y\) are different people then \(x\) is married to \(y\) or \(y\) is married to \(x\). This is not satisfied; there are people who are not married.
\end{enumerate}
\end{enumerate}

\pagebreak
\item Show that asymmetry (p4) and negative transitivity (p7) imply transitivity (p6).
\begin{proof}
Let \(Y\) be some set and \(R\) be some binary operation on \(Y\) with the properties asymmetry and negative transitivity.
Let \(x,y,z\in Y\).
Suppose \(xRy\) and \(yRz\).

By asymmetry \(xRy \rightarrow \text{not}\ yRx\).

By asymmetry \(yRz \rightarrow \text{not}\ zRy\).

By negative transitivity \((\text{not}\ zRy, \text{not}\ yRx) \rightarrow \text{not}\ zRx\).

Then by asymmetry, \(\text{not}\ zRx \to \text{not}\ \text{not}\ xRz \equiv xRz\).

Thus by asymmetry and negative transitivity we have \((xRy, yRz) \to xRz\) which is the property transitivity.
\end{proof}

\end{enumerate}
\end{document}
