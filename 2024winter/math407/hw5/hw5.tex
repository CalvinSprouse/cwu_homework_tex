% document style header
\documentclass[a4paper, 12pt]{config/homework}

% import default packages
\usepackage{config/defpackages}
% import custom math commands
\usepackage{config/domath}

% end preamble
\begin{document}

% document title
\noindent
\begin{tabularx}{\textwidth}{>{\centering\arraybackslash}X>{\centering\arraybackslash}X>{\centering\arraybackslash}X}
Calvin Sprouse & MATH407 HW5 & 2024 February 02\\
\midrule
\end{tabularx}

% choose two (or three for EC) papers to read to narrow down final project
% respond to each
% What are five important facts, words, concepts, etc. that you want to remember from the paper? Why do you want to remember them? Why are they significant?
% Write five questions that you developed as you read through the paper.
% Summarize your entire reading of this paper into one well-written paragraph. Try to use your own words as much as you can.
% Again, you should respond to the above questions for EVERY PAPER, so there will be 6 (or 9) responses total. After you are done with the all responses, please provide the following information:
% Rank the papers as ``most preferred project'', ``second most preferred project'', etc.
% Do you have a project partner in mind, and if so, please indicate who the person is?

% homework problems begin
% paper 1: thermodynamics https://royalsocietypublishing.org/doi/full/10.1098/rspa.2012.0683
\section{Thermodynamics as a theory of decision-making with information-processing costs}
Most preferred. Project partner: Nicholas Klein.

% What are five important facts, words, concepts, etc. that you want to remember from the paper? Why do you want to remember them? Why are they significant?
% Write five questions that you developed as you read through the paper.
% Summarize your entire reading of this paper into one well-written paragraph. Try to use your own words as much as you can.
\noindent
Five important facts, words, concepts, etc.\ from this paper:
\begin{enumerate}
\item Bounded rational decision-makers. Important because this is the type of decision maker being studied and represents a slightly more realistic decision maker bounded by information-processing resources.
\item That \(kT\) can be interpreted as a conversion from information to energy seems like an important perspective both for this understanding of utility and even for physics.
\item An environment can be characterized as having a rationality, ex.\ ambiguity is an anti-rational environment, and said environment can then be thought of as being a decision maker itself.
\item Reaction times are related, but not equivalent to, computational resources. This makes sense and explains how irrational choices can be formulated as rational when time to consider is limited.
\end{enumerate}

\noindent
Five questions that I developed in reading the paper:
\begin{enumerate}
\item How do we consider computation costs in every day decisions?
\item Is the idea that we only consider the difference in free energy, and not absolute free energy, related to the fact we consider proportional wealth change not absolute wealth in utility functions?
\item What is an information state?
\item What is a prior distribution?
\item Is there a meaning behind the authors use of square-brackets or are they equivalent to parenthesis?
\end{enumerate}

\noindent
Single-paragraph summary of the paper:

The physics understanding of thermodynamic statistics provides a framework for modeling decision making. The temperature of a physical system can be thought of as the rationality or deviation from pure rational decision making. This temperature is affected by the environment and limited computational resources. This treatment of decision makers as thermodynamic objects explains some `paradoxes' in utility theories by reframing irrational decisions as rational in the context of the decision makers resources and rationality.

% paper 2: savage axioms https://www.jstor.org/stable/1884324
\section{Risk, Ambiguity, and the Savage Axioms}
Second most preferred. Same project partner.

My apologies, I only had time to skim this paper a little as I've been sick and trying to recover for most of this week.

\end{document}
