% document style header
\documentclass[a4paper, 12pt]{config/homework}

% import default packages
\usepackage{config/defpackages}
% import custom math commands
\usepackage{config/domath}

% end preamble
\begin{document}

% document title
\noindent
\begin{tabularx}{\textwidth}{>{\centering\arraybackslash}X>{\centering\arraybackslash}X>{\centering\arraybackslash}X}
Calvin Sprouse & MATH407 HW4 & 2024 January 30\\
\midrule
\end{tabularx}

% homework problems begin
\begin{enumerate}
\item A decision maker is described by the utility function \(u(w)=w^{1/3}\). She is given the choice between two random amounts \(X_1\) and \(X_2\), in exchange for her entire present wealth \(w_0\). Suppose that
\[X_1 = \begin{cases}
\phantom{0}8 & \text{with probability}\ 0.5 \\
27 & \text{with probability}\ 0.5
\end{cases}\] and
\[X_2 = \begin{cases}
\phantom{0}1 & \text{with probability}\ 0.6 \\
64 & \text{with probability}\ 0.4
\end{cases}\]
\begin{enumerate}[label=(\alph*)]
\item Show that she prefers \(X_1\) to \(X_2\).

The expected utility of option 1 is given by
\[E(u(X_1)) = \frac{1}{2}8^{1/3} + \frac{1}{2}27^{1/3} = \frac{5}{2} = 2.5.\]
The expected utility of option 2 is given by
\[E(u(X_2)) = 0.6\times1^{1/3} + 0.4\times64^{1/3} = 2.2.\]
The expected utility of option 1 is greater than that of option 2 so option 1 is preferred.

\item Determine for what values of \(w_0\) she should decline the offer.

The decision make should decline the offer if the expected utility of an option, say \(X_1\) or \(X_2\), is less than the expected utility of doing nothing. The expected utility of doing nothing is to simply keep the wealth, that is,
\[E(u(w_0)) = w_0^{1/3}.\]
Since \(X_1\) is preferred to \(X_2\) we need only consider how doing nothing compares to \(X_1\). Thus we consider when \(E(u(w_0)) > E(u(X_1))\) which is given by \(w_0^{1/3} > 5/2\) which is given by
\[w_0 > \left(\frac{5}{2}\right)^{3} \approx 15.63.\]
Thus, the offer should be declined for a starting wealth of 1.52 units.

\item Give an example of a utility function in which she would prefer \(X_2\) to \(X_1\).

The utility function
\[u(w) = \frac{w^{0.9} - 1}{0.9}\]
results in the following expected utilities
\begin{align*}
E(u(X_1)) &= 13.29,
\\E(u(X_2)) &= 18.32.
\end{align*}
Thus for this utility function \(X_2\) is preferred to \(X_1\).

\end{enumerate}
\item Recall that the iso-elastic property says that for any \(k > 0\), \(u(kw) = f(k)u(w) + g(k)\) for some \(f(k)\) and \(g(k)\).
\begin{enumerate}[label=(\alph*)]
\item Identify the functions \(f(k)\) and \(g(k)\) in the case of \(u(w)=\ln(w)\).
\[u(kw) = \ln(kw) = \ln(w) + \ln(k).\]
Thus \(f(k)=1\) and \(g(k) = \ln(k)\).

\item Identify the functions \(f(k)\) and \(g(k)\) in the case of \(u(w)=\frac{w^\lambda - 1}{\lambda}\).
\begin{align*}
u(kw) &= \frac{k^\lambda w^{\lambda}}{\lambda}
\\&= \frac{k^\lambda w^{\lambda}}{\lambda} + \frac{k^\lambda}{\lambda} - \frac{k^{\lambda}}{\lambda}
\\&= \frac{(k^\lambda w^{\lambda} - k^\lambda) + k^{\lambda - 1}}{\lambda}
\\&= \frac{k^\lambda(w^\lambda - 1) + k^\lambda - 1}{\lambda}
\\&= k^\lambda \frac{w^\lambda - 1}{\lambda} + \frac{k^\lambda - 1}{\lambda}.
\end{align*}
Thus \(f(k) = k^\lambda\) and \(g(k) = u(k).\)

\end{enumerate}
\pagebreak
\item Recall that the Arrow-Pratt absolute risk aversion function is given by
\[A(w) = -\frac{\diff[2]{u(w)}{w}}{\diff{u(w)}{w}}.\]
\begin{enumerate}[label=(\alph*)]
\item Compute \(A(w)\) in the case of \(u(w)=\ln(w)\). Is \(A(w)\) non-increasing?
\begin{align*}
A(w) &= -\frac{\diff[2]{u(w)}{w}}{\diff{u(w)}{w}}
\\&= -\frac{\diff[2]{\ln(w)}{w}}{\diff{\ln(w)}{w}}
\\&= -\frac{-w^{-2}}{w^{-1}}
\\&= w^{-1}.
\end{align*}
\(A(w)\) is non-increasing.

\item Compute \(A(w)\) in the case of \(u(w)=\frac{w^{\lambda} - 1}{\lambda}\). Is \(A(w)\) non-increasing.
\begin{align*}
A(w) &= -\frac{\diff[2]{u(w)}{w}}{\diff{u(w)}{w}}
\\&= -\frac{\diff[2]{\frac{w^\lambda - 1}{\lambda}}{w}}{\diff{\frac{w^\lambda - 1}{\lambda}}{w}}
\\&= \frac{-(\lambda-1)w^{\lambda -2}}{w^{\lambda - 1}} = -(\lambda - 1)w^{-1}.
\end{align*}
\(A(w)\) is non-increasing.

\end{enumerate}
\end{enumerate}
\end{document}
