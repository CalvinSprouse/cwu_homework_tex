% document style header
\documentclass[a4paper, 12pt]{config/homework}

% import default packages
\usepackage{config/defpackages}
% import custom math commands
\usepackage{config/domath}

% end preamble
\begin{document}

% document title
\noindent
\begin{tabularx}{\textwidth}{>{\centering\arraybackslash}X>{\centering\arraybackslash}X>{\centering\arraybackslash}X}
Calvin Sprouse & PHYS474 Homework 3 & 2024 January 23\\
\midrule
\end{tabularx}

% homework problems begin
\begin{enumerate}
\item A particle is trapped in an infinite square well with a width \(a\). At \(t=0\) is is equally probable for the particle to be found anywhere on the left-side of the well and impossible for it to be found on the right-side of the well.
\begin{enumerate}[label=(\alph*)]
\item What is the normalized wavefunction \(\Psi(x,0)\) that represents this initial state? Note that \(\Psi\) breaks one of the fundamental rules about wavefunctions.

The wavefunction, disregarding the cases of \(x\) outside the range \([0,a]\), is given by
\[\Psi(x,0) = \begin{cases}
    A & 0 \le x \le \frac{a}{2} \\
    0 & \frac{a}{2} < x \le a
\end{cases}\]
where \(A\) is the normalization constant. We see the issue with \(\Psi\) at \(x=a/2\) where we have a discontinuity.
The normalization constant can be found by the typical normalization process:
\[1 = \bint{-\infty}{\infty}{\left|\Psi\right|^2}{x}.\]
We recognize \(\Psi(x,t=0)\) takes on non-zero values only on the interval \([0,\frac{a}{2}]\) and that \(\Psi(x,t=0)\) is real.
\[1 = A^2 \bint{0}{a/2}{}{x} = A^2 \frac{a}{2}.\]
Thus,
\[A = \sqrt{\frac{2}{a}}\]
and consequently
\[\Psi(x,0) = \begin{cases}
    \sqrt{\frac{2}{a}} & 0 \le x \le \frac{a}{2} \\
    \phantom{\sqrt{}}0 & \frac{a}{2} < x \le a
\end{cases}\]

\pagebreak
\item What is the probability that you would measure an energy of
\[E = \frac{4\pi^2\hbar^2}{2ma^2}\]
at \(t=0\).

% whence the quantum came.
The energy of any infinitely-square-well-ed particle is given by
\[E_n = \frac{n^2\pi^2\hbar^2}{2ma^2}.\]
By inspection, our particle is in the \(n=2\) state. The probability of finding a particle in the \(n=2\) state is given by \(\left| C_2 \right|^2\) where \(C_2\) is found through linearization of \(\Psi\). The \(n\)th linear coefficient is given as
\[C_n = \bint{-\infty}{\infty}{\psi_n^* \Psi}{x} = \frac{2}{a}\bint{0}{a/2}{\sin\left(\frac{n\pi}{a}x\right)}{x} = \frac{2}{n\pi}\left(1 - \cos\left(\frac{n\pi}{2}\right)\right).\]
Thus, \(C_2\) is given as
\[C_2 = \frac{2}{\pi},\]
hence, the probability of measuring the 2nd energy state is given as
\[P\left(E_2\right) = \left| C_2 \right|^2 = \left(\frac{2}{\pi}\right)^2 \approx 0.41 .\]

\end{enumerate}

\pagebreak
\item Consider the standard infinite square well with width \(a\). The stationary state solutions are \(\psi_n(x)\).
\begin{enumerate}[label=(\alph*)]
\item Compute \(\braket{x}\) and \(\braket{x^2}\) for \(\psi_n(x)\).

The expectation value of \(x\) is given by
\begin{align*}
\braket{x} &= \bint{-\infty}{\infty}{\psi_n^* [x] \psi_n}{x}
\\&= \bint{0}{a}{x \psi_n^*\psi_n}{x}
\\&= \frac{2}{a}\bint{0}{a}{x\sin^2\left(\frac{n\pi}{a}x\right)}{x}
\\&= \frac{2}{a}\left[\frac{x^2}{4} - \frac{x\sin\left(\frac{2\pi n}{a}x\right)}{4\left(\frac{n\pi}{a}\right)} - \frac{\cos\left(\frac{2\pi n}{a}x\right)}{8\left(\frac{n\pi}{a}\right)^2}\right]_0^a
\\&= \frac{2}{a} \left( \frac{a^2}{4} - \frac{a\sin(2\pi n)}{4\left(\frac{n\pi}{a}\right)} - \frac{\cos(2\pi n)}{8 \left(\frac{n\pi}{a}\right)^2} - 0 + 0 + \frac{1}{8\left(\frac{n\pi}{a}\right)^2} \right)
\\&= \frac{2}{a}\left(\frac{a^2}{2} - \frac{1}{8 \left(\frac{n\pi}{a}\right)^2} + \frac{1}{8 \left(\frac{n\pi}{a}\right)^2}\right)
\\&= \frac{a}{2}.
\end{align*}

The mean square displacement \(\braket{x^2}\) is given by
\begin{align*}
\braket{x^2} &= \bint{-\infty}{\infty}{\psi_n^*\left[x^2\right]\psi_n}{x}
\\&= \frac{2}{a} \bint{0}{a}{x^2 \sin^2\left(\frac{n\pi}{a}x\right)}{x}
\\&= \frac{2}{a} \left[
\frac{x^3}{6}
- \left(\frac{x^2}{4\left(\frac{n\pi}{a}\right)}
- \frac{1}{8\left(\frac{n\pi}{a}\right)^3}\right)
\sin\left(\frac{2\pi n}{a}x\right)
- \frac{x\cos\left(\frac{2\pi n}{a}x\right)}{4\left(\frac{n\pi}{a}\right)^2}
\right]_0^a
\\&= \frac{2}{a}\left(
\frac{a^3}{6}
- \left(
\frac{a^2}{4\left(\frac{n\pi}{a}\right)}
- \frac{1}{8\left(\frac{n\pi}{a}\right)^3}
\right)\sin(2\pi n)
- \frac{a \cos\left(2\pi n\right)}{4\left(\frac{n\pi}{a}\right)^2}
- 0 \right)
\\& \frac{2}{a} \left(\frac{a^3}{6} - 0 - \frac{a}{4\left(\frac{n\pi}{a}\right)^2}\right)
\\&= a^2 \left(\frac{1}{3} - \frac{1}{2(n\pi)^2}\right).
\end{align*}

\pagebreak
\item Compute \(\braket{p}\) and \(\braket{p^2}\) for \(\psi_n(x)\).

The mean momentum \(\braket{p}\) is given by
\[\braket{p} = \diff{\braket{x}}{t} = 0.\]

The mean square momentum \(\braket{p^2}\) is given by
\begin{align*}
\braket{p^2} &= \bint{-\infty}{\infty}{\psi_n^*\left[-i\hbar \diffp{}{x}\right]^2 \psi_n}{x}
\\&= \hbar^2 \bint{0}{a}{\psi_n^* \diffp[2]{}{x}\psi}{x}
\\&= \hbar^2 \left(\frac{n\pi}{a}\right)^2 \bint{0}{a}{\psi_n^* \psi}{x}
\\&= \frac{n^2 \hbar^2 \pi^2}{a^2}
\\&= 2m E_n.
\end{align*}

\pagebreak
\item Compute \(\sigma_x\) and \(\sigma_p\) and confirm that the uncertainty principle is satisfied for all allowed \(n\).

The standard deviation of position \(\sigma_x\) is given by
\begin{align*}
\sigma_x &= \sqrt{\braket{x^2} - \braket{x}^2}
\\&= \left( a^2\left(\frac{1}{3} - \frac{1}{2(n\pi)^2}\right) - \frac{a^2}{4}\right)
\\&= a \sqrt{\frac{1}{12} - \frac{1}{2(n\pi)^2}}.
\end{align*}

The standard deviation of momentum \(\sigma_p\) is given by
\begin{align*}
\sigma_p &= \sqrt{\braket{p^2} - \braket{p}^2}
\\&= \sqrt{2mE_n - (0)^2}
\\&= \sqrt{2m E_n}.
\end{align*}

To satisfy the uncertainty principle
\[\sigma_x \sigma_p \ge \frac{\hbar}{2}\]
for all \(n\). The product of standard deviations is given by
\begin{align*}
\sigma_x \sigma_p &= a \sqrt{\frac{1}{12} - \frac{1}{2(n\pi)^2}}\sqrt{2m E_n}
\\&= \hbar n \pi \sqrt{\frac{1}{12} - \frac{1}{2(n\pi)^2}}.
\end{align*}
For all allowed \(n\) \(\sigma_x\sigma_p\) is smallest for \(n=1\). Thus, if \(\sigma_x\sigma_p\) satisfies the uncertainty principle at \(n=1\) it satisfies the uncertainty principle at all \(n\). For \(n=1\)
\[\sigma_x \sigma_p = \hbar \pi \sqrt{\frac{1}{12} - \frac{1}{2\pi^2}}.\]
Thus our inequality can be expressed as
\[\pi \sqrt{\frac{1}{12} - \frac{1}{2\pi^2}} \ge \frac{1}{2}.\]
The quantity on the left is approximately 0.56 which is greater than the quantity on the right 0.5. Thus, \(\sigma_x\) and \(\sigma_p\) satisfy the uncertainty principle for all allowed \(n\).

\end{enumerate}

\pagebreak
\item A particle infinitely-square-well-ed to width \(a\) is initially observed in a quantum state described by the wavefunction
\[\Psi(x,0) = A \left(\psi_1(x) + \psi_3(x)\right),\]
where \(A\) is a real positive constant and both \(\psi_1(x)\) and \(\psi_3(x)\) are solutions to the time-independent Schr{\"o}dinger equation for \(n=1\) and \(n=3\) respectively.
\begin{enumerate}[label=(\alph*)]
\item Normalize \(\Psi(x,0)\) assuming \(\psi_1\) and \(\psi_3\) are separately normalized.

We begin with typical normalization.
\[1 = \bint{-\infty}{\infty}{\left|\Psi(x,0)\right|^2}{x} = A^2\left( \bint{-\infty}{\infty}{\left|\psi_1\right|^2}{x} + \bint{\infty}{\infty}{\left|\psi_3\right|^2}{x}\right) = 2A^2,\]
where we have used the assumption that \(\psi_1\) and \(\psi_3\) are separately normalized to evaluate the integrals. Thus,
\[A = 2^{-1/2}.\]

\item Compute \(\left|\Psi\right|^2\) and simplify as much as possible.

The time-dependent \(\Psi\) is obtained by adding the typical \(\phi(t)\) to each stationary state describing \(\Psi\).
\[\Psi = \frac{1}{\sqrt{2}} \left(\psi_1 e^{-it \frac{E_1}{\hbar}} + \psi_3e^{-it \frac{E_3}{\hbar}}\right).\]
Then,
\begin{align*}
\left|\Psi\right|^2 &= \Psi^* \Psi
\\&= \frac{1}{2} \left(\psi_1 e^{it \frac{E_1}{\hbar}} + \psi_3e^{it \frac{E_3}{\hbar}}\right) \left(\psi_1 e^{-it \frac{E_1}{\hbar}} + \psi_3e^{-it \frac{E_3}{\hbar}}\right)
\\&= \frac{1}{2} \left( \psi_1^2 + \psi_3^2 + \psi_1\psi_3\exp\left(\frac{1}{\hbar}\left( E_1 - E_3 -E_1 + E_3 \right) \right) \right)
\\&= \frac{1}{2} \left( \psi_1^2 + \psi_3^2 \right).
\end{align*}

\item If you measured the particle's energy, what value(s) might you possibly obtain and what is the probability of measuring them?

The factor \(1/2\) can be interpreted as both \(\left|C_1\right|^2\) and \(\left|C_3\right|^2\). Since this is the representation of \(\Psi\) as a linear combination of stationary states we say that all other coefficients \(C_n\) are 0 for \(n\ne1,3\). Thus, the energy states \(E_1\) and \(E_3\) are the only possible energy states each with an equal probability of being measured.

\end{enumerate}
\end{enumerate}
\end{document}
