% document style header
\documentclass[a4paper, 12pt]{config/homework}

% import default packages
\usepackage{config/defpackages}

% import custom math commands
\usepackage{config/domath}

% end preamble
\begin{document}

% document title
\noindent
\begin{tabularx}{\textwidth}{>{\centering\arraybackslash}X>{\centering\arraybackslash}X>{\centering\arraybackslash}X}
Calvin Sprouse & PHYS474 Homework 2 & Due 2024 Jan 17\\
\midrule
\end{tabularx}

% homework problems begin
\begin{enumerate}
\item Let the wavefunction \(\Psi(x,t)\) be a solution to the time-dependent Schr{\"o}dinger equation when the potential energy is given by \(V(x)\). What is the solution to the Shr{\"o}dinger equation if we now consider a potential of \(V(x) + V_0\) where \(V_0\) is a real positive constant.

Disclaimer: it is late, I'm very tired and have spend the last couple hours looking at this problem so type-setting seems very daunting.

Let \(\Psi'\) be our wavefunction for our new potential. If we do separation of variables as described in Griffiths \S 2.1 we arrive at Eq.\ 2.3:
\[i\hbar \frac{1}{\phi'} \diff{\phi'}{t} = -\frac{\hbar^2}{2m} \frac{1}{\psi} \diff[2]{\psi'}{x} + V + V_0.\]
Here is where the magic happens, a process I still have no justification for other than it works and not doing results in nonsense (or hours at a whiteboard). By subtracting \(V_0\) to the ``time-side'' of the equation and proceeding to Eq.\ 2.4 we find
\[i\hbar \frac{1}{\psi'} \diff{\psi'}{t} - V_0 = E.\]
Proceeding to solve this differential equation we find
\[\psi'(t) = \exp\left(\frac{-iEt}{\hbar}\right)\exp\left(\frac{-i V_0 t}{\hbar}\right).\]
Behold! Phase factor! It is unclear to me why this should be the ``correct'' thing to do but keeping \(V_0\) in the ``position-side'' is not. Perhaps keeping it on the ``position-side'' will work out in the end and produce a similar result. Nonetheless, it seems the only difference between \(\Psi \) and \(\Psi' \) is a phase-factor in the time-dependent component.

\pagebreak
\item A particle is observed in a quantum state described by the wavefunction
\[\Psi(x,t) = A\exp\left(-a\left(\frac{mx^2}{\hbar}+it\right)\right),\]
where \(A\) and \(a\) are real positive constants.
\begin{enumerate}[label = (\alph*)]

\item Normalize \(\Psi \).

% As shown in Griffiths, if \(\Psi \) is normalized at a particular time \(t\) then it is normalized at all times \(t\). We then normalize \(\Psi \) at \(t=0\).
\begin{align*}
1 &= \bint{-\infty}{\infty}{\left|\Psi\right|^2}{x}
\\&= \bint{-\infty}{\infty}{\Psi^* \Psi}{x}
\\&= \bint{-\infty}{\infty}{A\exp\left(-a\left(\frac{mx^2}{\hbar}+it\right)\right) A\exp\left(-a\left(\frac{mx^2}{\hbar}-it\right)\right)}{x}
\\&= A^2 \bint{-\infty}{\infty}{\exp\left(-a\left(\frac{mx^2}{\hbar} + it + \frac{mx^2}{\hbar} - it\right)\right)}{x}
\\&= A^2 \bint{-\infty}{\infty}{\exp\left(-\frac{2am}{\hbar}x^2\right)}{x}
\\&= A^2 \sqrt{\frac{\pi}{\frac{2am}{\hbar}}}.
\end{align*}
Solving for \(A\) we find
\[A = {\left(\frac{2am}{\pi \hbar}\right)}^{1/4}.\]

\pagebreak
\item What is the potential \(V(x)\) that this particle finds itself within?
\(\Psi \) is, by definition, a solution to the Schr{\"o}dinger equation. Thus,
\[i\hbar \diffp{\Psi}{t} = -\frac{\hbar^2}{2m} \diffp[2]{\Psi}{x} + V\Psi.\]
Arranging for the potential energy function \(V\) we get
\[V = \frac{1}{\Psi} \left(i\hbar \diffp{\Psi}{t} + \frac{\hbar^2}{2m} \diffp[2]{\Psi}{x}\right).\]
To make these partial derivatives less threatening, we can begin by writing \(\Psi \) in the form \(\Psi = A \psi(x) \phi(t)\).
The \(A\) component is simply \(A\), as found by normalization.
The exponential term in \(\Psi \) can be separated into an \(x\)-dependent and \(t\)-dependent component;
\[\exp\left(\frac{-amx^2}{\hbar} - ait\right) = \exp\left(-\frac{amx^2}{\hbar}\right)\exp\left(-ait\right),\]
which become \(\psi \) and \(\phi \) respectively. We know
\[\phi(t) = \exp\left(-\frac{iEt}{\hbar}\right) = \exp\left(-ait\right),\]
thus \(a = E / \hbar \). We can now write \(V\) in a more approachable way with ordinary derivatives:
\[V = \frac{1}{A\psi\phi} \left(i\hbar A\psi \diff{\phi}{t} + \frac{\hbar^2}{2m} A\phi \diff[2]{\psi}{x}\right).\]
The ordinary derivatives are
\[\diff{\phi}{t} = \diff{}{t}\left[\exp(-ait)\right] = -ai \exp(-ait) = -ai \phi,\]
and
\[\diff[2]{\psi}{x} =
% \diff[2]{}{x}\left[\exp\left(-\frac{amx^2}{\hbar}\right)\right] =
\diff{\psi}{x}\left[-\frac{2amx}{\hbar} \exp\left(-\frac{amx^2}{\hbar}\right)\right] =
% -\frac{2amx}{\hbar} \psi.
\frac{-2am}{\hbar}\left(1 - \frac{2am}{\hbar}x^2\right)\psi.
\]
Thus the potential energy function \(V\) is given by
\begin{align*}
V &= \frac{1}{A\psi\phi} \left(i\hbar A\psi \diff{\phi}{t} + \frac{\hbar^2}{2m} A\phi \diff[2]{\psi}{x}\right)
\\&= \frac{1}{A\psi\phi} \left(-i\hbar A\psi  \frac{E}{\hbar} i \phi + \frac{\hbar^2}{2m} A\phi \frac{-2Em}{\hbar^2}\left(1 - \frac{2Em}{\hbar^2}x^2\right)\psi\right)
\\&= E - E\left(1 - \frac{2Em}{\hbar^2}x^2\right)
\\&= \frac{2 m E^2 }{\hbar^2} x^2.
\end{align*}

\pagebreak
\item Determine the expectation values \(\braket{x}\), \(\braket{x^2}\), \(\braket{p}\), \(\braket{p^2}\).
\begin{enumerate}[label=\roman*.]

\item The expectation value of \(x\), \(\braket{x}\), is given by
\[\braket{x} = \bint{-\infty}{\infty}{x \left|\Psi\right|^2}{x}\]
which is an odd function evaluated over symmetric limits and therefore
\[\braket{x} = 0.\]

\item The mean square position, \(\braket{x^2}\), is given by
\begin{align*}
\braket{x^2} &= \bint{-\infty}{\infty}{x^2 \left|\Psi\right|^2}{x}
\\&= A^2 \bint{-\infty}{\infty}{x^2 \exp\left(- \frac{2am}{\hbar} x^2\right)}{x}
\\&= \sqrt{\frac{2am}{\pi\hbar}} \frac{\sqrt{\pi}}{2 {\left(\frac{2am}{\hbar}\right)}^{3/2} }
\\&= \frac{\hbar^2}{4Em}.
\end{align*}

\item The expectation momentum, \(\braket{p}\), is given by
\[\braket{p} = \diff{\braket{x}}{t} = 0.\]

\item The mean square momentum, \(\braket{p^2}\), is given by
\begin{align*}
\braket{p^2} &= \bint{-\infty}{\infty}{\Psi^* {\left[-i\hbar \diffp{}{x}\right]}^2 \Psi}{x}
\\&= -\hbar^2 \bint{-\infty}{\infty}{\Psi^* \diffp[2]{}{x}\Psi}{x}
\\&= -\hbar^2 \bint{-\infty}{\infty}{\Psi^* \left(\frac{-2am}{\hbar^2}(\hbar - 2amx^2)\right) \Psi}{x}
\\&= 2am \bint{-\infty}{\infty}{{\left|\Psi\right|}^2 (\hbar - 2amx^2)}{x}
\\&= 2am \left(\hbar\bint{-\infty}{\infty}{{\left|\Psi\right|}^2}{x} - 2am\bint{-\infty}{\infty}{x^2 {\left|\Psi\right|}^2}{x}\right)
\\&= 2am \left(\hbar - 2am \braket{x^2}\right)
\\&= 2am \left(\hbar - 2am \frac{\hbar^2}{4Em}\right)
\\&= \frac{2Em}{\hbar} \left(\hbar - \frac{2Em}{\hbar}\frac{\hbar^2}{4Em}\right)
\\&= Em.
\end{align*}
\end{enumerate}

\item Determine the standard deviations for position, \(\sigma_x\), and momentum, \(\sigma_p\).

The standard deviation of position is given by
\begin{align*}
\sigma_x &= \sqrt{\braket{x^2} - \braket{x}^2}
\\&= \sqrt{\left(\frac{\hbar^2}{4Em}\right) - {\left(0\right)}^2}
\\&= \frac{\hbar}{2 \sqrt{Em}}.
\end{align*}

The standard deviation of momentum is given by
\begin{align*}
\sigma_p &= \sqrt{\braket{p^2} - \braket{p}^2}
\\&= \sqrt{Em - {\left(0\right)}^2}
\\&= \sqrt{Em}.
\end{align*}

\item Are your values for \(\sigma_x\) and \(\sigma_p\) consistent with the uncertainty principle?

The uncertainty principle, in terms of position and momentum, states
\[\sigma_x \sigma_p \le \frac{\hbar}{2}.\]
Substituting known values we find
\[\sigma_x \sigma_p = \frac{\hbar}{2 \sqrt{Em}} \sqrt{Em} = \frac{\hbar}{2},\]
which satisfies the uncertainty principle. Though, \(\sigma_x \sigma_p\) are the limit of precision allowed by the uncertainty principle.
\end{enumerate}

\pagebreak
\item An electron is trapped in a harmonic quadratic potential. Suppose the expectation value for its position is given by \(\braket{x} = \frac{a}{2}\sin(\omega t)\). Here, \(a\) is a real constant with units of length and \(\omega \) is an angular frequency. What, if anything, can be concluded about the electron's momentum?

The expectation value of momentum can be found in terms of the expectation value of position:
\begin{align*}
\braket{p} &= m \diff{\braket{x}}{t}
\\&= m \diff{}{t}\left[\frac{a}{2}\sin(\omega t)\right]
\\&= \frac{ma \omega}{2} \cos(\omega t),
\end{align*}
where is is assumed that \(\omega \) is not a function of time.

We can begin to analyze the uncertainty, but ultimately without knowing the expectation of the square position or expectation of square momentum not much insight can be gained (trust me I tried).

\end{enumerate}
\end{document}
