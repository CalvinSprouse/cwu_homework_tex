% LaTeX Document made from HomeworkTemplate
% Last Updated: 2023 november 21

% document style header
\documentclass[a4paper, 12pt]{../../config/homework}

% import default packages
\usepackage{../../config/defpackages}

% import fun symbol packages (and coffee)
% \usepackage{.config/sympack}

% import custom math commands
\usepackage{../../config/domath}

% end preamble
\begin{document}

% document title
\noindent
\begin{tabularx}{\textwidth}{>{\centering\arraybackslash}X>{\centering\arraybackslash}X>{\centering\arraybackslash}X}
Calvin Sprouse & MATH260 Quiz 7 & Due 2023 Nov 30 1600\\\hline
\end{tabularx}

% homework problems begin
\begin{enumerate}
\item Use induction to prove that $\sum_{k=1}^{n}{k^3} = \frac{n^2 (n+1)^2}{4}$ for all $n\in\ints^+$.
\[\forall\, n \in \ints^+ \, \sum_{k=1}^{n}{k^3} = \frac{n^2 (n+1)^2}{4}.\]
\begin{proof}
Let $P(n)$ be the statement
\[\sum_{k=1}^{n}{k^3} = \frac{n^2(n+1)^2}{4},\]
for all $n\in\ints^+$.
We proceed with a proof by induction on $n$.
\\\textit{Base case:} $P(1)$ is the statement
\[\sum_{k=1}^{1}k^3 = \frac{1^2(1+1)^2}{4}.\]
Then
\begin{align*}
\sum_{k=1}^{1}k^3 &= 1,
\end{align*}
and
\begin{align*}
\frac{1^2 (1+1)^2}{4} &= \frac{1 (2)^2}{4},
\\&= \frac{4}{4},
\\&= 1.
\end{align*}
Thus, $P(1)$ is true.
\\\textit{Induction step:} Suppose $P(n)$, that is,
\[\sum_{k=1}^{n}k^3 = \frac{n^2(n+1)^2}{4}.\]
Then
\begin{align*}
\sum_{k=1}^{n+1}k^3 &= \sum_{k=1}^{n}k^3 + (n+1)^3,\\
&= \frac{n^2(n+1)^2}{4} + (n+1)^3,\\
&= \frac{n^2(n+1)^2 + 4(n+1)^3}{4},\\
&= \frac{n^4+6 n^3+13 n^2+12 n+4}{4},\\
&= \frac{(n+1)^2(n+2)^2}{4},\\
&= \frac{(n+1)^2(n+1+1)^2}{4}.
\end{align*}
Thus, from $P(n)$ we have proven $P(n+1)$. Therefore, by induction,\\$\sum_{k=1}^{n}{k^3} = \frac{n^2 (n+1)^2}{4}$ for all $n\in\ints^+$.
\end{proof}

\pagebreak
\item Prove that $4|(6^n - 2^n)$ for all $n\in\ints$.
\[\forall\,n\in\ints^+\,4|(6^n - 2^n).\]
\begin{proof}
Let $P(n)$ be the statement
\[4|(6^n - 2^n),\]
for all $n\in\ints^+$.
We proceed with a proof by induction on $n$.
\\\textit{Base case:} $P(1)$ is the statement
\[4|(6^1 - 2^1).\]
Let $k\in\ints$ be defined as
\[k=1.\]
Then
\begin{align*}
6^1 - 2^1 &= 6 - 2,
\\&= 4,
\\&= 4(1),
\\&= 4k.
\end{align*}
Thus, $4|(6^1 - 2^1)$, $P(1)$ is true.
\\\textit{Induction step:} Suppose $P(n)$, that is
\[4|(6^n - 2^n),\]
that there exists $k\in\ints$ such that
\[4k = 6^n - 2^n.\]
Then $P(n+1)$ is the statement
\[4|(6^{n+1} - 2^{n+1}).\]
Let $q\in\ints$ be defined as
\[q = 6^n + 2k.\]
Then
\begin{align*}
6^{n+1}-2^{n+1} &= 6\cdot6^n - 2\cdot2^n,\\
&= (4+2)6^n - 2\cdot2^n,\\
&= 4\cdot6^n + 2\cdot6^n -2\cdot2^n,\\
&= 4\cdot6^n + 2(6^n-2^n),\\
&= 4\cdot6^n + 2(4k),\\
&= 4(6^n + 2k),\\
&= 4q.
\end{align*}
Thus, $4|(6^{n+1} - 2^{n+1})$. Therefore, from $P(n)$ we have proven $P(n+1)$. Thusly, by induction, $4|(6^n - 2^n)$ for all $n\in\ints^+$.
\end{proof}

\pagebreak
\item Let $r\ne 1$ be a real number. Use induction to show that for any $m\in\ints^+$,\\$\sum_{k=m}^{n}{r^k} = \frac{r^m - r^{m+1}}{1-r}$ for all $n\in\ints^+$ with $n \ge m$.
\[\forall\, r\in\reals \left((r\neq 1) \rightarrow \forall\, n,m\in\ints^+\,\left((n \geq m) \rightarrow \left( \sum_{k=m}^{n}{r^k} = \frac{r^m - r^{n+1}}{1-r} \right)\right)\right).\]
\begin{proof}
Let $r\in\reals$ be given such that $r\ne 1$.
We proceed with a proof by induction on $n$.
\\\textit{Base case:} We will prove $P(1)$, that is $n=1$.
There is only one $m\in\ints^+$ such that $n \geq m$, that is $m=1$.
Then, $P(1)$ is the statement
\[\sum_{k=1}^{1} r^k = \frac{r^1 - r^{1+1}}{1-r}.\]
Then
\[\sum_{k=1}^{1} r^k = r,\]
and
\begin{align*}
\frac{r^1 - r^{1+1}}{1-r} &= \frac{r(1-r)}{1-r},\\
&= r.
\end{align*}
Thus, we have proven $P(1)$.
\\\textit{Induction step:} Suppose $P(n)$, that is
\[\sum_{k=m}^{n} r^k = \frac{r^m - r^{n+1}}{1-r}.\]
Then $P(n+1)$ is the statement
\[\sum_{k=m}^{n+1}{r^k} = \frac{r^m - r^{n+2}}{1-r}.\]
Then
\begin{align*}
\sum_{k=m}^{n+1}{r^k} &= \sum{k=m}^n r^k + r^{n+1},\\
&= \frac{r^m - r^{n+1}}{1-r} + r^{n+1},\\
&= \frac{r^m - r^{n+1}}{1-r} + \frac{(1-r)}{(1-r)}r^{n+1},\\
&= \frac{r^m - r^{n+1} + (1-r)r^{n+1}}{1-r},\\
&= \frac{r^m - r^{n+1} + r^{n+1} - r\cdot r^{n+1}}{1-r},\\
&= \frac{r^m - r^{n+1+1}}{1-r},\\
&= \frac{r^m - r^{n+2}}{1-r}.
\end{align*}
Thus we have proven $P(n+1)$ by supposing $P(n)$. Therefore, by induction, we prove
\[\forall\, r\in\reals \left((r\neq 1) \rightarrow \forall\, n,m\in\ints^+\,\left((n \geq m) \rightarrow \left( \sum_{k=m}^{n}{r^k} = \frac{r^m - r^{n+1}}{1-r} \right)\right)\right).\]
\end{proof}
\textit{I couldn't think of an elegant way to restate it.}

\pagebreak
\item Let $A=\left\{ x\in\ints : x\mod 15 = 10\right\}$ and $B=\left\{ x\in\ints : x\mod 3 =1 \right\}$.
\begin{enumerate}[label=(\alph*)]
\item Prove $A \subseteq B$.
\[\forall\,x \in \ints\,\left(x\in A \rightarrow x\in B\right).\]
\begin{proof}
Let $x\in\ints$ be given. Suppose $x\in A$, that is, $x\mod15=10$. Then, by definition, there exists $q\in\ints$ such that
\[x = 15q + 10.\]
Let $k\in\ints$ be defined as
\[k = 5q + 9.\]
Then
\begin{align*}
x &= 15q + 10,
\\&= 15q + 9 + 1,
\\&= 3(5q + 9) + 1,
\\&= 3k + 1.
\end{align*}
Thus, $x\mod3 = 1$. Therefore, if $x\in A$ then $x\in B$. Thus, $A \subseteq B$.
\end{proof}
\item Either show that $B\subseteq A$, or explain why $B \nsubseteq A$.
\\$B\nsubseteq A$. Suppose $x = 1$. Then $1\mod3 = 1$ but $1\mod15\neq 10$. Thus, $\forall x \in \ints\, (x\in B \rightarrow x \in A)$ is not true, so $B \nsubseteq A$.
\end{enumerate}

\vspace{2\singlelineheight}
\item Suppose $A$ and $B$ are subsets of a universe $\mathcal{U}$. Show that if $A\subseteq B^c$, then $A\cap B = \emptyset$.
\begin{proof}
Let \(A\) and \(B\) be subsets of a universe \(\mathcal{U}\). We will use the contrapositive, that is, if \(A \cap B \ne \emptyset \) then \(A \nsubseteq B^c\). Suppose \(A \cap B \ne \emptyset \). Then, there exists some \(x\in\mathcal{U}\) such that \(x\in A\cap B\). Thus, \(x\in A\) and \(x \in B\). Equivalently, \(x\in A\) and \(x \notin B^c\). Therefore, if \(A \cap B \ne \emptyset \) then \(A \nsubseteq B^c\). By the contrapositive, if \(A \subseteq B^c\) then \(A \cap B = \emptyset \).
\end{proof}

\end{enumerate}
\end{document}
