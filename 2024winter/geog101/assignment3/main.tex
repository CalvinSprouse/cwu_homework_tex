% LaTeX for GEOG101 Assignment 1
% declare document
\documentclass[a4paper, 12pt]{article}

% include the titling package
\usepackage{titling}
\usepackage{datetime}
\usepackage{authblk}
\usepackage{geometry}
\usepackage{setspace}
\usepackage[bottom]{footmisc}

% reduce page margins
\geometry{left=3cm, right=3cm}

% title information
\title{Self-Destruction of the European Union}
\author{Calvin Sprouse}
\affil{Geography 101 World Regional Geography}
\date{2024 February 20}

% configure the title block
\setlength{\droptitle}{-7em}
\pretitle{\begin{flushleft}\LARGE}
\posttitle{\par\end{flushleft}}
\preauthor{\begin{flushleft}\large}
\postauthor{\par\end{flushleft}}
\predate{\begin{flushleft}\large}
\postdate{\par\end{flushleft}}

% set spacing
\doublespacing

% the prompt
% Answer the following question in essay form.  This is not a research paper.  The best answers will be those which combine class lecture notes and the textbook and other readings made available on CANVAS for the course.

% Do the best that you can without plagiarizing others and careful with grammar and spelling.  There are no tricks here. The point is to determine levels of comprehension of readings and lectures (1000 words, double spaced, is probably the minimum it would take to answer properly)

% It has been argued in class that the creation of the European Union is a post-``Reformation,'' ``Enlightenment'' project. What does this mean and why does this project seem to be faltering leading to a dangerous confusion over what a ``nation,'' and then a ``nation-state,'' should consist of as the basis for geopolitical ``sovereignty?''

% part 1. talk about why the creation of the EU is a post-reformation enlightenment project.
% part 2. talk about where the EU is faltering.
% part 3. talk about the impact of this faltering.

\begin{document}
\maketitle

The Reformation is a period of great change in the Catholic Church. At the time, the Catholic Church was exposed for hypocritical messaging: outwardly expressing that life is not to be enjoyed while their upper echelon lived luxuriously. The Church also sold indulgences, forgiveness of sins at a price. This was the source of fantastic wealth in the Catholic Church since it's hard to experiment with death and people like reassurance they're going to the right place: heaven. Such a practice was also completely unfounded in the Bible, and caused issue with the monk Martin Luther. Those who believed the Church had gone off script would be called Protestants. They would start their own family of faiths dedicated toward having a more personal relationship with God attained through reading and interpreting the Bible individually. In other words, they would reason their own faith. This movement was popular with the merchant class as it allowed individuals to interpret away their guilt; however, this movement was rather un-popular with the Catholic Church who saw it as a threat to their power. The Protestants and Catholics fought a series of brutal wars to convert one another ultimately ending in stalemate truce known as the Peace of Westphalia. Herein lies the origins of sovereignty; to prevent further war states were granted the right to determine their peoples religion and assured the protection of their self-determination by neighboring states.

The rulers of states now had a new goal: to turn their peoples into people. Just as when conquered people needed to be assimilated over time into a nation to ensure stability, people claimed by a ruler for the purpose of sovereignty did not always blend and needed transition time. The European Union is one such attempt at turning peoples into people. The idea that the great nations of Europe should come together and form one governing structure for the guarantee of sovereignty is an Enlightenment dream. Dreams, however, do not always come true. Assimilation to a common peoples may be the ideal of Enlightenment states, but it's not always the execution. This non-uniform assimilation is referred to as ``assimilation to'' or cultural violence: the elimination of particular culture in favor of another. This uneven assimilation is similar to the antithesis to an Enlightenment nation based off sharing an ethnicity or fraternity. The joining of peoples into people often involves the evolution of power, raising larger nations as people overcome differences and increase cooperation. The failure to do so, or the employment of cultural violence, devolves political power as oppressed peoples feel the need to protect their ways of life in smaller ethnically oriented nations. Thus confusion arises over these two types of nations: (1) a nation formed on ethnic lines and (2) a nation formed in a melting pot of peoples on the basis of reason. The European Union aims to promote type 2 nations by turning the European peoples into a people. It's clear, however, that this assimilation is not executed evenly and consequently promotes the devolution into type 1 nations.

Yugoslavia was a state created from smaller ethnic nations following WWI. During WWII this nation composed of very distinct ethnic groups would take different sides as the powers exerted influence. North Yugoslavia, which originated from the Austro-Hungarian empire with its Latin based language was swayed to the western cause by European powers. This divided these already distinct peoples which eventually lead to civil wars for independence. Eventually the nation of Yugoslavia was disassociated and its power devolved from a nation of many peoples to individual ethnic nations. These new nations, formed on the basis not of Enlightenment ideals but of ethnicity, would then have their sovereignty recognized by the European Union. Hence, a great confusion. The great Enlightenment project of Europe is now seen recognizing the devolution of power which stands against the European Unions very thesis of existence. At a time when the scars of cultural violence are fresh in the minds of many Europeans this sets a precedent that threatens the European Union. If the former Yugoslavians can split into a nation of their own people to protect their culture and identity then why can't other European nations? The situation is worsened by major European nations like Britain choosing to leave the union devolving power into yet another nation based more on ethnicity than Enlightenment ideals. The European Union furthers its own dissolution by failing to prevent these movements. Assimilating with people is difficult as people tend to focus on differences, it requires self-sacrifice and patience. Such sacrifice becomes considerably more difficult when neighbors are seen opting out with the recognition of the European Union. The simultaneous recognition of two different types of nations furthers an ideological momentum which seeks to devolve power and dissolve the European Union with its own permission.

\end{document}
