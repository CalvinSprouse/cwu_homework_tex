% LaTeX for GEOG101 Assignment 1
% declare document
\documentclass[a4paper, 12pt]{article}

% include the titling package
\usepackage{titling}
\usepackage{datetime}
\usepackage{authblk}
\usepackage{geometry}
\usepackage{setspace}
\usepackage[bottom]{footmisc}

% reduce page margins
\geometry{left=3cm, right=3cm}

% title information
\title{Peanuts are not a nut, they are legumes}
\author{Calvin Sprouse}
\affil{Geography 101 World Regional Geography}
\date{2024 March 13}

% configure the title block
\setlength{\droptitle}{-7em}
\pretitle{\begin{flushleft}\LARGE}
\posttitle{\par\end{flushleft}}
\preauthor{\begin{flushleft}\large}
\postauthor{\par\end{flushleft}}
\predate{\begin{flushleft}\large}
\postdate{\par\end{flushleft}}

% set spacing
\doublespacing%

% Answer the following question in essay form.  This is not a research paper.  The best answers will be those which combine class lecture notes and the textbook and other readings made available on CANVAS for the course.
% Do the best that you can without plagiarizing others and careful with grammar and spelling.  There are no tricks here. The point is to determine levels of comprehension of readings and lectures (1500 words, double spaced, is probably the minimum it would take to answer both questions properly)
%
% In a nutshell, why did I spend so much lecture time on the evolution of Western Civilization (Europe and European-derived) in this course on “World Regions”?
%
% In a much larger shell, describe in detail the two major phases of Western imperialism (what were Europeans after, what did they find, when and where?) and how each phase impacted different parts of the non-Western World in different ways (use detailed examples); impacts, indeed, that have shaped most of the contemporary characteristics of each “World Region” to this day, as summarized in your textbook.
% 1. mercantilism; 2. industrialization

\begin{document}
\maketitle

If one were to take a snapshot of the world in the 1400s and pick out the country most likely to take over the world, they would likely pick China. If one were to repeat this experiment in the 16th century, they might pick Spain or Portugal. Yet, one would be wrong on both accounts. If one were to take a snapshot of the world today, they would not see a global Spanish, Portuguese, or Chinese system. They would instead see a western system of Europe and European derived countries. They would see clear signatures of Europe in the global capitalist trend, the focus on individual military development, and the continual development of democratic systems. So in a nutshell, no other country or system impacted the world in the way Europe would. To understand the world today requires an understanding of how it was developed and the people who developed it.

The western world follows from the collapse of Rome, where people fled the cities to their own rural land holdings. This began the dark ages of feudal Europe and would set the stage for continual warfare. For the early Europeans this was fierce competition bread superior military power as polities had to advance to avoid being taken over. But military might alone does not make a country, yet alone people who will develop a global system. The Europeans would do something revolutionary for the time, they would let merchants be merchants relatively unimpeded. Every other country would treat these self-centered traders as the lowest ranks of society, including great countries like China. But with Europe being a mess of fractal polities, no one polity could control merchants and risk being left behind. So merchants ran wild. They began to build city states, free of the influence of polities and solidifying their role in the Europe to come. No other country had allowed these merchants to run wild before and the effect is profound. The world today is full of merchants, a staple of global western influence.

Europe would also, in the course of its development, arrive at a concept ubiquitous in the modern discourse of countries: sovereignty. During the European Renaissance a notion was developed by some disgruntled catholic monks. That was the idea that one could choose their faith and these monks began a protestant movement against the Catholic church. Consequently the Catholic people fought a series of brutal wars against the protestants. With the Peace of Westphalia these wars ended and sovereignty was established for countries such that their ruler could determine the religion of their people. The only way to enforce this was with the definition of borders. Thus the Enlightenment began with people being able to reason their choice of faith, or lack thereof, and borders were created solidifying the influence of rulers. Eventually reason would permeate into politics and Europe would propose nations built on the basis of rational thought, a sharp contrast from the typical nation founded on blood. Borders, sovereignty, capitalism, and governing bodies built on reason are not only staples of todays world but European Renaissance and Enlightenment concepts. Europe, indeed, changed the world.

The spread of European influence was not necessarily due to the quality of their ideas, but to the strength of their nations both militaristically and economically. Post-Enlightenment Europe had developed a military might to challenge most countries in the world, even the ancient China. To further their economic development Europe sought the rest of the world. They sent their merchants in droves in search of trading partners and stumbled upon undeveloped land rich in resources. The Spanish and Portuguese in particular found mountains of silver which could have greatly elevated their economic status. China, at the time, was not interested in European wares of any kind and would only accept silver thus making their commodities very valuable. This is when one might get the notion that Spain and Portugal could accelerate their development beyond even China. China, at the time, had developed a superiority complex upon realizing no other country could teach them anything. They burned their ships and required that any country wishing to trade with the great China must come to them. So the Spanish and Portuguese merchants did, but not under the flag of Spain and Portugal. Those two countries maintained some level of restriction on their merchant class. As it turns out restricting a class of self-serving traders when your neighbor does not tends not to end well. The Spanish and Portuguese merchants took those mountains of silver and traded them with other European countries who traded them to China. Thus the European economy grew tremendously.

The mountains of silver were not all that was found in South America, there was also fertile land for crops and an underdeveloped native population who could not resist the military might of Europe. Thus began European imperialism of these non-European countries. What would later become known as Brazil is one such coastal colony of Europe. The fertile land was ripe for cash crops which would make Europe a lot of money. But to get these cash crops out of the country Europe would need to change the way these lands were layed out. Cities were moved to the coast such that European ships could better access the products of the land. These coastal developments are primate cities and motivate the flow of resources out of the land and away from the working people. Europeans would take these raw resources and bring them back home to be processed into finished goods. Finished goods could then be sailed back to underdeveloped nations and sold in captive markets. Thus a population could be forced to extract valuable resources from their own land and be wholly dependent on external fabrication, never developing for their own self-sufficiency. This is the development of underdevelopment.

Europe does this in more than just South America; the land known today as India is another victim of European imperialism. When the East India Company took over the country they set out to create another colony to further European development. But India already had a thriving textile pre-industry at the time and was set to be the main textile producer for the world. This would have made the people of India wealthy and developed their country only a little behind European countries. So naturally Europe dismantled the industry, moved resources to coastal cities for export, and began processing raw cottons for resale back to India. This process is motivated by the merchants who make money on both legs of a journey either bringing resources from and bringing products to a country. Thus a local industry represents a threat to their profits even if they own said local industry. The Indian cities were even moved coastal to further this process thus taking a country which may have developed into a competitive modern nation and setting it back centuries.

These countries, subject to the throes of European mercantilism, could not even have the ``benefit'' of being able to produce their own resources and make money. Competition for resources drives merchants to develop and eventually develop tools to reduce the need for manual labor. These machines drive up production at the cost of allowing people to make a barely living wage. Thus these people flee to the coastal export cities leaving the interior of already underdeveloped countries to be occupied by resource producing machines and not people. This furthers the development of underdevelopment but lines the pockets of merchants. Now coastal primate cities are overpopulated and ripe for an ecological disaster, yet another consequence of European industrialization.

At the expense of the development of the rest of the world Europe exerted their imperial forces. This turned these young countries, which still had a chance to develop, into resource farms for Europe. They are forever delayed from reaching the level of civilization of the rest of the world as Europe fundamentally restructures them for export and squanders local industrial development. Even when freed these countries are geographically setup for export dependence and lack the internal development to create industry and land based trade with neighbors. Even worse is their geographical phenotype, primate coastal cities, are poised for ecologic disaster as European industrialization raises sea levels. Europe changed the world, not for the quality of their ideas, but for their willingness to do something different that accelerated them beyond the world and into the globe.

\end{document}
