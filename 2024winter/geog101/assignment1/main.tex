% LaTeX for GEOG101 Assignment 1
% declare document
\documentclass[a4paper, 12pt]{article}

% include the titling package
\usepackage{titling}
\usepackage{datetime}
\usepackage{authblk}
\usepackage{geometry}
\usepackage{setspace}

% reduce page margins
\geometry{left=3cm, right=3cm}

% title information
\title{Geography as a Holistic Science}
\author{Calvin Sprouse}
\affil{Geography 101 World Regional Geography}
\date{2024 January 22}

% configure the title block
\setlength{\droptitle}{-10em}
\pretitle{\begin{flushleft}\LARGE}
\posttitle{\par\end{flushleft}}
\preauthor{\begin{flushleft}\large}
\postauthor{\par\end{flushleft}}
\predate{\begin{flushleft}\large}
\postdate{\par\end{flushleft}}

% end preamble
\begin{document}

% print the title section
\maketitle

% As argued in class and in your readings, Geography should be practiced as a science “holistically.” Using examples, explain why this is the case and why, indeed, the science of Geography does not fit well within the disciplinary framework of modern higher education.
% 750 words about

\doublespacing\noindent
Why should Geography be practiced as a holistic rather than disciplinary science?
\vspace*{1\baselineskip}

Put succinctly, ``geographers do to space what historians do to time'' (Kevin Archer). It is well known that history can not be fully understood through just one disciplinary perspective. Geographers face a similar challenge in exploring the relationship humans have with space, their environment. We can begin to see this problem when we ask what happens when humans interact with their environment. An economist says they harvest resources. A cartographer says they make boundaries. A historian says they make history. Well duh, the historian elaborates to say they create structures of government and record events. The artist says they make art. What should the geographer say then? They say, succinctly, all of the above and that from all of the above emerges civilization. The story of how that happens requires, of course, a little bit of what everyone else said.

Consider the western Renaissance, what is it? Ask any one discipline and get one answer. But no single answer is satisfactory. We turn instead to the multi-disciplinary approach which is the realization that every single answer contributes to the story in some meaningful way. It seems natural to begin with the name, renaissance is French for rebirth. Rebirth of what is a natural second question but will require some more disciplines to help answer. Next is the historian, who says the western Renaissance occurs around the 14th century marking a transition out of the medieval period. It should be noted that the historian is already well versed in a multidisciplinary approach and will give a holistic answer. That answer will typically involve some notion of events causally preceding other events. A very useful perspective yet not quite satisfactory. An economist chimes in, in the early Renaissance European merchants went wild. They traded with seldom seen fervor and brutality. Some even took a looser definition of trading which today we call piracy. A hoplologists ears perk at the mention of piracy. They know European ships of the time presented a unique application of Chinese gunpowder, a technology previously only used in fireworks. All this talk of war and piracy inspires the artist to mention that such a dreary time was well represented by the artists living then. The sociologist points out the dreary was an intentional effect of religious doctrine of the time. An architect would speak to the increasing grandeur of the homes of ordinary citizens, a change from the typical contrast of hobbles to churches. The disciplines continue ad nauseam, even reaching the astronomer who describes how humans place in not just the world but the cosmos is finally realized in the relative motions of celestial bodies. Clearly, there is a lot to say.

As the squabble abates, the possibly overwhelmed geographer is left with a pile puzzle pieces. All eyes and ears turn to them now, the disciplines expect another contribution to this pool of knowledge. But the geographer, diabolical in their ways, does not intent to add another piece to the pile. The geographer sees the relationality of reality, the tabs and blanks of the pieces surely fit together in some way. The geographer sees an ordering to the pieces, a causal chain or hierarchy. They construct for the disciplines a story composed of the shared knowledge. The western Renaissance is the rebirth of cities, of city people. The ancient nations of Rome were the original great cities, hence rebirth. Their collapse fragmented the west in borders, cultures,  and developmental relationships. The church survived most unscathed and resumed its place at the head of these western fragments, though it too was not completely whole. These shards of the great cities fought over borders, ideas, and resources both internally and amongst each other. Each nation sought to superimpose their will on another, to secure surplus as if their life depended on it. That surplus could be used to make cities and those city people would make their nations greater than the others. The western medieval world was harsh, and bred fighting nations. As these nations grew their merchants retained this indomitable spirit. The church would lose their iron grip finger by finger. The merchants, bred from war, were let loose upon the nations of the east whose people were unprepared. Those same merchants brought their spoils home, not to share but to trade. They bought their way out of the hobbles and into the cities. When the cities weren't big enough they made them bigger. The great cities of the ancient world were reborn in the west but with a twist. The wealth used to build these new great cities was held in greater proportion by individuals, the merchants, not by the empires. These new city people did more art, more astronomy, more trade. The western Renaissance catapulted the squabbling European nations far above the world, straight to global supremacy.

The geographer has assembled their puzzle. A story built from the knowledge of disciplines and glued by causal relations and centered on humans interactions with their environments. The satisfactory answer is displayed for all to marvel. The disciplines naturally bicker about having left out this or that, they've focused on the `hole' aspect of holism and not the rest. Thinking holistically requires these sorts of compromises, seeing the forest for the trees necessitates some broad strokes. But in the process of studying the forest something is learned about the trees. The forest, as it turns out, is one big organism bound by relationships between fungi, flora, fauna, and of course trees. Just as civilizations, it turns out, are webs of relations between humans, their resources, their economies, their past, their faith, or holistically speaking, their environment. That is the story for the holistic geographer to tell.

\end{document}
