% LaTeX for GEOG101 Assignment 1
% declare document
\documentclass[a4paper, 12pt]{article}

% include the titling package
\usepackage{titling}
\usepackage{datetime}
\usepackage{authblk}
\usepackage{geometry}
\usepackage{setspace}
\usepackage[bottom]{footmisc}

% reduce page margins
\geometry{left=3cm, right=3cm}

% title information
\title{Power for me but not for thee (or us)}
\author{Calvin Sprouse}
\affil{Geography 101 World Regional Geography}
\date{2024 March 04}

% configure the title block
\setlength{\droptitle}{-7em}
\pretitle{\begin{flushleft}\LARGE}
\posttitle{\par\end{flushleft}}
\preauthor{\begin{flushleft}\large}
\postauthor{\par\end{flushleft}}
\predate{\begin{flushleft}\large}
\postdate{\par\end{flushleft}}

% set spacing
\doublespacing

% the prompt
% Answer the following question in essay form.  This is not a research paper.  The best answers will be those which combine class lecture notes and the textbook and other readings made available on CANVAS for the course.
% Do the best that you can without plagiarizing others and careful with grammar and spelling.  There are no tricks here. The point is to determine levels of comprehension of readings and lectures (750 words, double spaced, is probably the minimum it would take to answer properly)
% As discussed in class, even though Latin American countries achieved ‘independence” from European control in the early 1800s---that is, not many years after the American Revolution---these newly independent countries took a profoundly different social, political, economic-urban, and cultural development trajectory than newly independent North American colonies.
% Taking into consideration the imperial history of these former colonies, why is this the case and how has this different development trajectory resulted in most of the main characteristics of modern-day Latin America?

\begin{document}
\maketitle

% just jotting ideas here, do not submit this
% A big reason for this change is the independence was motivated by the local powerful players who elected to retain the social system, a top-down authoritarian system. This was independence without a revolution. These countries have never known a reformed social structure present in other colonies. A big reason for this was the lack of European inhabitants in all social levels of these countries during their journey to independence. In the American colonies, for example, Europeans occupied nearly every socio-economic level and thus fought to reform the social structure. IN latin countries Europeans occupied the highest levels of society and thus did not desire to change the social structure. Even moving forward these countries have experienced revolutions. but these revolutions only replace the power positions with new players, they do not reform the system. That is because these revolutions often occur as internal military coups which aim to move but not disperse power.

% begin
% The differences in the evolutionary trajectory of the Latin American colonies to the North American colonies can be explained primarily by the relationship the colonies had with Europe.
The countries that emerged from the Latin American and North American colonies differ vastly in global power, socio-economic structure, and political organization. This is all in spite of being former European colonies subject to heavy western influence and gaining independence from Europe at a similar time. These differences begin even before the colonies achieve independence and set the scene on which independence will be attained from Europe.

Before independence from Europe the North American and Latin American colonies served different purposes to European powers and express those phenotypically. The North American colonies, which become the modern day United States of America, were founded for settlement purposes. The settlers of these colonies were Europeans who left their country to find individual success. The colonies that sprung from these settlers were of European make. They were made by the people for the people. As these colonists developed their new home they paid taxes, tribute, to Europe and enjoyed the benefits of European technology throughout their development. These settlers created an egalitarian political structure and put systems in place for internal sustainment. They were built by, designed for, supported by, and lived in by Europeans.

The Latin American colonies were not. While the new world was a settlement the Latin American colonies were conquered. The Latin American colonies were created by conquering existing empires and placing them under European control. These conquered colonies were not for settlement and contained a minority of Europeans. What Europeans did live in these colonies were placed in positions of power with the objective of subjugating a population for the benefit of Europe. This objective forced a restructuring of the colonies. While the North American colonists designed their society for sustainment the Latin American conquerors designed a society for export production. Cities of the Latin American colonies were created on the coast as beacons for export, with little to no attention placed on connecting colonies by land. Valuable agricultural land was used not for the growth of crops necessary to sustain a population, but crops that would turn a profit in Europe. These colonies were organized in a top-down hierarchy, forcing a population of conquered peoples to work for the benefit of an external power.

It is already clear that these colonies will experience a different developmental track. Before either colony would gain independence their society has been fundamentally altered by European influence. The North American colonies created an egalitarian self-sustaining structure from the ground up. The Latin American colonies were subjugated by a minority of European settlers who restructured their society for the export production of profitable goods. Furthermore, the composition of these colonies was very different at the time of their revolutions. The export focus of Latin American colonies motivated the slave trade. Now the Latin American colonies are composed of a small minority of European leaders, subjugated native people, and displaced subjugated people. In fact the Latin American colonial powers preferred to replace the population of native peoples with enslaved peoples. Being displaced without language or strong social connections made rebellion harder to organize. Before these countries even began to separate from Europe they were on different footing.

Independence for the two colonies would be approached in very different ways, cementing the different developmental trajectories. The North American colonies, being egalitarian in structure and composed of European people in all levels of society, fought as one to create their egalitarian society. These people were products of the enlightenment and formed their newly independent thusly. When the Latin American colonies sought independence they did not desire to reform the social structure in the way the North American colonies had done by separating from Europe. The Latin American colonial revolt happened much the same way as anything in the Latin American colonies happened: top-down. The heads of these colonies, the Europeans put in place to control the populous, decided they would maintain their power without the responsibility to Europe. When independence was gained nothing changed to the vast majority of Latin Americans. They were still a subjugated people working for external export of profitable goods under a European authoritarian leadership. Only now that European leadership was pocketing more profit than before. Unlike the structure of power, top-down, the money in this newly independent society would remain at the top.

This sets the tone for the future development of these colonies and their peoples. The North American colonies built an enlightenment project based on their egalitarian, independent, western merchant mindset. Their enlightenment project lives off these ideals, with arguable success\footnote{For another day.}, to this day. The countries emergent from Latin American independence, on the other hand, continue the cycle of top-down authoritarian revolutions. Despite many changes of power, often violent, the power structure itself remains mostly unchanged. Each new leader assumes the seat of the prior with many vying for their turn. The types of revolutions these countries experience are organized not by the people, as in the North American revolution, but by the powers at be: the military, powerful internal organizations, sometimes external powers\footnote{See Chiquita, formerly the United Fruit Company.}. To this day the modern Latin American countries express the same traits as their colonial past. Their cities are coastal and serve as ports of export for a few primary goods to be taken to other countries. Their farmland is tied to this purpose through hereditary ownership and exploitation. Contrast to the United States of America with a great variety of coastal and internal cities and diverse farming for self sustainment.

\end{document}
