% LaTeX for GEOG101 Assignment 1
% declare document
\documentclass[a4paper, 12pt]{article}

% include the titling package
\usepackage{titling}
\usepackage{datetime}
\usepackage{authblk}
\usepackage{geometry}
\usepackage{setspace}

% reduce page margins
\geometry{left=3cm, right=3cm}

% title information
\title{Why the Western Renaissance}
\author{Calvin Sprouse}
\affil{Geography 101 World Regional Geography}
\date{2024 February 05}

% configure the title block
\setlength{\droptitle}{-10em}
\pretitle{\begin{flushleft}\LARGE}
\posttitle{\par\end{flushleft}}
\preauthor{\begin{flushleft}\large}
\postauthor{\par\end{flushleft}}
\predate{\begin{flushleft}\large}
\postdate{\par\end{flushleft}}

% end preamble
\begin{document}

% print the title section
\maketitle

% Explain, in your own words, why I have spent so much lecture time on the historical-geographic evolutions and cultural characteristics of the “Western Renaissance” in what is called “Europe” today in comparison to those of other regions of world?
% 750 words about

\doublespacing\noindent
Why has class content focused so much on the historical-geographic evolutions and cultural characteristics of the Western Renaissance in Europe in comparison to other regions of the world?
\vspace*{1\baselineskip}

Class content has focused primarily on the Western Renaissance in Europe because the outcome of this Western Renaissance is the world we live in today. Unique to every other world system before, the Western Renaissance created a self-driving force catapulting the western civilizations far beyond the worldly expanse of ancient empires like China and into the global system seen today. We focus on this sudden emergence of a global power because it is surprising. The Western Renaissance continues shape the world today in a way no other historical event does.

If one were to rank which empires would take over the world in 1492 the young European nations would not be high on the list. Without question China would be the safe bet. It is clear, however, that anyone betting on China in 1492 would be mildly displeased with their gamble today. China was the safe bet for many reasons. Of all the civilizations at the time China was the most ancient and had stabilized at the highest level of civilizational development. The intentional process of stabilization takes time, time the western nations just born of Roman ash could not even comprehend. Europe was, at the time, a fragmented collection of bickering neighbors. To great ancient nations like China they would seem little more than children playing in the sand. But these bickering children were learning something that thousands of years of stability would never have taught: ruthless fighting. The fragments of Rome fought for survival on every field. Their minds and money made weapons and their weapons made them dangerous. A stable nation has no need for weapons, but the fierce competition to be the conquerer and not conquered drove the west to the forefront of warfare.

Weapons and war, while important, do not guarantee a spot at the global table. The west had something far more dangerous in store, a Pandoras box for the globe. The power vacuum left behind by Rome set loose a beast that had been shackled for as long as humanity could remember. The merchants went wild. Every civilization before the Western Renaissance kept these self-indulgent beings under heavy regulation. A merchant represented the ultimate threat to power because they could not only buy power, but in their own self-interest would use that power to buy more power. Once the merchant was set free there would be no turning back. Before the world could gasp the merchants bought their own cities, they became masters of their own fate as free people. These city states became a nexus for Renaissance ideas and western development.

The greedy grabby west, driven by their merchants, began to reach far beyond continental Europe. Everywhere they went merchants pulled a line tying up the ends of the world in a great spiders web. They held council to divide foreign lands of Africa and the Americas for the creation of more wealth. These conquered lands and peoples became a part of the western civilization weather they wanted to or not. When the merchants had purchased enough power and their world was weaved tighter they challenged the church. The idea that one should remain in their place was repulsive to the merchant who served themselves above all else. So it went. Following the protestant reformists the west fragmented into Christian and Catholic. The neighbors fought once again but this time not as children in the sandbox, as adults who invested time and money just to think of the next great way to end lives. When enough was enough the west got together to end all war. They created borders to separate Christian from Catholic and declared that these independent nations would have self-determination and would defend the same rights for their neighbors.

China scoffed at the barbarians. They, after all, still saw them as children to their time scale. Their civilizations were shallow. So China retreated within, for the world had nothing left to offer. The great western machine roared onwards. With all the noise of inventing, fighting, trading, building, and planning, it seems someone forgot to check on the driver. The strength of the Western Renaissance is not that they let their merchants go wild nor that they formulated a power structure on the basis of reason over blood. It's that they set into motion something that could not, and would not, be stopped. The Western Renaissance is an unchecked force accelerating the development of civilization in a way never before seen. When China grows or stabilizes it's because they chose too. But the west has no breaks, no steering wheel. The western civilizations rocket onwards, driven by an unseen force to grow. We study the Western Renaissance in so much detail because it has shaped the world we live in and changed the course of global affairs.

\end{document}
