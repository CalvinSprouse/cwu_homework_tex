% document style header
\documentclass[a4paper, 12pt]{config/homework}

% import default packages
\usepackage{config/packages}
\usepackage{config/commands}

% end preamble
\begin{document}

% document title
\noindent
Calvin Sprouse \hfill PHYS 342 Homework 4 \hfill 2024 May 3

% homework problems begin
% Textbook problems 2.29, 2.36, 3.1, 3.9, 3.11, 3.13, 3.31, 3.32.
\bigskip\noindent
\textbf{Problem 2.29.} Consider a system of two Einstein solids, with \(N_A=300\), \(N_B=200\), and \(q=100\). Compute the entropy of the most likely macrostate. Also compute the entropy over long time scales, assuming that all macrostates are accessible. Neglect the factor of Boltzmann's constant in the definition of entropy; for systems this small, it is best to think of entropy as a pure number.

\noindent
The most likely macrostate is the macrostate with the highest multiplicity. The total multiplicity, \(\Omega\), is given by
\[\Omega = \Omega_A \Omega_B,\]
where \(\Omega_A\) and \(\Omega_B\) are the multiplicities of solid \(A\) and solid \(B\) respectively. The multiplicity of an Einstein solid is given by
\[\Omega_A(N_A,q_A) = \frac{\left(q_A+N_A-1\right)!}{q_A!\left(N_A-1\right)!}.\]
For solid \(A\) and solid \(B\) we have
\[\Omega_A(N_A, q_A) = \frac{\left(q_A + 299\right)!}{q_A!299!}, \qquad
\Omega_B(N_B, q_B) = \frac{\left(q_B + 199\right)!}{q_B!199!}.\]
Notice, \(q_B = q - q_A\). Then,
\[\Omega_B(N_B, q_B) = \Omega_B(N_B, q-q_A) = \frac{\left(299 - q_A\right)!}{(100-q_A)!199!}.\]
We shall take for granted that \(\Omega\) is maximized for \(\Omega_A = \Omega_B\). Then,
\[\frac{\left(q_A + 299\right)!}{q_A!299!} = \frac{\left(299 - q_A\right)!}{(100-q_A)!199!}
\Rightarrow
\frac{\left(299 + q_A\right)!}{\left(299 - q_A\right)!} \frac{\left(100 - q_A\right)!}{q_A!}
= \frac{299!}{199!}.\]
This expression not only gave me a headache, but also crashed Mathematica. I will, begrudgingly, take the solution provided by Table 3.1; that is, the most likely macrostate, the one with the highest multiplicity, is found when \(q_A=60\). Then, by Table 3.1,
\[\frac{S_{\text{max}}}{k} = \qty{264.4}{}.\]
Over long timescales, \(q_A\) may be found at any value between 0 and 100. Then,
\[\frac{S_{\text{long}}}{k} = \ln\left(\sum_{q_A=0}^{100}\Omega_A(N_A,q_A)\Omega_B(N_B,100-q_A)\right) \approx 267.\]

% \bigskip
\pagebreak
\noindent
\textbf{Problem 2.36.} For either a monoatomic ideal gas or a high-temperature Einstein solid, the entropy is given by \(Nk\) times some logarithm. The logarithm is never large, so if all you want is an order-of-magnitude estimate, you can neglect it: \(S \sim Nk\). That is, the entropy in fundamental units is on the order of the number of particles in the system. This conclusion turns out to be true for most systems, with some important exceptions at low temperatures where the particles are behaving in an orderly way. Make a very rough estimate of the entropy of each of the following: (i) this book (a kilogram of carbon compounds); (ii) a moose (\(\qty{400}{\kg}\) of water); the sun (\(\qty{2e30}{\kg}\) of ionized hydrogen).

\noindent
In general, the number of molecules in a substance with a given mass and known composition is given by
\[N = \left(\text{mass}\right)\left(\text{molar mass}\right)\left(A_N\right),\]
where \(A_N=\qty{6.022e23}{\per\mol}\) is Avogadros number. Since we are taking the order of magnitude approach,
\[Nk \sim \mathcal{O}(\text{mass})\mathcal{O}(\text{molar mass})^{-1}\left(10^{23}\right)\left(10^{-23}\right) = \frac{\mathcal{O}(\text{mass})}{\mathcal{O}(\text{molar mass})}.\]
\begin{enumerate}[label=(\roman*)]
\item For this book, \(\mathcal{O}(\text{mass}) = 10^0\) and \(\mathcal{O}(\text{molar mass})=10^{-3}\). Then, \[S\sim\qty{e3}{\joule\per\kelvin}.\]

\item For a moose, \(\text{H}_2\text{O}\), \(\mathcal{O}(\text{mass})=10^2\) and \(\mathcal{O}(\text{molar mass})=10^{-2}\). Then, \[S\sim\qty{e4}{\joule\per\kelvin}.\]

\item For the Sun, \(\mathcal{O}(\text{mass})=10^{30}\) and \(\mathcal{O}(\text{molar mass})=10^{-3}\). Then, \[S\sim\qty{e33}{\joule\per\kelvin}.\]

\end{enumerate}

\pagebreak
\noindent
\textbf{Problem 3.1.} Use Table 3.1 to compute the temperatures of solid \(A\) and solid \(B\) when \(q_A=1\). Then, compute both temperatures when \(q_A=60\). Express the solutions in terms of \(\epsilon/k\), and then in kelvins assuming that \(\epsilon=\qty{0.1}{\eV}\).

\noindent
The temperature of solid \(A\) is given by Equation 3.6 to be
\[T_A = \left(\diffp{U}{S}\right)_{N,V}
= \frac{(q_A + 1)\epsilon - (q_A - 1)\epsilon}{\left(\frac{S_A}{k}\right)_{q_A+1}k - \left(\frac{S_A}{k}\right)_{q_A-1}k}
= \frac{2}{\left(\frac{S_A}{k}\right)_{q_A+1}k - \left(\frac{S_A}{k}\right)_{q_A-1}}\frac{\epsilon}{k},\]
where \(k\) is the Boltzmann constant expressed in units of electron-volts. The temperature of solid \(B\) is given in a similar form to be
\[T_B =\frac{2}{\left(\frac{S_B}{k}\right)_{q_A+1}k - \left(\frac{S_B}{k}\right)_{q_A-1}}\frac{\epsilon}{k}.\]
The values of \(S_A/k\) and \(S_B/k\) for some value of \(q_A\) can be read from Table 3.1.

\noindent
Let \(q_A = 1\). Then,
\begin{align*}
T_A &= \frac{2}{10.7 - 0}\frac{\epsilon}{k} = \qty{216.9}{\kelvin}, \\
T_B &= \frac{2}{187.5-185.3}\frac{\epsilon}{k} = \qty{1055}{\kelvin}.
\end{align*}

\noindent
Let \(q_A=60\). Then,
\begin{align*}
T_A &= \frac{2}{160.9 - 157.4}\frac{\epsilon}{k} = \qty{663.1}{\kelvin}, \\
T_B &= \frac{2}{107.0 - 103.5}\frac{\epsilon}{k} = \qty{34.90}{\kelvin}.
\end{align*}


\bigskip
\noindent
\textbf{Problem 3.9.} In solid carbon monoxide, each CO molecule has two possible orientations: CO or OC. Assuming that these orientations are completely random (which is not quite true but close), calculate the residual entropy of a mole of carbon monoxide.

\noindent
Since there are two states we can think of the residual state of solid CO as \(N=N_A\) coins. The multiplicity of \(N\) coins is \(2^N\). Then,
\[S = k\ln\left(2^N\right) = N_A k \ln(2),\]
where \(N_A\) is Avogadros number.


% \bigskip
\pagebreak
\noindent
\textbf{Problem 3.11.} In order to take a nice warm bath, 50 litres of water at \(\qty{55}{\celsius}\) is mixed with 25 litres of water at \(\qty{10}{\celsius}\). How much new entropy has been generated by mixing the water?

\noindent
The total change in entropy is given by
\[\Delta S_\text{tot} = \Delta S_h + \Delta S_c.\]
The change in entropy of the hot water is given by Equation 3.19 to be
\[\Delta S_h = \defint{T_{ih}}{T_f}{\frac{C_V}{T}}{T} = m_h c_w \defint{T_{ih}}{T_f}{T^{-1}}{T} = m_h c_w\ln\left(\frac{T_f}{T_{ih}}\right),\]
where \(c_w\) is the specific heat capacity of water. The change in entropy of the cold water follows a similar form:
\[\Delta S_c = m_c c_w \ln\left(\frac{T_f}{T_{ic}}\right).\]
Then,
\[S_\text{tot} = c_w \left(m_h \ln\left(\frac{T_f}{T_{ih}}\right) + m_c \ln\left(\frac{T_f}{T_{ic}}\right)\right).\]
The mass of water can be found from the volume by the density of water
\[m = vD,\]
where \(D=\qty{1000}{\gram\per\liter}\). The final temperature of the combination can be found using an equation derived in Homework 2 Problem 1.42:
\[T_f = \frac{T_{ic}m_c + T_{ih}m_h}{m_c + m_h},\]
where we assumed the specific heat capacity of the hot water is equal to the specific heat capacity of the cold water. Substituting provided values yields
\[\Delta S = \qty{770}{\joule\per\kelvin}.\]


% \bigskip
\pagebreak
\noindent
\textbf{Problem 3.13.} When the sun is high in the sky, it delivers approximately 1000 watts of power to each square meter of earth's surface. The temperature of the surface of the sun is about \(\qty{6000}{\kelvin}\), while that of the earth is about \(\qty{300}{\kelvin}\).
\begin{enumerate}[label=\textbf{(\alph*)}]
\item Estimate the entropy created in one year by the flow of solar heat onto a square meter of the earth.
\item Suppose grass is planted on the square meter of earth under investigation in part \textbf{(a)}. Some people might argue that the growth of the grass (or of any other living thing) violates the second law of thermodynamics, because disorderly nutrients are converted into an orderly life form. How should one respond to this argument?
\end{enumerate}
\begin{enumerate}[label=\textbf{(\alph*)}]
\item Suppose the Sun spends 6 hours a day ``high in the sky'' with respect to a particular square meter of Earth. Then, the amount of heat energy, \(Q\), delivered to that particular square meter is given by dimensional analysis to be
\[Q = \left(\qty[per-mode=fraction]{1000}{\joule\per\second\per\meter\squared}\right)\left(\qty[per-mode=fraction]{3600}{\second\per\hour}\right)\left(\qty[per-mode=fraction]{6}{\hour\per\day}\right)\left(\qty[per-mode=fraction]{365}{\day\per\year}\right)\left(\qty[per-mode=fraction]{1}{\year}\right)\left(\qty[per-mode=fraction]{1}{\meter\squared}\right)= \qty{7.884e9}{\joule}.\]
The change in entropy of the surface of the Earth due to a gain in heat energy is given by Equation 3.17 to be
\[\Delta S = \frac{Q}{T}.\]
Then,
\[\Delta S_{\earth} = \frac{Q}{T_{\earth}} = \frac{\qty{7.884e9}{\joule}}{\qty{300}{\kelvin}} = \qty[per-mode=fraction]{2.628e7}{\joule\per\kelvin}.\]
The change in entropy of the Sun due to the loss of this heat energy is given by
\[\Delta S_{\astrosun} = -\frac{Q}{T_{\astrosun}} = -\frac{Q}{20 T_{\earth}} = -\frac{\Delta S_{\earth}}{20}.\]
The total entropy of the universe changes by an amount
\[\Delta S = \Delta S_{\astrosun} + \Delta S_{\earth} = \frac{19}{20}\Delta S_{\earth}.\]
\item In Problem 2.36 we stated that entropy is on an order-of-magnitude of \(Nk\).
\end{enumerate}


% \bigskip
\pagebreak
\noindent
\textbf{Problem 3.31.} Experimental measurements of heat capacities are often represented in reference works as empirical formulas. For graphite, a formula that works well over a fairly wide range of temperatures is, for one mole,
\[C_P = a + bT - \frac{c}{T^2},\]
where \(a=\qty{16.86}{\joule\per\kelvin}\), \(b=\qty{4.77e-3}{\joule\per\kelvin\squared}\), and \(c=\qty{8.54e5}{\joule\kelvin}\). Suppose, then, that a mole of graphite is heated at constant pressure from \(\qty{298}{\kelvin}\) to \(\qty{500}{\kelvin}\). Calculate the increase in its entropy during this process. Add on the tabulated value of \(S(\qty{298}{\kelvin})\) from the back of this book to obtain \(S(\qty{500}{\kelvin})\).

\noindent
The change in entropy is given by Equation 3.17 to be
\[\Delta S = \defint{T_i}{T_f}{\frac{C_P}{T}}{T}.\]
Then,
\[\Delta S = a\ln\left(\frac{T_f}{T_i}\right) + b\left(T_f - T_i\right) + \frac{c}{2}\left(T^{-2}_f - T^{-2}_i\right).\]
Substituting given values,
\[\Delta S = \qty{6.59}{\joule}.\]


% \bigskip
\pagebreak
\noindent
\textbf{Problem 3.32.} A cylinder contains one liter of air at room temperature, \(\qty{300}{\kelvin}\), and atmospheric pressure, \(\qty{e5}{\Pa}\). At one end of the cylinder is a massless piston, whose surface area is \(\qty{0.01}{\meter\squared}\). Suppose that the piston is pushed in very suddenly, exerting a force of \(\qty{2000}{\newton}\). The piston moves only one millimeter, before it is stopped by an immovable barrier of some sort.
\begin{enumerate}[label=\textbf{(\alph*)}]
\item How much work has been done on this system?
\item How much heat has been added to the gas?
\item Assuming that all the energy added goes into the gas (not the piston or the cylinder walls), by how much does the energy of the gas increase?
\item Use the thermodynamic identity to calculate the change in the entropy of the gas once it has again reached equilibrium.
\end{enumerate}
\bigskip
\begin{enumerate}[label=\textbf{(\alph*)}]
\item The work done on the system, \(W\), can be found by
\[W = Fd,\]
where \(F=\qty{2000}{\newton}\) is the force and \(d=\qty{1}{\milli\meter}\) is the displacement. Then,
\[W = \left(\qty{2000}{\newton}\right)\left(\qty{1e-3}{\milli\meter}\right) = \qty{2}{\joule}.\]
\item No heat has been added to the gas. The only energy that has been put into the system has been in the form of mechanical compressive work.
\item The first law of thermodynamics states
\[\Delta U = Q + W.\]
From part \textbf{(a)}, \(W=\qty{2}{\joule}\), and from part \textbf{(b)}, \(Q=0\). Then,
\[\Delta U = \qty{2}{\joule}.\]
\item The thermodynamic identity is given by Equation 3.46 to be
\[\text{d}U = T\text{d}S - P\text{d}V.\]
Rearranging and taking the difference interpretation of the derivative yields
\[\Delta S = \frac{P}{T}\Delta V + \frac{1}{T}\Delta U.\]
Then,
\[\Delta S = \frac{\qty{e6}{\Pa}}{\qty{300}{\kelvin}}\left(\qty{e-2}{\meter\squared}\right)\left(-\qty{e-3}{\meter}\right) + \frac{1}{\qty{300}{\kelvin}}\left(\qty{2}{\joule}\right)
= \qty[per-mode=fraction, parse-numbers=false]{\frac{1}{300}}{\joule\per\kelvin}.\]
\end{enumerate}
\end{document}
