% document style header
\documentclass[a4paper, 12pt]{config/homework}

% import default packages
\usepackage{config/packages}
\usepackage{config/commands}

% end preamble
\begin{document}

% document title
\noindent
\hfill Calvin Sprouse \hfill PHYS 342 Homework 4 \hfill 2024 May 3 \hfill

% homework problems begin
% Textbook problems 2.29, 2.36, 3.1, 3.9, 3.11, 3.13, 3.31, 3.32.
\bigskip\noindent
\textbf{Problem 2.29.} Consider a system of two Einstein solids, with \(N_A=300\), \(N_B=200\), and \(q=100\). Compute the entropy of the most likely macrostate. Also compute the entropy over long time scales, assuming that all macrostates are accessible. Neglect the factor of Boltzmann's constant in the definition of entropy; for systems this small, it is best to think of entropy as a pure number.

\noindent
Response.

% \bigskip
\pagebreak
\noindent
\textbf{Problem 2.36.} For either a monoatomic ideal gas or a high-temperature Einstein solid, the entropy is given by \(Nk\) times some logarithm. The logarithm is never large, so if all you want is an order-of-magnitude estimate, you can neglect it: \(S \sim Nk\). That is, the entropy in fundamental units is on the order of the number of particles in the system. This conclusion turns out to be true for most systems, with some important exceptions at low temperatures where the particles are behaving in an orderly way. Make a very rough estimate of the entropy of each of the following: (i) this book (a kilogram of carbon compounds); (ii) a moose (\(\qty{400}{\kg}\) of water); the sun (\(\qty{2e30}{\kg}\) of ionized hydrogen).

\noindent
In general, the number of molecules in a substance with a given mass and known composition is given by
\[N = \left(\text{mass}\right)\left(\text{molar mass}\right)^{-1}\left(A_N\right),\]
where \(A_N=\qty{6.022e23}{\per\mol}\) is Avogadros number. Since atomic masses are often reported in atomic mass units this will often be multiplied by a conversion factor \\\(\qty{1.66e-27}{\kg}\). Since we are taking the order of magnitude approach,
\[Nk \sim \mathcal{O}(\text{mass})\left(\mathcal{O}(\text{molar mass})\times 10^{-27}\right)^{-1}10^{23}\times 10^{-23}.\]
\begin{enumerate}[label=(\roman*)]
\item For this book, \(\mathcal{O}(\text{molar mass})=10^1\) and \(\mathcal{O}(\text{mass})=10^0\). Then,
\[S \sim 10^0 \left(10^1 \times 10^{-27}\right)^{-1} = 10^{26}.\]
\item For a moose, \(\text{H}_2\text{O}\), \(\mathcal{O}(\text{molar mass})=10^1\) and \(\mathcal{O}(\text{mass})=10^2\). Then,
\[S \sim 10^1 \times \left(10^2 \times 10^{-27}\right)^{-1} = 10^{26}.\]
\item For the sun, \(\mathcal{O}(\text{molar mass})=10^0\) and \(\mathcal{O}(\text{mass})=10^{30}\). Then,
\[S \sim 10^{30} \left(10^1 \times 10^{-27}\right)^{-1} = 10^{56}.\]
\end{enumerate}

\pagebreak
\noindent
\textbf{Problem 3.1.} Use Table 3.1 to compute the temperatures of solid \(A\) and solid \(B\) when \(q_A=1\). Then, compute both temperatures when \(q_A=60\). Express the solutions in terms of \(\epsilon/k\), and then in kelvins assuming that \(\epsilon=\qty{0.1}{\eV}\).

\noindent
Response


\bigskip
\noindent
\textbf{Problem 3.9.} In solid carbon monoxide, each CO molecule has two possible orientations: CO or OC. Assuming that these orientations are completely random (which is not quite true but close), calculate the residual entropy of a mole of carbon monoxide.

\noindent
Response


\bigskip
\noindent
\textbf{Problem 3.11.} In order to take a nice warm bath, 50 litres of water at \(\qty{55}{\celsius}\) is mixed with 25 litres of water at \(\qty{10}{\celsius}\). How much new entropy has been generated by mixing the water?

\noindent
Response.


\bigskip
\noindent
\textbf{Problem 3.13.} When the sun is high in the sky, it delivers approximately 1000 watts of power to each square meter of earth's surface. The temperature of the surface of the sun is about \(\qty{6000}{\kelvin}\), while that of the earth is about \(\qty{300}{\kelvin}\).
\begin{enumerate}[label=\textbf{(\alph*)}]
\item Estimate the entropy created in one year by the flow of solar heat onto a square meter of the earth.
\item Suppose grass is planted on the square meter of earth under investigation in part \textbf{(a)}. Some people might argue that the growth of the grass (or of any other living thing) violates the second law of thermodynamics, because disorderly nutrients are converted into an orderly life form. How should one respond to this argument?
\end{enumerate}
\begin{enumerate}[label=\textbf{(\alph*)}]
\item 
\end{enumerate}


\bigskip
\noindent
\textbf{Problem 3.31.} Experimental measurements of heat capacities are often represented in reference works as empirical formulas. For graphite, a formula that works well over a fairly wide range of temperatures is, for one mole,
\[C_P = a + bT - \frac{c}{T^2},\]
where \(a=\qty{16.86}{\joule\per\kelvin}\), \(b=\qty{4.77e-3}{\joule\per\kelvin\squared}\), and \(c=\qty{8.54e5}{\joule\kelvin}\). Suppose, then, that a mole of graphite is heated at constant pressure from \(\qty{298}{\kelvin}\) to \(\qty{500}{\kelvin}\). Calculate the increase in its entropy during this process. Add on the tabulated value of \(S(\qty{298}{\kelvin})\) from the back of this book to obtain \(S(\qty{500}{\kelvin})\).

\noindent
Response.


\bigskip
\noindent
\textbf{Problem 3.32.} A cylinder contains one liter of air at room temperature, \(\qty{300}{\kelvin}\), and atmospheric pressure, \(\qty{10e5}{\Pa}\). At one end of the cylinder is a massless piston, whose surface area is \(\qty{0.01}{\meter\squared}\). Suppose that the piston is pushed in very suddenly, exerting a force of \(\qty{2000}{\newton}\). The piston moves only one millimeter, before it is stopped by an immovable barrier of some sort.
\begin{enumerate}[label=\textbf{(\alph*)}]
\item How much work has been done on this system?
\item How much heat has been added to the gas?
\item Assuming that all the energy added goes into the gas (not the piston or the cylinder walls), by how much does the energy of the gas increase?
\item Use the thermodynamic identity to calculate the change in the entropy of the gas once it has again reached equilibrium.
\end{enumerate}
\begin{enumerate}[label=\textbf{(\alph*)}]
\item 
\end{enumerate}
\end{document}
