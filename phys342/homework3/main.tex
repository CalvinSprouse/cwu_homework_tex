% document style header
\documentclass[a4paper, 12pt]{config/homework}

% import default packages
\usepackage{config/packages}
\usepackage{config/commands}

% end preamble
\begin{document}

% document title
\noindent
\hfill Calvin Sprouse \hfill PHYS 342 Homework 3 \hfill 2024 April 19 \hfill

% homework problems begin
% Textbook problems 2.1, 2.4, 2.5, 2.8, 2.26
\bigskip\noindent
\textbf{Problem 2.1:} Suppose you flip four fair coins.
\begin{enumerate}[label=\textbf{(\alph*)}]
\item Make a list of all possible outcomes. \\
\bigskip\noindent
The set of all possible outcomes, \(S\), is
\begin{enumerate}[label=\(S_{\arabic*}\)]
\item \(=\{H,H,H,H\}\),
\item \(=\{H,H,H,T\}\),
\item \(=\{H,H,T,H\}\),
\item \(=\{H,T,H,H\}\),
\item \(=\{T,H,H,H\}\),
\item \(=\{H,H,T,T\}\),
\item \(=\{H,T,H,T\}\),
\item \(=\{H,T,T,H\}\),
\item \(=\{T,H,T,H\}\),
\item \(=\{T,T,H,H\}\),
\item \(=\{T,H,H,T\}\),
\item \(=\{H,T,T,T\}\),
\item \(=\{T,H,T,T\}\),
\item \(=\{T,T,H,T\}\),
\item \(=\{T,T,T,H\}\),
\item \(=\{T,T,T,T\}\).
\end{enumerate}
\pagebreak
\item Make a list of all the different ``macrostates'' and their probabilities.
The set of all possible macrostates, \(p_n\) where \(p\) denotes the probability of finding \(n\) heads, is
\begin{align*}
p_0 &= 1/16, \\
p_1 &= 4/16, \\
p_2 &= 6/16, \\
p_3 &= 4/16, \\
p_4 &= 1/16.
\end{align*}
\item Compute the multiplicity of each macrostate using the combinatorial formula 2.6, and check that these results agree with brute force counting.
\bigskip\noindent
The combinatorial formula 2.6 states
\[\Omega(N,n)=\frac{N!}{n!(N-n)!},\]
where \(N\) is the number of objects, in this case \(N=4\) coins, and \(n\) is macrostate to count, in this case \(n\) is the number of heads. Then,
\begin{align*}
\Omega(4,0) &= 1,\\
\Omega(4,1) &= 4,\\
\Omega(4,2) &= 6,\\
\Omega(4,3) &= 4,\\
\Omega(4,4) &= 1.
\end{align*}
The numerator of \(p_n\) is the number of microstates for a designated macrostates. With this in mind, the combinatorial formula agrees with brute force counting.
\end{enumerate}

\pagebreak\noindent
\textbf{Problem 2.4:} Calculate the number of possible five-card poker hands, dealt from a deck of 52 cards. A royal flush consists of the five highest-ranking cards of any one of the four suits. What is the probability of being dealt a royal flush?
\bigskip\noindent\\
The number of ways to combine 52 cards into groups of 5 without replacement is given by the permutation formula,
\[P(N,n)=\frac{N!}{(N-n)!},\]
where \(N=52\) and \(n=5\). Then,
\[P(52,5) = \frac{52!}{(52-5)!} = \qty{6.4974e+6}{}.\]
For the probability of being dealt a royal flush we consider a probability tree. Suppose there are no other players, they would add complications at each step of the tree. Then, at the outset, there are 20 cards which correspond to a royal flush available to draw; that is, the probability of moving past node one is \(20/52\). Now, the player has drawn some card of a given suit and so is restricted to drawing cards from the same suit. There remain four cards of the same suit that the player must draw to obtain a royal flush. Thus nodes two, three, four, and five, have probabilities \(4/51\), \(3/50\), \(2/49\), \(1/48\) respectively. Here we see that the first node had probability \(4\times 5\), representing the four suits. So, the probability of drawing a royal flush from a specific suit is given by \(5!/(52!/(52-5)!)\). The probability of drawing a royal flush of any suit can be found by adding each suits royal flush probability together. That is,
\[p = 4\frac{5!(52-5)!}{52!} \approx \qty{1.539e-6}{}.\]
Notably, the probability of drawing a royal flush of a specific suit is a similar form as the inverse of the multiplicity formula.


% \pagebreak\noindent
% \textbf{Problem 2.5:} For an einstein solid with each of the following values of \(N\) and \(q\), list all of the possible microstates, count them, and verify formula 2.9.
% \begin{enumerate}[label=\textbf{(\alph*)}]
% \item \(N=3\), \(q=4\)
% \item \(N=3\), \(q=5\)
% \item \(N=3\), \(q=6\)
% \item \(N=4\), \(q=2\)
% \item \(N=4\), \(q=3\)
% \item \(N=1\), \(q=\text{anything}\)
% \item \(N=\text{anything}\), \(q=1\)
% \end{enumerate}
% \bigskip\noindent


\pagebreak\noindent
\textbf{Problem 2.8:} Consider a system of two Einstein solids, \(A\) and \(B\), each containing \(N=10\) oscillators, sharing a total of \(q=20\) units of energy. Assume the solids are weakly coupled, and that the total energy is fixed.
\begin{enumerate}[label=\textbf{(\alph*)}]
\item How many different macrostates are available to the system?\\\noindent
There exist \(q+1=21\) different macrostates of the system.

\item How many different microstates are available to the system?\\\noindent
By Eq.\ 2.9,
\[\Omega(N,q)=\frac{(N+q-1)!}{q!(N-1)!}\approx\qty{6.9e10}{}.\]


\item Assuming that this system is in thermal equilibrium, what is the probability of finding all the energy in solid \(A\)?\\
\bigskip\noindent

The configuration corresponding to finding all the energy in \(A\) has a number of microstates given by
\[\Omega(N_A,q)\Omega(N_B,0).\]
Then the probability is given by the number of microstates for this configuration over the total number of microstates which are equally likely:
\[P = \frac{\Omega(N_A,q)\Omega(N_B,0)}{\Omega(N,q)} \approx \qty{1.5e-4}{}.\]


\item What is the probability of finding exactly half of the energy in solid \(A\)?\\\noindent
By similar reasoning above, this configuration has multiplicity
\[\Omega(N_A,q/2)\Omega(N_B,q/2).\]
Then the probability is given similarly to be
\[P = \frac{\Omega(N_A,q/2)\Omega(N_B,q/2)}{\Omega(N,q)}\approx\qty{0.12}{}.\]


\item Under what circumstances would this system exhibit irreversible behavior? \\\noindent
For this system, irreversible behavior corresponds to a continuous trend towards decreasing entropy. For example, should the system go from the state described in part \textbf{(d)} to the state described in part \textbf{(c)}, that would be described as irreversible behavior.


\end{enumerate}
\pagebreak\noindent
\textbf{Problem 2.26:} Consider an ideal monoatomic gas that lives in a two-dimensional universe, occupying an area \(A\). Find a formula for the multiplicity of this gas, analogous to Equation 2.40. \\ \noindent
We begin with Equation 2.29 expressed in terms of area rather than volume:
\[\Omega_1 = AA_p.\]
Using the same reasoning to arrive at Equation 2.30 we say our total energy \(U\) defines a momentum circle in 2-dimensional momentum space; that is,
\[U = \frac{1}{2m}\left(p_x^2 + p_y^2\right).\]
Thus, we have a circle of radius \(\sqrt{2mU}\). Then the area of momentum space is really the circumference of this circle. We invoke the Heisenberg uncertainty principle to obtain the same result as in Equation 2.33 and thus
\[\Omega_1 = \frac{A A_p}{h^2}.\]
For a two particle system we find the circumference of a circle in 4-dimensional space such that
\[\Omega_2 = \frac{A^2}{h^4}\times\left(\text{circumference of momentum hypercircle}\right).\]
Then applying the indisinguishability correction,
\[\Omega_2 = \frac{1}{2}\frac{A^2}{h^4}\times\left(\text{circumference of momentum hypercircle}\right).\]
We generalize to \(N\) particles in the same way as in Equation 2.38 to obtain
\[\Omega_N = \frac{1}{N!}\left(\frac{A}{h^2}\right)^N \times\left(\text{circumference of momentum hypercircle}\right).\]
For this calculation we can take Equation 2.39 and substitute \(d=2N\) instead of \(d=3N\). This recovers the expected behavior for a single particle: the circumference of a circle. Then,
\[\Omega_N = \frac{1}{N!}\left(\frac{A}{h^2}\right)^N \frac{2\pi^{2N/2}}{\left(\frac{2N}{2}-1\right)!} \left(2mU\right)^{\frac{2N-1}{2}}.\]
Reducing the fractions and assuming that \(N\gg 1\),
\[\Omega_N \approx \frac{\pi^N}{(N!)^2}\frac{A^N}{h^{2N}}\left(2mU\right)^{N}.\]

\end{document}
