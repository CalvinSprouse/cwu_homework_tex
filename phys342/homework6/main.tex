% document style header
\documentclass[a4paper, 12pt]{config/homework}

% import default packages
\usepackage{config/packages}
\usepackage{config/commands}

% end preamble
\begin{document}

% document title
\noindent
Calvin Sprouse \hfill PHYS342 Homework 6 \hfill 2024 May 17

% homework problems begin
% Textbook problems 4.1, 4.4, 4.5, 4.8, 4.9, 4.14, 4.18*, 4.19
\bigskip
\noindent
\textbf{Problem 4.1.} Recall Problem 1.34, which concerned an ideal diatomic gas taken around a rectangular cycle on a \(PV\) diagram. Suppose now that this system is used as a heat engine, to convert the heat added into mechanical work.
\begin{enumerate}[label=\textbf{(\alph*)}]
\item Evaluate the efficiency of this engine for the case \(V_2 = 3V_1\), and \(P_2 = 2P_1\).
\item Calculate the efficiency of an ``ideal'' engine operating between the same temperature extremes.
\end{enumerate}
\noindent
Response.

\pagebreak
\noindent
\textbf{Problem 4.4.} It has been proposed to use the thermal gradient of the ocean to drive a heat engine. Suppose that at a certain location the water temperature is \(\qty{22}{\celsius}\) at the ocean surface and \(\qty{4}{\celsius}\) at the ocean floor.
\begin{enumerate}[label=\textbf{(\alph*)}]
\item What is the maximum possible efficiency of an engine operating between these two temperatures?
\item If the engine is to produce \(\qty{1}{\giga\watt}\) of electrical power, what minimum volume of water must be processed (to such out the heat) in every second?
\end{enumerate}
\noindent
Response.

\pagebreak
\noindent
\textbf{Problem 4.5.} Prove directly (by calculating the heat taken in and the heat expelled) that a Carnot engine using an ideal gas as the working substance has an efficiency of \(1 - T_c/T_h\).

\noindent
Response.

\pagebreak
\noindent
\textbf{Problem 4.8.} Can you cool off your kitchen by leaving the refrigerator door open?

\noindent
A room cannot be cooled by leaving the refrigerator open. The refrigerator cools the space inside the fridge by pumping heat into the surrounding environment. The surrounding environment is then heated and the space inside the fridge is cooled. By leaving the door open, a small temperature gradient is then created as the air inside is cooled and the air outside is heated. In the best case scenario where this requires no energy the room is left at the same overall temperature but there exists a separation. In reality, this whole process requires some extra work which is cast into the room as waste heat. Then, the fridge is simply acting as a poor space heater.

\pagebreak
\noindent
\textbf{Problem 4.9.} Estimate the maximum possible coefficient of performance of a household air conditioner. Use any reasonable values for the reservoir temperatures.

\noindent
The coefficient of performance, \(c\), is given by Equation \_ to be
\[ .\]

\pagebreak
\noindent
\textbf{Problem 4.14.} A heat pump is an electrical device that heats a building by pumping heat in from the cold outside. In other words, it's the same as a refrigerator, but its purpose is to warm the hot reservoir rather than to cool the cold reservoir (even though it does both). Let us define the following standard symbols, all taken to be positive by convention:
\begin{align*}
   T_h &= \text{temperature inside the building}
\\ T_c &= \text{temperature outside}
\\ Q_h &= \text{heat pumped into building in 1 day}
\\ Q_c &= \text{heat taken from outdoors in 1 day}
\\ W   &= \text{electrical energy used by heat pump in 1 day}
\end{align*}
\begin{enumerate}[label=\textbf{(\alph*)}]
\item Explain why the coefficient of performance (COP) for a heat pump should be defined as \(Q_h/W\).
\item What relation among \(Q_h\), \(Q_c\), and \(W\) is implied by energy conservation alone? Will energy conservation permit the COP to be greater than 1?
\item Use the second law of thermodynamics to derive an upper limit on the COP, in terms of the temperatures \(T_h\) and \(T_c\) alone.
\item Explain why a heat pump is better than an electric furnace, which simply converts electrical work directly into heat.
\end{enumerate}
\noindent
Response.

\pagebreak
\noindent
\textbf{Problem 4.18*.} Derive Equation 4.10 for the efficiency of the Otto cycle.

\noindent
Response.

\pagebreak
\noindent
\textbf{Problem 4.19.} The amount of work done by each stroke of an automobile engine is controlled by the amount of fuel injected into the cylinder: the more fuel, the higher the temperature and pressure at points 3 and 4 in the cycle. But according to Equation 4.10, the efficiency of the cycle depends only on the compression ratio which is always teh same for any particular engine, not on the amount of fuel consumed. Do you think this conclusion still holds when various other effects such as friction are taken into account? Would you expect a real engine to be most efficient when operating at high power or at low power?

\noindent
Response.
\end{document}
