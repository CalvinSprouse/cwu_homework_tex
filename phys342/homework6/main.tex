% document style header
\documentclass[a4paper, 12pt]{config/homework}

% import default packages
\usepackage{config/packages}
\usepackage{config/commands}

% end preamble
\begin{document}

% document title
\noindent
Calvin Sprouse \hfill PHYS342 Homework 6 \hfill 2024 May 17

% homework problems begin
% Textbook problems 4.1, 4.4, 4.5, 4.8, 4.9, 4.14, 4.18*, 4.19
\bigskip
\noindent
\textbf{Problem 4.1.} Recall Problem 1.34, which concerned an ideal diatomic gas taken around a rectangular cycle on a \(PV\) diagram. Suppose now that this system is used as a heat engine, to convert the heat added into mechanical work.
\begin{enumerate}[label=\textbf{(\alph*)}]
\item Evaluate the efficiency of this engine for the case \(V_2 = 3V_1\), \(P_2 = 2P_1\).
\item Calculate the efficiency of an ``ideal'' engine operating between the same temperature extremes.
\end{enumerate}
\noindent
Response.

\pagebreak
\noindent
\textbf{Problem 4.4.} It has been proposed to use the thermal gradient of the ocean to drive a heat engine. Suppose that at a certain location the water temperature is \(\qty{22}{\celsius}\) at the ocean surface and \(\qty{4}{\celsius}\) at the ocean floor.
\begin{enumerate}[label=\textbf{(\alph*)}]
\item What is the maximum possible efficiency of an engine operating between these two temperatures?
\item If the engine is to produce \(\qty{1}{\giga\watt}\) of electrical power, what minimum volume of water must be processed (to such out the heat) in every second?
\end{enumerate}
\noindent
Response.

\pagebreak
\noindent
\textbf{Problem 4.5.} Prove directly (by calculating the heat taken in and the heat expelled) that a Carnot engine using an ideal gas as the working substance has an efficiency of \(1 - T_c/T_h\).

\noindent
Response.

\pagebreak
\noindent
\textbf{Problem 4.8.} Can you cool off your kitchen by leaving the refrigerator door open?

\noindent
Response.

\pagebreak
\noindent
\textbf{Problem 4.9.} Estimate the maximum possible coefficient of performance of a household air conditioner. Use any reasonable values for the reservoir temperatures.

\noindent
Response.

\pagebreak
\noindent
\textbf{Problem 4.14.} A heat pump is an electrical device that heats a building by pumping heat in from the cold outside. In other words, it's the same as a refrigerator, but its purpose is to warm the hot reservoir rather than to cool the cold reservoir (even though it does both). Let us define the following standard symbols, all0taken to be positive by convention:

\noindent
Response.

\pagebreak
\noindent
\textbf{Problem 4.18*.}

\noindent
Response.

\pagebreak
\noindent
\textbf{Problem 4.19.}

\noindent
Response.
\end{document}
