% document style header
\documentclass[a4paper, 12pt]{config/homework}

% import default packages
\usepackage{config/defpackages}
% import custom math commands
\usepackage{config/domath}

% end preamble
\begin{document}

% document title
\noindent
\begin{tabularx}{\textwidth}{>{\centering\arraybackslash}X>{\centering\arraybackslash}X>{\centering\arraybackslash}X}
Calvin Sprouse & PHYS342 Homework 1 & 2024 April 05\\
\midrule
\end{tabularx}

% homework problems begin
% Textbook problems 1.4, 1.7, 1.9, 1.10, 1.11, 1.17(a) and (b), 1.18, 1.19, 1.23
\vspace{\baselineskip}
\begin{enumerate}
\item[1.4:] Does it ever make sense to say that one object is ``twice as hot'' as another? Does it matter whether one is referring to Celsius or kelvin temperatures? Explain.

It does not make sense to say that one object is ``twice as hot'' as another in the Celsius or Fahrenheit scale. These temperature scales are non-linear, though when talking about positive temperatures in these scales one could say ``twice as hot'' and still get their point across. Only in the linear scale, kelvin, does this statement make sense for all temperatures.

\vspace{\baselineskip}
\item[1.7:] When the temperature of liquid mercury increases by one degree Celsius (or one kelvin), its volume increases by one part in 5500. The fractional increase in volume per unit change in temperature (when the pressure is held fixed) is called the thermal expansion coefficient, \(\beta\),
\[\beta \equiv \frac{\Delta V / V}{\Delta T},\]
where \(V\) is the volume and \(T\) is the temperature of the substance in question. For mercury, \(\beta=\qty[parse-numbers=false]{1/5500}{\per\kelvin}\).
\begin{enumerate}[label=(\alph*.)]
\item Get a mercury thermometer, estimate the size of the bulb at the bottom, and then estimate what the inside diameter of the tube has to be in order for the thermometer to work as required. Assume that the thermal expansion of the glass is negligible.



\item The thermal expansion coefficient of water varies significantly with temperature. It is \(\qty{7.5e-4}{\per\kelvin}\) at \(\qty{100}{\celsius}\), but decreases as the temperature is lowered until it becomes 0 at \(\qty{4}{\celsius}\). Below \(\qty{4}{\celsius}\) it is slightly negative, reaching a value of \(\qty{-0.68e-4}{\per\kelvin}\) at \(\qty{0}{\celsius}\). This behavior is related to the fact that ise is les dense than water. With this behavior in mind, imagine the process of a lake freezing over, and discuss in some detail how this process would be different if the thermal expansion coefficient of water were always positive.



\end{enumerate}
\pagebreak
\item[1.9:] What is the volume of one mole of air, at room temperature and \(\qty{1}{\atm}\) pressure?



\vspace{\baselineskip}
\item[1.10:] Estimate the number of air molecules in an average-sized room.


\vspace{\baselineskip}
\item[1.11:] Rooms \(A\) and \(B\) are the same size and are connected by an open door. Room \(A\), however, is warmer. Which room contains the greater mass of air? Explain carefully.



\pagebreak
\item [1.17:] Even at low density, real gases don't quite obey the ideal gas law. A systematic way to account for deviations from ideal behavior is the virial expansion,
\[PV=nRT\left(1 + \frac{B(T)}{V/n} + \frac{C(T)}{(V/n)^2} + \cdots\right),\]
where the functions \(B(T)\), \(C(T)\), and so on are called the virial coefficients. When the density of the gas is fairly low, so that the volume per mole is large, each term in the series is much smaller than the one before. In many situations it's sufficient to omit the third term and concentrate on the second, whose coefficient \(B(T)\) is called the second virial coefficient. Here are some measured values of the second virial coefficient for nitrogen \(\text{N}_2\):
% insert table here
\begin{enumerate}
\item[(a.)] For each temperature in the table, compute the second term in the virial equation, \(B(T)/(V/n)\), for nitrogen at atmospheric pressure. Discuss the validity of the ideal gas law under these conditions.



\item[(b.)] Think about the forces between molecules, and explain why we might expect \(B(T)\) to be negative at low temperatures but positive at high temperatures.



\end{enumerate}
\pagebreak
\item[1.18:] Calculate the rms speed of a nitrogen molecule at room temperature.



\vspace{\baselineskip}
\item[1.19:] Suppose you have a gas containing hydrogen molecules and oxygen molecules in thermal equilibrium. Which molecules are moving faster, on average? By what factor?



\vspace{\baselineskip}
\item[1.23:] Calculate the total thermal energy in a liter of helium at room temperature and atmospheric pressure. Then repeat the calculation for a liter of air.



\end{enumerate}
\end{document}
