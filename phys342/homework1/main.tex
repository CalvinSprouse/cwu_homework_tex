% document style header
\documentclass[a4paper, 12pt]{config/homework}

% import default packages
\usepackage{config/defpackages}
% import custom math commands
\usepackage{config/domath}

% end preamble
\begin{document}

% document title
\noindent
\begin{tabularx}{\textwidth}{>{\centering\arraybackslash}X>{\centering\arraybackslash}X>{\centering\arraybackslash}X}
Calvin Sprouse & PHYS342 Homework 1 & 2024 April 05\\
\midrule
\end{tabularx}

% homework problems begin
% Textbook problems 1.4, 1.7, 1.9, 1.10, 1.11, 1.17(a) and (b), 1.18, 1.19, 1.23
\vspace{\baselineskip}
\begin{enumerate}
\item[1.4:] Does it ever make sense to say that one object is ``twice as hot'' as another? Does it matter whether one is referring to Celsius or kelvin temperatures? Explain.

It does not make sense to say that one object is ``twice as hot'' as another in the Celsius or Fahrenheit scale. These temperature scales are non-linear, though when talking about positive temperatures in these scales one could say ``twice as hot'' and still get their point across. Only in the linear scale, kelvin, does this statement make sense for all temperatures.

\vspace{\baselineskip}
\item[1.7:] When the temperature of liquid mercury increases by one degree Celsius (or one kelvin), its volume increases by one part in 5500. The fractional increase in volume per unit change in temperature (when the pressure is held fixed) is called the thermal expansion coefficient, \(\beta\),
\[\beta \equiv \frac{\Delta V / V}{\Delta T},\]
where \(V\) is the volume and \(T\) is the temperature of the substance in question. For mercury, \(\beta=\qty[parse-numbers=false]{1/5500}{\per\kelvin}\).
\begin{enumerate}[label=(\alph*.)]
\item Get a mercury thermometer, estimate the size of the bulb at the bottom, and then estimate what the inside diameter of the tube has to be in order for the thermometer to work as required. Assume that the thermal expansion of the glass is negligible.

Let \(d\) be the spacing between 1 kelvin marks on the thermometer. Suppose the bulb is a cylinder of radius \(R\) and height \(L\), and that the tube is a cylinder of radius \(r\) and length \(l\). As the temperature of the mercury rises by one kelvin the liquid must increase in volume by \(\Delta V = \pi r^2 d\). Working with the smallest measurable temperature the initial volume of liquid must be the volume of the bulb: \(V=\pi R^2 L\).

Suppose the bulb has radius \(R=\qty{2}{\milli\meter}\) and length \(L=\qty{4}{\milli\meter}\). Further suppose the tube has a length \(L=100d\), corresponding to a range of \(\qty{100}{\kelvin}\) and that \(d=\qty{1}{\milli\meter}\). Then the radius, \(r\), of the tube can be found by
\[\beta=\frac{\Delta V/ V}{\Delta T} \Rightarrow \sqrt{\frac{\Delta T \beta V}{\pi d}} = r.\]
Thus,
\[r = \sqrt{\frac{\qty[parse-numbers=false]{\frac{1}{5500}}{\per\kelvin}\times\qty{1}{\kelvin}\times\left(\qty{2}{\milli\meter}\right)^2\times\qty{4}{\milli\meter}}{\qty{1}{\milli\meter}}} = \qty{0.0539}{\milli\meter}.\]

\item The thermal expansion coefficient of water varies significantly with temperature. It is \(\qty{7.5e-4}{\per\kelvin}\) at \(\qty{100}{\celsius}\), but decreases as the temperature is lowered until it becomes 0 at \(\qty{4}{\celsius}\). Below \(\qty{4}{\celsius}\) it is slightly negative, reaching a value of \(\qty{-0.68e-4}{\per\kelvin}\) at \(\qty{0}{\celsius}\). This behavior is related to the fact that ice is less dense than water. With this behavior in mind, imagine the process of a lake freezing over, and discuss in some detail how this process would be different if the thermal expansion coefficient of water were always positive.

If the thermal expansion coefficient of water were always positive, the lake would simply form a sheet, or block, of ice which was at a lower water level than when the lake was liquid. In contrast, as lakes freeze the variation of waters expansion, particularly at the end of freezing when water begins to expand, causes significant erosion of the lake bed. Furthermore, the last bit of expansion can cause small overhangs to form on the shore of the lake. These overhangs at the shore would not form if the expansion coefficient remained positive as the water would just decrease in volume continuously.

\end{enumerate}
\vspace{\baselineskip}
\item[1.9:] What is the volume of one mole of air, at room temperature and \(\qty{1}{\atm}\) pressure?

The volume of air, which we shall assume is an ideal gas, is given by the ideal gas law to be
\[V = \frac{nRT}{P}.\]
From the problem statement, \(n=\qty{1}{\mol}\), \(T=\qty{293}{\kelvin}\), \(R=\qty{8.31}{\joule\per\mol\per\kelvin}\), and \(P=\qty{1.013e5}{\pascal}\). Then,
\[V = \frac{\qty{1}{\mol}\times\qty{293}{\kelvin}\times\qty{8.31}{\joule\per\mol\per\kelvin}}{\qty{1.013e5}{\pascal}}=\qty{0.0240}{\meter\cubed}.\]

\pagebreak
\item[1.10:] Estimate the number of air molecules in an average-sized room.

Suppose the average sized room is a cube with side lengths 5 meters. We then take standard temperature and pressure: \(T=\qty{293}{\kelvin}\), and \(P=\qty{1.013e5}{\pascal}\). The number of molecules in an ideal gas is given by
\[N = \frac{PV}{kT},\]
where \(V=\qty{125}{\meter\cubed}\), and \(k=\qty{1.381e-23}{\joule\per\kelvin}\).
Then
\[N = \frac{\qty{1.013e5}{\pascal}\times\qty{125}{\meter\cubed}}{\qty{1.381e-23}{\joule\per\kelvin}\qty{293}{\kelvin}} = \qty{3.13e27}.\]

\vspace{\baselineskip}
\item[1.11:] Rooms \(A\) and \(B\) are the same size and are connected by an open door. Room \(A\), however, is warmer. Which room contains the greater mass of air? Explain carefully.

Rooms \(A\) and \(B\) have the same mass of air. If we consider the rooms before room \(A\) was heated, perhaps that morning before sunrise, then they are in equilibrium and thus have the same mass of air. Then, the sun rose. Room \(A\) is heated and yet it's volume remains unchanged. Thus, the pressure rises in room \(A\). Room \(A\) may exchange a higher energy gas particle with room \(B\) as the two come back into thermal equilibrium. However, in doing so, room \(A\) will be at a lower pressure and thus room \(B\) will be more likely to ``send over'' a gas particle. Thus the two rooms will maintain an equilibrium in molecule count.

\vspace{\baselineskip}
\item [1.17:] Even at low density, real gases don't quite obey the ideal gas law. A systematic way to account for deviations from ideal behavior is the virial expansion,
\[PV=nRT\left(1 + \frac{B(T)}{V/n} + \frac{C(T)}{(V/n)^2} + \cdots\right),\]
where the functions \(B(T)\), \(C(T)\), and so on are called the virial coefficients. When the density of the gas is fairly low, so that the volume per mole is large, each term in the series is much smaller than the one before. In many situations it's sufficient to omit the third term and concentrate on the second, whose coefficient \(B(T)\) is called the second virial coefficient. Here are some measured values of the second virial coefficient for nitrogen \(\text{N}_2\):
\begin{table}[H]
\begin{center}
\begin{tabular}{cc}
\(T\) {[}K{]} & \(B\) \(\left[\text{cm}^3\,\text{mol}^{-1}\right]\) \\ \hline
100           & -160                                   \\
200           & -35                                    \\
300           & -4.2                                   \\
400           & 9.0                                    \\
500           & 16.9                                   \\
600           & 21.3
\end{tabular}
\end{center}
\end{table}
\begin{enumerate}
\item[(a.)] For each temperature in the table, compute the second term in the virial equation, \(B(T)/(V/n)\), for nitrogen at atmospheric pressure. Discuss the validity of the ideal gas law under these conditions.

For nitrogen at atmospheric pressure, the volume of gas is given by
\[V = \frac{nRT}{P},\]
where \(R\) is a universal gas constant, \(T\) is given, \(P=\qty{1.013e5}{\pascal}\), and the \(n\) term, being present in both numerator and denominator, divides out in the calculation of \(C\) from \(B\).

\begin{table}[H]
\begin{center}
\begin{tabular}{cc}
\(T\) {[}K{]} & \(B/(V/n)\) \\ \hline
100           & -19504                                 \\
200           & -2133                                    \\
300           & -170                                  \\
400           & 274                                   \\
500           & 412                                   \\
600           & 434
\end{tabular}
\end{center}
\end{table}

\item[(b.)] Think about the forces between molecules, and explain why we might expect \(B(T)\) to be negative at low temperatures but positive at high temperatures.

At low temperatures individual gas molecules lack the energy required to overcome attractive forces that turn a gas into a liquid or solid. Thus we anticipate a corrective term to represent gas particles coming together. At higher temperatures a gas particle becomes ``ignorant'' to these attractive forces, which have less-negligible impacts at ``room temperatures'', and act more freely.

\end{enumerate}
\pagebreak
\item[1.18:] Calculate the rms speed of a nitrogen molecule at room temperature.

The rms speed of a nitrogen gas molecule, (\(\text{N}_2\)), at some temperature is given by
\[v_\text{rms} = \sqrt{\frac{3kT}{m}},\]
where \(m=\qty{4.65e-26}{\kg}\). Then,
\[v_\text{rms} = \sqrt{\frac{3\times\qty{1.381e-23}{\joule\per\kelvin}\times\qty{293}{\kelvin}}{{\qty{4.65e-26}{\kg}}}} = \qty{511}{\meter\per\second}.\]

\vspace{\baselineskip}
\item[1.19:] Suppose you have a gas containing hydrogen molecules and oxygen molecules in thermal equilibrium. Which molecules are moving faster, on average? By what factor?

The ratio of rms velocities between the oxygen molecules, \((\text{O}_2)\), and hydrogen molecules, \((\text{H}_2)\), is given by
\[
v_{\text{rms},\text{O}_2}^2 m_{\text{O}_2} = v_{\text{rms},\text{H}_2}^2 m_{\text{H}_2}
\Rightarrow
\frac{v_{\text{rms},\text{O}_2}}{v_{\text{rms},\text{H}_2}} = \sqrt{\frac{m_{\text{H}_2}}{m_{\text{O}_2}}}.
\]

Substituting the masses of both molecules we find
\[\frac{v_{\text{rms},\text{O}_2}}{v_{\text{rms},\text{H}_2}} = \qty{1.022e-14}{}.\]
That is, the oxygen molecules are moving, on average, many orders of magnitude slower than the hydrogen molecules.

\vspace{\baselineskip}
\item[1.23:] Calculate the total thermal energy in a liter of helium at room temperature and atmospheric pressure. Then repeat the calculation for a liter of air.

The total thermal energy of an ideal gas is given by
\[U = Nf\frac{1}{2}kT,\]
where \(NkT\) can be found via the ideal gas law as
\[NkT = PV.\]
One liter is equivalent to \(\qty{1e-3}{\meter\cubed}\), which is \(V\). A helium molecule, \((\text{H}_2)\), is a diatomic molecule and thus has 3 translational degrees of freedom and two rotational: \(f=5\). Thus, the total thermal energy stored in this system is given by
\[U = \frac{PVf}{2} = \frac{\qty{1.013e5}{\pascal}\times\qty{1e-3}{\meter\cubed}\times5}{2} = \qty{253.25}{\joule}.\]

\end{enumerate}
\end{document}
