% document style header
\documentclass[a4paper, 12pt]{config/homework}

% import default packages
\usepackage{config/packages}
\usepackage{config/commands}

% end preamble
\begin{document}

% document title
\noindent
Calvin Sprouse \hfill PHYS 342 Homework 7 \hfill 2024 May 31
\bigskip

% homework problems begin
% Textbook problems 5.6, 5.18, 5.32, 5.33
\bigskip\noindent
\textbf{Problem 5.6.} A muscle can be thought of as a fuel cell, producing work from the metabolism of glucose:
\[\text{C}_6\text{H}_{12}\text{O}_6 + 6\text{O}_2 \longrightarrow 6\text{CO}_2 + 6\text{H}_2\text{O}.\]
\begin{enumerate}[label=\textbf{(\alph*)}]
\item Use the data at the back of this book to determine the values of \(\Delta H\) and \(\Delta G\) for this reaction, for one mole of glucose. Assume that the reaction takes place at room temperature and atmospheric pressure.
\item What is the maximum amount of work that a muscle can preform, for each mole of glucose consumed, assuming ideal operation?
\item Still assuming ideal operation, how much heat is absorbed or expelled by the chemicals during the metabolism of a mole of glucose?
\item Use the concept of entropy to explain why the heat flows in the direction it does.
\item How would your answers to parts (b) and (c) change, if the operation of the muscle is not ideal?
\end{enumerate}
\begin{enumerate}[label=\textbf{(\alph*)}]
\item From Chemistry classes,
\[\Delta H = H(\text{products}) - H(\text{reactants}),\]
and likewise for \(\Delta G\). Then,
\begin{align*}
\Delta H &= 6\left(\qty{-393.5}{\kilo\joule}\right) + 6\left(\qty{-285.8}{\kilo\joule}\right) - \left(\qty{-1273}{\kilo\joule}\right) = \qty{-2803}{\kilo\joule}, \\
\Delta G &= 6\left(-\qty{394.4}{\kilo\joule}\right) + 6\left(-\qty{-237.1}{\kilo\joule}\right) - \left(\qty{-910}{\kilo\joule}\right) = \qty{-2879}{\kilo\joule}.
\end{align*}
\bigskip
\item By Equation~5.8,
\[\Delta G \le W_\text{other}.\]
Then,
\[W_\text{other} \le \qty{2879}{\kilo\joule},\]
where the direction of the inequality has reversed because \(\Delta G\) is negative.
\bigskip
\item Equation~5.9 gives the balance of energy changes as
\[\Delta G = \Delta H - T \Delta S.\]
Then,
\[T\Delta S = \Delta H - \Delta G = \qty{76}{\kilo\joule}.\]
Thus,
\(\qty{76}{\kilo\joule}\) must be absorbed by the muscle under ideal operation.
\bigskip
\item \(\Delta S\) for the reaction may be calculated as in part (a):
\[\Delta S = 6\left(\qty{214}{\joule\per\kelvin}\right) + 6\left(\qty{70}{\joule\per\kelvin}\right) - \left(\qty{212}{\joule\per\kelvin}\right) - 6\left(\qty{205}{\joule\per\kelvin}\right) = \qty{262}{\joule\per\kelvin}.\]
The mechanism for increasing the entropy during this reaction is indicated by Equation~5.8 as a flow of temperature or heat energy.
\bigskip
\item Outside the ideal operation of engines, new entropy is created. This new entropy reduced the work output. The answer to part (b) is still accurate, in a non-ideal case it would just be impossible to extract that much other work. The amount of heat that must flow into the system, part (c), would be reduced if new entropy is created, and may even result in heat flowing out of the system.
\end{enumerate}

\pagebreak\noindent
\textbf{Problem 5.18.} Imagine that you drop a brick on the ground and it lands with a thud. Apparently, the energy of this system tends to spontaneously decrease. Explain why.

\bigskip\noindent
At the instant of contact with the ground, the bricks kinetic energy goes to 0 and it's thermal energy increases proportionally. This sudden increase in thermal energy takes the brick and immediate surroundings out of thermal equilibrium with other points of thermal contact. As the energy dissipates from the site of contact, into the brick or into the ground, the system returns to thermal equilibrium. The flow of thermal energy out of the brick and nearby ground system into the environment is spontaneous. Thus, the energy of the system is decreasing spontaneously.

\pagebreak\noindent
\textbf{Problem 5.32.} The density of ice is \(\qty{917}{\kg\per\meter\cubed}\).
\begin{enumerate}[label=\textbf{(\alph*)}]
\item Use the Clausius-Clapeyron relation to explain why the slope of the phase boundary between water and ice is negative.
\item How much pressure would need to be put on an ice cube to make it melt at \(\qty{-1}{\celsius}\)?
\item Approximately how deep under a glacier would one need to be before the weight of the ice above gives the pressure found in part (b)?
\item Make a rough estimate of the pressure under the blade of an ice skate, and calculate the melting temperature of ice at this pressure. Some authors have claimed that skaters glide with very little friction because the increased pressure under the blade melts the ice to create a thin layer of water. Is this claim plausible?
\end{enumerate}
\begin{enumerate}[label=\textbf{(\alph*)}]
\item The Clausius-Clapeyron relation is given by Equation~4.67 as
\[\diff{P}{T} = \frac{L}{T\Delta V}.\]
Considering the transition from ice to water, the latent heat, \(L\), is positive and temperature, \(T\), is also positive. Since water is more dense than ice the change in volume, \(\Delta V\), is negative. Thus, the boundary line between the solid and liquid states of water is negative.
\bigskip
\item The slope of the boundary line between water and ice is given by the Clausius-Clapeyron relation,
\[\diff{P}{T} = \frac{L}{T\Delta V}.\]
The latent heat, \(L\), may be expressed in terms of the heat of the phase change, \(Q\), and the mass of the substance \(m\). Then,
\[\diff{P}{T} = \frac{Qm}{T \left(mD_\text{ice}^{-1} - mD_\text{water}^{-1}\right)},\]
where \(D_\text{ice}\) and \(D_\text{water}\) represent the density and we have assumed the mass is not changing during the phase change. Expressed in SI, the latent heat of the solid to liquid transition in water is
\[L = \qty{3.3335e5}{\joule\per\kilogram}.\]
The density of ice is given above and the density of cold liquid water is
\[D_\text{water} = \qty{1e3}{\kg\per\meter\cubed}.\]
Then,
\[P_\text{melt}(t) = \diff{P}{T}(t-T) = \frac{Q}{T \left(D_\text{ice}^{-1} - D_\text{water}^{-1}\right)}(t-T).\]
Substituting given quantities yields
\[P_\text{melt}(\qty{-1}{\celsius}) = \qty{1.35e7}{\pascal} = 133\,\text{atm}.\]
\bigskip
\item Consider a volume of glacial ice defined by a constant density \(D\) and height \(z\). The glacial ice under investigation below this column then experiences a pressure
\[P = Dzg.\]
Given a particular pressure and constant density the amount of glacial ice needed, expressed in terms of the column height \(z\), is given by
\[z = \frac{P}{Dg}.\]
Then, substituting known and given quantities yields
\[z = \qty{1375}{\meter}.\]
\bigskip
\item Consider the most extreme case of an ice skater gliding along the tip of just one of their skates. Suppose the blade is \(\qty{5}{\milli\meter}\) wide and the angled portion at the tip measures \(\qty{10}{\milli\meter}\) long. Then, the area in contact with the ice, \(A\), is
\[A = \left(\qty{5}{\milli\meter}\right)\left(\qty{10}{\milli\meter}\right) = \qty{50}{\milli\meter\squared}.\]
Further suppose the skater has a mass of \(m=\qty{70}{\kilogram}\). Let the gravitational acceleration on the surface of the Earth be \(g=\qty{10}{\meter\per\second\squared}\). Then the pressure exerted on the ice by the tip of the skate is approximately
\[P = \frac{\left(\qty{70}{\kilogram}\right)\left(\qty{10}{\meter\per\second\squared}\right)}{\qty{50e-6}{\meter\squared}} = \qty{14e6}{\pascal} \approx 138\,\text{atm}.\]
Based on the expression derived above this would lower the freezing point of water to just below \(\qty{-1}{\celsius}\). Supposing the ice skating facility was keeping the water at just below the freezing point then there may be a slight surface melt creating a slight gliding effect. However, as soon as the skater uses the full blade, whose length is on the order of centimeters, the pressure drops by an order of magnitude. Should the skater use both skates the pressure halves. The surface gravity on Earth is slightly less, skaters are often lighter, and it's unlikely that a skating rink is kept so close to the melting point of water to begin with. Which is to say that pressure melting is likely not a significant factor in normal skating.
\end{enumerate}

\pagebreak\noindent
\textbf{Problem 5.33.} An inventor proposes to make a heat engine using water/ice as the working substance, taking advantage of the fact that water expands as it freezes. A weight to be lifted is placed on top of a piston over a cylinder of water at \(\qty{1}{\celsius}\). The system is then placed in thermal contact with a low-temperature reservoir at \(\qty{-1}{\celsius}\) until the water freezes into ice, lifting the weight. The weight is then removed and the ice is melted by putting it in contact with a high-temperature reservoir at \(\qty{1}{\celsius}\). The inventor is pleased with this device because it can seemingly preform an unlimited amount of work while absorbing only a finite amount of heat. Explain the flaw in the inventor's reasoning, and use the Clausius-Clapeyron relation to prove that the maximum efficiency of this engine is still given by the Carnot formula, \(1 - T_c/T_h\).

\bigskip\noindent
The inventor fails to consider that the work done on the environment by the freezing of water, lifting the weight, is limited. For a weight of mass \(m\) to be raised a height \(h\) an amount of work \(W=mgh\) is required where \(g\) is the surface gravity of Earth. To reset the engine an amount of heat \(Q_h=L\) is needed. The efficiency of the engine may then be defined as the desired output, \(W\), divided by the required input, \(Q_h\):
\[e = \frac{W}{Q_h} = \frac{mgh}{L}.\]
The Clausius-Clapeyron relation is given by Equation~5.47,
\[\diff{P}{T} = \frac{L}{T\Delta V}.\]
Let \(T_c\) and \(T_h\) be the temperature of the cold and hot reservoir respectively. Increasing the mass of the weight to be lifted increases the pressure on the water thus lowering the freezing point. The maximum weight which can be lifted, and in turn the maximum work which can be done by the water, is limited by the temperature of the cold reservoir. Suppose the piston as a surface area \(A\). Then, the change in volume, \(\Delta V\), of Equation~5.47 may be expressed as \(\Delta V = Ah\). Then,
\[\diff{P}{T} = \frac{L}{T_h A h}.\]
This may be expressed as a differential equation,
\[\text{d}P = \frac{L}{T_h A h} \text{d}T,\]
which when integrated could be expressed as
\[P = \frac{L}{T_h A h} \left(T_h - T_c\right).\]
The pressure on the water can be found in terms of the weight and surface area to be
\[P = \frac{mg}{A}.\]
Then,
\[\frac{mgh}{L} = 1 - \frac{T_c}{T_h}.\]
Substituting back into the expression for the efficiency of this ice engine yields
\[e = 1 - \frac{T_c}{T_h},\]
the maximum possible weight, or maximum possible efficiency, of the engine as expected. Lifting any weight less than the maximum weight represents a less efficient operating mode of the ice engine.


\end{document}
