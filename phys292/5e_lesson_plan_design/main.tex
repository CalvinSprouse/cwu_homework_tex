% LaTeX Document made from HomeworkTemplate
% Last Updated: 2023 november 05

% document style header
\documentclass[a4paper, 12pt]{../../config/homework}

% import default packages
\usepackage{../../config/defpackages}

% import fun symbol packages (and coffee)
% \usepackage{.config/sympack}

% import custom math commands
\usepackage{../../config/domath}

% custom commands reminder:
% \matr                 easy matrices
% \reals                real numbers
% \zero                 zero vector/matrix
% \field                mathbb{F}
% \ip                   brackets (like \braket)
% \vspan                the span 'function'
% \rcurs                script-r (also br and hr for vectors}
% \el                   Euler-Lagrange L
% \given                prints \left( #1 | #2 \right)
% \midpipe              prints (and spaces) a pipe character
% \singlelineheight     the height of a line
% \bint                 bounded integral: integrand, lower, upper, variable
% \pvec                 primed vector: symbol, subscript
% \drinkproof           proof ender with the double ravens

% end preamble
\begin{document}
\begin{center}
\begin{tabularx}{\textwidth}{| >{\raggedright\arraybackslash}X |}
\hline
\textbf{Teacher:} Calvin Sprouse
\\\hline

\textbf{Date:} 2023 December 01
\\\hline

\textbf{Subject / Grade level:} PHYS 183
\\\hline

\textbf{Topic:} The Pendulum Circuit
\\\hline

\textbf{Materials:}
\\\hline

\textbf{Lesson Objective(s):}
\begin{itemize}
\item Students will learn basic circuit design and component selection to achieve a purpose.
\item Students will learn about physics analogues: situations where two seemingly unrelated physical systems are governed by the same equations.
\end{itemize}
\\\hline

\textbf{ENGAGEMENT:} Students will be presented with a theoretical pendulum and calculate the period and velocity as a function of time.
\\\hline

\textbf{EXPLORATION:} Students will then be presented with the schematic of a voltage divider circuit. Given an AC input students will need to find a pair of resistors such that the graph of output voltage is identical to their theoretical pendulum (they can configure the period of the AC source).
\\\hline

\textbf{EXPLANATION:} Students will then build the pendulum and circuit (does the cart software do circuits? if so they can do the graphs on the same window! Otherwise most software can to a csv export and they can use excel to plot the curves over one another? The goal here is just to demonstrate that it works.
\\\hline

\textbf{ELABORATION:} Students will then find an equation the relates their pendulum mass and length to a resistor value such that for any length and mass pendulum they can pick a resistor pair to make an analogous circuit.
\\\hline

\textbf{EVALUATION:} Students will be given a new mass and length and must build a circuit and demonstrate they are analogues.
\\\hline

\end{tabularx}
\end{center}

I know theres a lot here and I have no idea if this goes above and beyond the 183 level. I really wanted to touch on analagous physical systems because I think they are tremendously important and highlight the power of both physics and math. I also wanted to do something with EM since I think its a really fun topic and the more students can link it to real life experience the easier it will be. I had another idea of having students build a rope out of a series of magnets and plastic spacers. Students could figure out that the rope could theoretically pull an infinite force (in a frictionless world). By giving them a surface with particular friction students could be tasked with figuring out what length spacer is the maximum allowed to pull a certain mass. Both are fun ideas!
\end{document}
