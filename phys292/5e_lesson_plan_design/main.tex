% LaTeX Document made from HomeworkTemplate
% Last Updated: 2023 november 05

% document style header
\documentclass[a4paper, 12pt]{../../config/homework}

% import default packages
\usepackage{../../config/defpackages}

% import fun symbol packages (and coffee)
% \usepackage{.config/sympack}

% import custom math commands
\usepackage{../../config/domath}

% custom commands reminder:
% \matr                 easy matrices
% \reals                real numbers
% \zero                 zero vector/matrix
% \field                mathbb{F}
% \ip                   brackets (like \braket)
% \vspan                the span 'function'
% \rcurs                script-r (also br and hr for vectors}
% \el                   Euler-Lagrange L
% \given                prints \left( #1 | #2 \right)
% \midpipe              prints (and spaces) a pipe character
% \singlelineheight     the height of a line
% \bint                 bounded integral: integrand, lower, upper, variable
% \pvec                 primed vector: symbol, subscript
% \drinkproof           proof ender with the double ravens

% end preamble
\begin{document}
\begin{center}
\begin{tabularx}{\textwidth}{| >{\raggedright\arraybackslash}X |}
\hline
\textbf{Teacher:} Calvin Sprouse
\\\hline

\textbf{Date:} 2023 December 03
\\\hline

\textbf{Subject / Grade level:} PHYS 183
\\\hline

\textbf{Topic:} The Pendulum Circuit
\\\hline

\textbf{Materials:}
\begin{itemize}
\item A pendulum that supports swapping the string.
\item A set of two weights that hook onto the loops of a string.
\item A set of two lengths of string.
\item A protractor for measuring pendulum angle
\item Access to a computer and a link to the PHET Circuit Construction Kit: AC module.
\item Link to a desmos digital notebook that has been created as detailed below.
\end{itemize}
\\\hline

\textbf{Lesson Objective(s):}
\begin{itemize}
\item Students will learn basic circuit design and component selection to achieve a purpose.
\item Students will learn about physics analogues: situations where two seemingly unrelated physical systems are governed by the same equations.
\end{itemize}
\\\hline

\textbf{ENGAGEMENT:} Students will first be given the equations that govern a simple pendulum,
\[\theta(t) = \theta_0 \cos\left(\frac{2\pi}{T}t + \phi\right), \quad T=2\pi\sqrt{\frac{L}{g}},\]
where each term will be defined by an accompanying diagram. Students will also be told this is an approximate equation that is only good if their pendulum stays under about \(\pi/12\) radians. As a refresher exercise they will be asked to convert this to degrees.

If there is time, students could be asked to think about why mass does not appear in the equation for period. A good answer might be one that mentions how the acceleration of an object under just gravity is independent of mass.
\\\hline
\end{tabularx}

\begin{tabularx}{\textwidth}{| >{\raggedright\arraybackslash}X |}
\hline
\textbf{EXPLORATION:} Students will then be presented with a labeled schematic and equations for a simple AC circuit voltage divider,
\[V_\text{out}(t) = A\sin(\omega t), \quad A = \frac{R_2}{R_1 + R_2}V_0.\]
They will then be tasked with identifying related quantities to their pendulum equations. In this case
\[A = \theta_0, \quad \frac{2\pi}{T} = \omega.\]
\textit{A potential point of tension: \(A=\theta_0\) is not technically correct by units. Some students may point this out and thats a very good thing to recognize.}
\\\hline

\textbf{EXPLANATION:} Students will then build their AC circuit digitally. A value for \(R_1\), such that calculations are nice and clean, will be provided on the worksheet as well as a value for \(V_0\) that the students must use for the rest of the worksheet. Students will also choose some value of \(L\) and \(\theta_0\), so long as \(\theta_0 < \pi/12\) radians, and calculate a value of \(R_2\) and \(\omega \) to create a digital analogue.
\\\hline

\textbf{ELABORATION:} Students will verify their circuit works with some digital view in the circuit software. They will then be given a link to some pre-existing Desmos notebook where a pendulum and circuit function have already been defined and they can tweak the values of \(R_2, A, \theta_0, L\), and \(\phi_0\). They will then have to plug in calculated values and figure out how to change \(\phi_0\) such that their waves visually line up. As a hint/refresher some trig identities will have been provided at the front of the worksheet. Students should recognize how to convert sin functions to cos functions.
\\\hline

\textbf{EVALUATION:} Students will be asked to think about another situation that could be modeled with a sin or cos function. They should recall, perhaps, orbital mechanics. If they have any other thoughts, like a spring oscillator, they are welcome to use those. They should then identify the equivalent quantities to their AC circuit. For orbital mechanics the amplitude, \(A\), is like the radius of the orbit and the frequency is inversely related to the period.

If there is time they can investigate this third model and build a circuit analogue but honestly I think this will already be a long lab.
\\\hline

\end{tabularx}
\end{center}
\end{document}
