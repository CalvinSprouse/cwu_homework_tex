% document style header
\documentclass[a4paper, 12pt]{config/homework}

% import default packages
\usepackage{config/packages}
\usepackage{config/commands}

% end preamble
\begin{document}

% document title
\noindent
Calvin Sprouse \hfill PHYS 475 Exam 2 \hfill 2024 May 23
\bigskip

% homework problems begin
\noindent
Many semiconductor devices are designed to confine electrons within a thin layer that is only a few nanometers thick. When a potential difference is applied across such a layer, the electrons respond as though they are trapped within a microscopic capacitor. If our ``capacitor'' plates are separated by a distance, \(L\), that is of the order of the de Broglie wavelength of a trapped electron, we must apply a quantum treatment to the study of its behavior. We can treat this ``capacitor'' like an infinite square well,
\[V(x) = \begin{cases}
0, & 0 \le x \le L, \\
\infty, & \text{otherwise}.
\end{cases}\]
Applying a potential difference of \(\Delta V_0\) across it introduces an additional potential energy of,
\[V'(x) = \frac{e\Delta V_0}{L}x.\]
Use perturbation theory to determine the first-order correction to the energy of the \(n\)th eigenstate associated with the perturbation.

\bigskip\noindent
The perturbation hamiltonian, \(\hat{H}'\), is simply the applied potential,
\[\hat{H}' = \frac{e}{L}\Delta V_0 \hat{x}.\]
The first-order energy correction is given by Griffiths Equation 7.9 to be
\[E_n^1 = \bra*{\psi^0_n}\hat{H}'\ket*{\psi^0_n},\]
where the unperturbed stationary states, \(\psi_n^0\), are given by Griffiths Equation 2.31 to be
\[\psi_n^0 = \sqrt{\frac{2}{L}}\sin\left(\frac{n\pi}{L}x\right).\]
The first-order energy correction may be found by evaluating the integral corresponding to Equation 7.9; that is,
\[E_n^1 = \defint{0}{L}{\sqrt{\frac{2}{L}}\sin\left(\frac{n\pi}{L}x\right) \frac{e\Delta V_0}{L}x \sqrt{\frac{2}{L}}\sin\left(\frac{n\pi}{L}x\right)}{x},\]
where we have recognized that the only non-zero terms in the integral are in the region \(0\le x \le L\). Let \(a\) be defined as
\[a \equiv \frac{n\pi}{L}.\]
Then,
\[E_n^1 = \frac{2e\Delta V_0}{L^2} \defint{0}{L}{x\sin^2\left(ax\right)}{x}.\]
This integral is given by Equation 17 on the class Table of Integrals:
\[\int{x\sin^2\left(ax\right)\,\text{d}x} = \frac{x^2}{4} - \frac{x\sin(2ax)}{4a} - \frac{\cos(2ax)}{8a^2}.\]
Then,
\[E_n^1 = \frac{2e\Delta V_0}{L^2} \left[
\frac{L^2}{4} - \frac{L^2\sin\left(2n\pi\right)}{8(n\pi)^2} - \frac{L^2\cos\left(2n\pi\right)}{8(n\pi)^2} - 0 + 0 + \frac{L^2}{8\left(n\pi\right)^2}
\right],\]
where trivial trigonometric functions have been simplified. We recognize that \\ \(\sin(2\pi n) = 0\) and \(\cos(2\pi n) = 1\) for \(n\in\ints^+\). Thus, the first-order energy correction of the \(n\)th eigenstate is given by
\[E_n^1 = \frac{e\Delta V_0}{2}.\]
\end{document}
