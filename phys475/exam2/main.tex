% document style header
\documentclass[a4paper, 12pt]{config/homework}

% import default packages
\usepackage{config/packages}
\usepackage{config/commands}

% end preamble
\begin{document}

% document title
\noindent
Calvin Sprouse \hfill PHYS 475 Exam 2 \hfill 2024 May 23
\bigskip

% homework problems begin
\noindent
Many semiconductor devices are designed to confine electrons within a thin layer that is only a few nanometers thick. When a potential difference is applied across such a layer, the electrons respond as though they are trapped within a microscopic capacitor. If our ``capacitor'' plates are separated by a distance, \(L\), that is of the order of the de Broglie wavelength of a trapped electron, we must apply a quantum treatment to the study of its behavior. We can treat this ``capacitor'' like an infinite square well,
\[V(x) = \begin{cases}
0, & 0 \le x \le L, \\
\infty, & \text{otherwise}.
\end{cases}\]
Applying a potential difference of \(\Delta V_0\) across it introduces an additional potential energy of,
\[V'(x) = \frac{e\Delta V_0}{L}x.\]
Use perturbation theory to determine the first-order correction to the energy of the \(n\)th eigenstate associated with the perturbation.
\end{document}
