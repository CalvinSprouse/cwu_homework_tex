% document style header
\documentclass[a4paper, 12pt]{config/homework}

% import default packages
\usepackage{config/packages}
\usepackage{config/commands}

% end preamble
\begin{document}

% document title
\noindent
\hfill Calvin Sprouse \hfill PHYS 475 Homework 4 \hfill 2024 May 6 \hfill
\bigskip

% homework problems begin
\begin{enumerate}
\item Consider an infinite square well that runs from 0 to \(a\). As a perturbation, we place a delta-function bump at the center of the well,
\[\hat{H}' = \alpha\delta\left(x-\frac{a}{2}\right),\]
where \(\alpha\) is a constant.
\begin{enumerate}[label=(\alph*)]
\item Determine the first-order correction to the allowed energies. Why are the energies for even \(n\) unperturbed?
\item Determine the first three non-zero terms in the perturbation expansion,
\[\psi^1_n = \sum_{m\ne n}\frac{\bra*{\psi_m^0}\hat{H}'\ket*{\psi_n^0}}{E_n^0-E_m^0}\psi_m^0,\]
of the correction to the ground state \(\psi_1^1\).
\end{enumerate}
% solutions
\begin{enumerate}[label=(\alph*)]
\item 
\end{enumerate}

\pagebreak
\item The allowed energies for the harmonic oscillator are
\[E_n = \left(n + \frac{1}{2}\right)\hbar\omega,\]
where \(\omega = \sqrt{k/m}\) is the classical frequency and the potential energy is
\[V(x) = \frac{1}{2}kx^2.\]
Suppose the spring constant increases lightly, \(k\to\left(1+\epsilon\right)k\).
\begin{enumerate}[label=(\alph*)]
\item Determine the exact new energies, then expand the formula as a power series in \(\epsilon\) up to second order.
\item Calculate the first-order perturbation to the energy using
\[E_n^1=\bra*{\psi_n^0}\hat{H}'\ket*{\psi_n^0}.\]
To preform this calculation, it will be necessary to determine what \(\hat{H}'\) is in this case. Compare this result with the result from part (a).
\end{enumerate}
% solutions
\begin{enumerate}[label=(\alph*)]
\item 
\end{enumerate}
\end{enumerate}
\end{document}
