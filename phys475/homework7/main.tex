% document style header
\documentclass[a4paper, 12pt]{config/homework}

% import default packages
\usepackage{config/packages}
\usepackage{config/commands}

% end preamble
\begin{document}

% document title
\noindent
Calvin Sprouse \hfill PHYS 475 Homework 7 \hfill 2024 June 03
\bigskip

% homework problems begin
\begin{enumerate}
\item Using the test wavefunction
\[\psi(x) = A\exp\left[-bx^2\right],\]
obtain the lowest upper-bound you can on the ground-state energies associated with the following one-dimensional potential energies.
\begin{enumerate}[label=(\alph*)]
\item \(V(x)=\alpha |x|\).
\item \(V(x)=\alpha x^4\).
\end{enumerate}
\bigskip
The normalization constant, \(A\), is given by Griffiths Equation~8.3,
\[A = \left(\frac{2b}{\pi}\right)^{1/4}.\]
Then,
\[\psi(x) = \left(\frac{2b}{\pi}\right)^{1/4}\exp\left[-bx^2\right].\]
The expectation value of kinetic energy, \(\expval{T}\), is independent of the potential energy function and can be found in general for a given test wavefunction. For the test wavefunction given, the expectation value of kinetic energy is given by Griffiths Equation~8.5,
\[\expval{T} = \frac{\hbar^2 b}{2m}.\]
The expectation value for total energy, \(\expval{H}\), is given by Griffiths Equation~8.4,
\[\expval{H} = \expval{T} + \expval{V} = \frac{\hbar^2 b}{2m} + \expval{V},\]
where Griffiths Equation~8.5 was substituted.
\begin{enumerate}[label=(\alph*)]
\pagebreak
\item The expectation value of potential energy, \(\expval{V}\), is given by
\[\expval{V} = \defint{-\infty}{\infty}{\psi^* V \psi}{x} = \sqrt{\frac{2b}{\pi}}\alpha\defint{-\infty}{\infty}{|x|e^{-2bx^2}}{x}.\]
The integrand is symmetric about \(x=0\); that is,
\[\expval{V} = 2 \alpha \sqrt{\frac{2b}{\pi}} \defint{0}{\infty}{xe^{-2bx^2}}{x}.\]
\(\expval{V}\) is of the form of Equation~8 on the class Integral Table,
\[\defint{0}{\infty}{x^m e^{-ax^2}}{x} = \frac{\Gamma\left(\frac{m+1}{2}\right)}{2a^{\frac{m+1}{2}}},\]
where \(a=2b\) and \(m=1\). Then,
\[\expval{V} = 2\alpha\sqrt{\frac{2b}{\pi}}\frac{1}{4b} = \frac{\alpha}{\sqrt{2b\pi}}.\]
The expectation value of total energy is then
\[\expval{H} = \frac{\hbar^2 b}{2m} + \frac{\alpha}{\sqrt{2b\pi}}.\]
To minimize with respect to \(b\) we first take the derivative with respect to \(b\),
\[\diff{}{b}\expval{H} = \frac{\hbar^2}{2m} - \frac{1}{2}\frac{\alpha}{\sqrt{2\pi}}b^{-3/2}.\]
The minimum value is given by solutions to
\[\diff{}{b}\expval{H} = 0.\]
This is satisfied for
\[b = \left( \frac{m\alpha}{\hbar^2\sqrt{2\pi}} \right)^{2/3}.\]
The minimum expectation value of total energy is given by substituting the minimized value of \(b\),
\[\expval{H}_\text{min} = \frac{\hbar^2}{2m}\left( \frac{m\alpha}{\hbar^2\sqrt{2\pi}} \right)^{2/3} + \frac{\alpha}{\sqrt{2\pi}}\sqrt{\frac{1}{\left( \frac{m\alpha}{\hbar^2\sqrt{2\pi}} \right)^{2/3}}} = \frac{3}{2}\left(\frac{\alpha^2\hbar^2}{2\pi m}\right)^{1/3}.\]
Finally,
\[\frac{3}{2}\left(\frac{\alpha^2 \hbar^2}{2\pi m}\right) \ge E_\text{gs}.\]
\pagebreak
\item The expectation value of potential energy, \(\expval{V}\), is given by
\[\expval{V} = \defint{-\infty}{\infty}{\psi^* V \psi}{x} = \alpha\sqrt{\frac{2b}{\pi}}\defint{-\infty}{\infty}{x^4e^{-2bx^2}}{x}.\]
Like in part (a), this integrand is symmetric about \(x=0\),
\[\expval{V} = 2\alpha\sqrt{\frac{2b}{\pi}}\defint{0}{\infty}{x^4e^{-2bx^2}}{x}.\]
Also like in part (a), \(\expval{V}\) is of the form of Equation~8 of the class Integral Table where \(a=2b\) and \(m=4\). Then, with \(\Gamma\) as the Gamma-function,
\[\expval{V} = 2\alpha\sqrt{\frac{2b}{\pi}}\frac{\Gamma\left(\frac{5}{2}\right)}{2(2b)^{5/2}} = 2\alpha\sqrt{\frac{2b}{\pi}}\frac{\frac{3\sqrt{\pi}}{4}}{2(2b)^{5/2}} = \frac{3\alpha}{16b^2}.\]
The expectation value of total energy is then
\[\expval{H} = \frac{\hbar^2 b}{2m} + \frac{3\alpha}{16b^2}.\]
The minimum value of \(\expval{H}\) is given by solutions to
\[\diff{}{b}\expval{H} = 0,\]
where
\[\diff{}{b}\expval{H} = \frac{\hbar^2}{2m} - \frac{3\alpha}{8}b^{-3}.\]
This is satisfied for
\[b = \left(\frac{3\alpha m}{4\hbar^2}\right)^{1/3}.\]
Then,
\[\expval{H}_\text{min} = \frac{\hbar^2 }{2m}\left(\frac{3\alpha m}{4\hbar^2}\right)^{1/3} + \frac{3\alpha}{16}\left(\frac{3\alpha m}{4\hbar^2}\right)^{-2/3}= \frac{3}{4}\left(\frac{3\alpha \hbar^4}{4m^2}\right)^{1/3}.\]
Finally,
\[\frac{3}{4}\left(\frac{3\alpha \hbar^4}{4m^2}\right)^{1/3} \ge E_\text{gs}.\]
\end{enumerate}

\pagebreak
\item Use the WKB approximation to determine the allowed energies, \(E_n\), of an infinite-square well with a ``shelf'' of height \(V_0\) extending half-way across,
\[V(x) = \begin{cases}
V_0, & 0 < x < a/2, \\ 0, & a/2 < x < a, \\ \infty, & \text{else}.
\end{cases}\]
Express the solution in terms of \(V_0\) and \(E_n^0\), the energy for the \(n\)th state of the infinite-square well. Assume that \(E_1^0 > V_0\), but do not assume \(E_n \gg V_0\).

\bigskip
Response.

\pagebreak
Use the WKB approximation result for transmission probability,
\[T \approx e^{-2\gamma},\]
where
\[\gamma \equiv \frac{1}{\hbar}\defint{0}{a}{|p(x)|}{x},\]
to compute the approximate transmission probability for a particle with energy \(E\) that encounters a finite-square barrier with height \(V_0 > E\) and width \(2a\). Compare the solution to the exact result,
\[T^{-1} = 1 + \frac{V_0^2}{4E(V_0-E)}\sinh^2\left(\frac{2a}{\hbar}\sqrt{2m(V_0-E)}\right),\]
which should reduce to the solution in the WKB regime where \(T\ll 1\).

\bigskip
Response.

\end{enumerate}
\end{document}
