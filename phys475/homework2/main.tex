% document style header
\documentclass[a4paper, 12pt]{config/homework}

% import default packages
\usepackage{config/packages}
% import custom math commands
\usepackage{config/commands}

% end preamble
\begin{document}

% document title
\noindent
\hfill Calvin Sprouse \hfill PHYS 475 Homework 2 \hfill 2024 April 10 \hfill
\bigskip

% homework problems begin
\begin{enumerate}
\item \begin{enumerate}[label=(\alph*)]
\item Normalize \(R_{2,0}\) for hydrogen and construct the wavefunction \(\psi_{2,0,0}\). \bigskip

% From Griffiths Table 4.7,
% \[R_{2,0} = \frac{1}{\sqrt{2}}a^{-3/2}\exp\left[-\frac{r}{2a}\right]\left(1 - \frac{1}{2}\frac{r}{a}\right).\]
% Let \(AR_{2,0}\) be normalized for some \(A\). Then,
% \begin{align*}
% 1 &= \bint{0}{\infty}{\left| AR_{2,0} \right|^2 r^2}{r}
% \\&= A^2 \bint{0}{\infty}{\left(rR_{2,0}\right)^2}{r}
% \\&= A^2 \bint{0}{\infty}{\frac{r^2}{2}a^{-3}\exp\left[-\frac{r}{a}\right]\left(1 - \frac{r}{2a}\right)^2}{r}
% \\&= A^2 \frac{1}{2a^3}\bint{0}{\infty}{r^2\exp\left[-\frac{r}{a}\right]\left(1-\frac{r}{2a}\right)^2}{r}
% \\&= A^2 \frac{1}{2a^3} \bint{0}{\infty}{\left[ r^2e^{-r/a} - \frac{1}{a}r^3e^{-r/a} + \frac{1}{4a^2}r^4e^{-r/a}\right]}{r},
% \end{align*}
% where each of these three integrals are of the form of integral (7) on the provided integral table; that is,
% \[\bint{0}{\infty}{x^n e^{-ax}}{x}=\frac{\Gamma(n+1)}{a^{n+1}}
% \Rightarrow \bint{0}{\infty}{r^n e^{-r/a}}{r} = n! a^{n+1}.\]
% Then,
% \begin{align*}
% 1 &= A^2 \frac{1}{2a^3} \left(3!a^3 - 4!a^3 + \frac{1}{4}5!a^3\right)
% \\&= A^2 \frac{1}{2} \left(6 - 24 + 30\right)
% \\&= 6A^2,
% \end{align*}
% which is satisfied for \(A=\sqrt{1/6}\).

From Griffiths Eq.\ 4.82,
\[R_{2,0} = \frac{c_0}{2a}\left(1- \frac{r}{2a}\right)\exp\left[-\frac{r}{2a}\right],\]
where \(c_0\) is the normalization constant. Then,
\begin{align*}
1 &= \bint{0}{\infty}{\left| R_{2,0} \right|^2 r^2}{r}
\\&= \left(\frac{c_0}{2a}\right)^2 \bint{0}{\infty}{\left(1 - \frac{r}{2a}\right)^2 \exp\left[-\frac{r}{a}\right]r^2}{r}
\\&= \left(\frac{c_0}{2a}\right)^2 \left(
    \bint{0}{\infty}{r^2 e^{-r/a}}{r}
    - \frac{1}{a}\bint{0}{\infty}{r^3 e^{-r/a}}{r}
    + \frac{1}{4a^2}\bint{0}{\infty}{r^4 e^{-r/a}}{r}
    \right),
\end{align*}
where all three integrals are of the form of integral (7) from the provided integral table; that is,
\[\bint{0}{\infty}{x^n e^{-ax}}{x}
=\frac{\Gamma(n+1)}{a^{n+1}}
\Rightarrow \bint{0}{\infty}{r^{n} e^{-r/a}}{r}
= n! a^{n+1}.\]
Then,
\begin{align*}
1 &= \left(\frac{c_0}{2a}\right)^2 \left(2!a^3 - \frac{1}{a}3!a^4 + \frac{1}{4a^2}4!a^5\right)
\\&= \left(\frac{c_0}{2a}\right)^2 \left(2a^3 - 6a^3 + 6a^3\right)
\\&= c_0^2 \frac{1}{4a^2}2a^3
\\&= c_0^2 \frac{a}{2},
\end{align*}
which is satisfied for
\[c_0 = \sqrt{\frac{2}{a}}.\]
Thus,
\[R_{2,0} = \frac{a^{-3/2}}{\sqrt{2}}\left(1-\frac{r}{2a}\right)\exp\left[-\frac{r}{2a}\right].\]
The wavefunction, \(\psi_{2,0,0}\), is given by
\[\psi_{2,0,0} = R_{2,0}Y_0^0,\]
where \(Y_0^0\) is given by Griffiths Table 4.3 as
\[Y_0^0 = \sqrt{\frac{1}{4\pi}}.\]
Thus,
\[\psi_{2,0,0} = \frac{a^{-3/2}}{\sqrt{8\pi}}\left(1-\frac{r}{2a}\right)\exp\left[-\frac{r}{2a}\right].\]

\bigskip
\item Normalize \(R_{2,1}\) for hydrogen and construct the wavefunction \(\psi_{2,1,1}, \psi_{2,1,0}\), and \(\psi_{2,1,-1}\). \bigskip

% From Griffiths Table 4.7,
% \[R_{2,1}
% = \frac{1}{2\sqrt{6}}a^{-3/2}\exp\left[-\frac{r}{2a}\right]\left(\frac{r}{a}\right)
% = \frac{1}{2\sqrt{3}}\left(\frac{a}{r}-\frac{1}{2}\right)R_{2,0}
% .\]

From Griffiths Eq.\ 4.83,
\[R_{2,1}
= \frac{c_0}{4a^2}r\exp\left[-\frac{r}{2a}\right],\]
where \(c_0\) is the normalization constant. Then,
\begin{align*}
1 &= \bint{0}{\infty}{\left| R_{2,1} \right|^2 r^2}{r}
\\&= \left(\frac{c_0}{4a^2}\right)^2 \bint{0}{\infty}{r^4\exp\left[-\frac{r}{a}\right]}{r},
\end{align*}
which is an integral of the same form as in part (a). Thus,
\begin{align*}
1 &= \left(\frac{c_0}{4a^2}\right)^2 4!a^5
\\&= c_0^2 a \frac{3}{2},
\end{align*}
which is satisfied for
\[c_0 = \sqrt{\frac{2}{3a}}.\]
Thus,
\[R_{2,1} = \frac{a^{-5/2}}{\sqrt{24}}r \exp\left[-\frac{r}{2a}\right].\]
The wavefunctions, \(\psi_{2,1,1}, \psi_{2,1,0}\), and \(\psi_{2,1,-1}\) are given by
\begin{align*}
\psi_{2,1,1} &= R_{2,1}Y_1^1, \\
\psi_{2,1,0} &= R_{2,1}Y_1^0, \\
\psi_{2,1,-1} &= R_{2,1}Y_1^{-1}.
\end{align*}
The angular components of the wavefunctions, \(Y_1^1\), \(Y_1^0\), and \(Y_1^{-1}\), are given by Griffiths Table 4.3 as
\begin{align*}
Y_1^1 &= -\sqrt{\frac{3}{8\pi}}\sin[\theta]\exp\left[i\phi\right], \\
Y_1^0 &= \sqrt{\frac{3}{4\pi}}\cos[\theta], \\
Y_1^{-1} &= \sqrt{\frac{3}{8\pi}}\sin[\theta]\exp\left[-i\phi\right].
\end{align*}
Then,
\begin{align*}
\psi_{2,1,1} &= -\frac{a^{-5/3}}{\sqrt{64\pi}}r\exp\left[i\phi-\frac{r}{2a}\right]\sin[\theta],\\
\psi_{2,1,0} &= \frac{a^{-5/3}}{\sqrt{32\pi}}r\exp\left[-\frac{r}{2a}\right]\cos[\theta],\\
\psi_{2,1,-1} &= \frac{a^{-5/3}}{\sqrt{64\pi}}r\exp\left[-\left(i\phi+\frac{r}{2a}\right)\right]\sin[\theta].
\end{align*}

\end{enumerate}
\pagebreak
\item \begin{enumerate}[label=(\alph*)]
\item Determine \(\langle r \rangle\) and \(\langle r^2 \rangle\) for an electron in the ground state of the hydrogen atom. Express solutions in terms of the Bohr radius, \(a\). \bigskip

The ground state wavefunction for hydrogen is given by \(\psi_{1,0,0}\) where from Griffiths Table 4.3 and 4.7,
\[\psi_{1,0,0} = R_{1,0}Y_0^0 = 2a^{-3/2}e^{-r/a}\sqrt{\frac{1}{4\pi}} = \frac{1}{\sqrt{\pi a^3}}e^{-r/a}.\]
Then,
\begin{align*}
\expval{r^n} &= \bint{0}{2\pi}{\bint{0}{\pi}{\bint{0}{\infty}{\left(\psi_{1,0,0}^* r^n \psi_{1,0,0}\right) r^2\sin(\theta)}{r}}{\theta}}{\phi}
\\&= \frac{4\pi}{\pi a^3} \bint{0}{\infty}{r^{n+2} \exp\left[-2\frac{r}{a}\right]}{r},
\end{align*}
where this integral is of the form of integral (7) from the provided integral table; that is,
\[\bint{0}{\infty}{x^n e^{-ax}}{x}=\frac{\Gamma(n+1)}{a^{n+1}}
\Rightarrow \bint{0}{\infty}{r^{n+2} e^{-2r/a}}{r} = (n+3)! \left(\frac{a}{2}\right)^{n+2}.\]
Thus,
\[\expval{r^n} = \frac{a^n}{2^{(n+1)}}(n+2)!.\]
Therefore,
\begin{align*}
\expval{r} &= \frac{3}{2}a, \\
\expval{r^2} &= 3a^2 .
\end{align*}

\item Determine \(\langle x \rangle\) and \(\langle x^2 \rangle\) for an electron in the ground state of the hydrogen atom. If the symmetry of the ground state is exploited, there will not be any new integration for this calculation. \bigskip



\pagebreak
\item Determine \(\langle x^2 \rangle\) for an electron in a hydrogen atom in the state \(n=2\), \(l=1\), \(m=1\). It is helpful to use the fact that \(x=r\sin(\theta)\cos(\phi)\). \bigskip

The wavefunction of this state is given by \(\psi_{2,1,1}\) where from Griffiths Table 4.3 and 4.7
\[\psi_{2,1,1} = R_{1,0}Y_1^1 = -\frac{1}{2\sqrt{6}}a^{-3/2}\frac{r}{a}\exp\left[-\frac{r}{2a}\right]\sqrt{\frac{3}{8\pi}}\sin[\theta]\exp\left[i\phi\right].\]
Then,
\begin{align*}
\expval{x^2} &= \bint{0}{2\pi}{\bint{0}{\pi}{\bint{0}{\infty}{
    \left(\psi_{2,1,1}^* x^2 \psi_{2,1,1}\right) r^2\sin(\theta)
}{r}}{\theta}}{\phi}
\\&= \bint{0}{2\pi}{\bint{0}{\pi}{\bint{0}{\infty}{
    \left(\psi_{2,1,1}^* \psi_{2,1,1}\right) r^2(\sin(\theta))^2(\cos(\theta))^2r^2\sin(\theta)
}{r}}{\theta}}{\phi}
\\&= \bint{0}{2\pi}{\bint{0}{\pi}{\bint{0}{\infty}{
    \frac{1}{24a^3}\left(\frac{r}{a}\right)^2e^{-r/a}\frac{3}{8\pi} r^4(\sin[\theta])^5(\cos[\theta])^2
}{r}}{\theta}}{\phi}
\\&= \frac{1}{64\pi a^5} \bint{0}{2\pi}{\bint{0}{\pi}{\bint{0}{\infty}{
    e^{-r/a} r^6 (\sin[\theta])^5 (\cos[\theta])^2
}{r}}{\theta}}{\phi}
\\&= \frac{1}{64\pi a^5}
\bint{0}{2\pi}{\left(\cos[\phi]\right)^2}{\phi}
\bint{0}{\pi}{\left(\sin[\theta]\right)^5}{\theta}
\bint{0}{\infty}{r^6\exp\left[-\frac{r}{a}\right]}{r},
\end{align*}
where we have an integral of the form of integral (7) from the provided integral table; that is,
\[\bint{0}{\infty}{x^n e^{-ax}}{x}=\frac{\Gamma(n+1)}{a^{n+1}}
\Rightarrow \bint{0}{\infty}{r^{6} e^{-r/a}}{r} = 6! a^7.\]
Then,
\[\expval{x^2} = \frac{45}{4\pi}a^2 \bint{0}{2\pi}{\left(\cos[\phi]\right)^2}{\phi}
\bint{0}{\pi}{\left(\sin[\theta]\right)^5}{\theta},\]
where the remaining integrals are of the form that Mathematica can solve\footnote{I could too, but would rather not apply integration by parts many times over.}.
Then,
\[\expval{x^2} = \frac{45}{4\pi}a^2 \frac{16}{15} \pi = 12 a^2.\]

\end{enumerate}
\pagebreak
\item \begin{enumerate}[label=(\alph*)]
\item Starting with \([r_i,p_j]=-[p_i,r_j]=i\hbar\delta_{ij}\) and \([r_i,r_j]=[p_i,p_j]=0\), where the index \(i\) stands for \(x,y\), or \(z\), and \(r_x=x\), \(r_y=y\), \(r_z=z\), work out the following commutator relations:
\begin{alignat*}{3}
[L_z,x] & =i\hbar y, & \qquad\qquad [L_z,y] & =i\hbar x, & \qquad\qquad [L_z,z] & =0, \\
[L_z,p_x]&=i\hbar p_y, & [L_z, p_y]&=i\hbar p_x, & [L_z, p_z]&=0.
\end{alignat*} \bigskip



\item Use the results from part (a) and the definitions
\begin{align*}
L_x&=yp_z-zp_y, \\
L_y&=zp_x-xp_z, \\
L_z&=xp_y-yp_x,
\end{align*}
to obtain \([L_z,L_x]=i\hbar L_y\). \bigskip


\item Evaluate the commutators \([L_z, r^2]\) and \([L_z, p^2]\), where \(r^2 = x^2 + y^2 + z^2\) and \(p^2 = p^2_x + p^2_y + p^2_z\). \bigskip



\item Show that the Hamiltonian, \[\hat{H}=\frac{\hat{p}^2}{2m} + \hat{V},\] commutes with all three components of \(\hat{\vec{L}}\) if \(\hat{V}\) depends only on \(r\). \bigskip



\end{enumerate}
\end{enumerate}
\end{document}
