% document style header
\documentclass[a4paper, 12pt]{config/homework}

% import default packages
\usepackage{config/defpackages}
% import custom math commands
\usepackage{config/domath}

% end preamble
\begin{document}

% document title
\noindent
\begin{tabularx}{\textwidth}{>{\centering\arraybackslash}X>{\centering\arraybackslash}X>{\centering\arraybackslash}X}
Calvin Sprouse & PHYS475 Homework 1 & 2024 April 03\\
\midrule
\end{tabularx}

% homework problems begin
\begin{enumerate}
\item Use separation of variables in cartesian coordinates to solve the infinite cubical well. The potential of the infinite cubical well is
\[V(x,y,z)=\begin{cases}
0, & \text{if}\ x,y,z\ \text{are all between 0 and}\ a, \\
\infty, & \text{otherwise}.
\end{cases}\]
\begin{enumerate}[label=(\alph*.)]
\item Find the stationary states and their corresponding energies. Apply boundary conditions at the boundaries of the well. Note that solutions should depend on three distinct quantum numbers.

We begin with the time-independent Schr{\"o}dinger equation (TISE),
\[-\frac{\hbar^2}{2m}\nabla^2\psi + V\psi = E\psi,\]
where by our choice of cartesian coordinates we define \(\nabla\) as
\[\nabla^2 = \diffp[2]{}{x} + \diffp[2]{}{y} + \diffp[2]{}{z}.\]
We have two regions: (i.) outside the box, and (ii.) inside the box. Outside the box, \(V(x,y,z)=\infty\), we have the wavefunction
\[\psi_i(x,y,z)=0.\]
Inside the box, \(V(x,y,z)=0\), the TISE can be expressed as
\[-\frac{\hbar^2}{2m}\left( \diffp[2]{\psi}{x} + \diffp[2]{\psi}{y} + \diffp[2]{\psi}{z} \right) = E\psi.\]
We assume that \(\psi\) has separable solutions in all three dimensions; that is,
\[\psi(x,y,z) = X(x)Y(y)Z(z).\]
Substituting this expression for \(\psi\), dividing both sides by \(XYZ\), replacing our partial derivatives with ordinary derivatives, and rearranging the \(\hbar\) term yields
\[-\frac{1}{X}\diff[2]{X}{x} - \frac{1}{Y}\diff[2]{Y}{y} - \frac{1}{Z}\diff[2]{Z}{z} = \frac{2m}{\hbar^2}E.\]
Each term on the left is a function of only one variable and is independent of each other. Therefore, each term on the left side must be constant. We use our typical \(k^2\) constant and define one for each term: \(k^2_x, k^2_y, k^2_z\). Looking into the \(x\)-term we say
\[-\frac{1}{X}\diff[2]{X}{x} = k^2_x\]
which is a simple differential equation solved with exponential functions or equivalently sine and cosine functions. Since we are dealing with a trapt particle, sine and cosine functions are a natural choice. Notice that each of these terms has the same structure and boundary conditions so we need only solve one. Continuing with the \(x\)-term we have a general solution,
\[X(x) = A_x\sin(k_x x) + B_x\cos(k_x x).\]
Our boundary conditions are (1.) \(X(0)=0\), and (2.) \(X(a)=0\). Our first condition requires that \(B_x=0\). Our second condition then requires that
\[k_x = \frac{n_x\pi}{a},\ n_x\in\mathbb{N}.\]
Thus, our solution for the \(x\)-term can be written as
\[X(x) = A_x\sin\left(\frac{n_x\pi}{a}x\right),\ n_x\in\mathbb{N}.\]
The \(y\) and \(z\) terms have the same structure:
\begin{align*}
Y(y) &= A_y\sin\left(\frac{n_y\pi}{a}y\right),\ n_y\in\mathbb{N}; \\
Z(z) &= A_z\sin\left(\frac{n_z\pi}{a}z\right),\ n_z\in\mathbb{N}.
\end{align*}
Then our wavefunction inside the box, \(\psi_{ii}\), can be written as
\[\psi_{ii}(x,y,z) = A_x A_y A_z\sin\left(\frac{n_x\pi}{a}x\right)\sin\left(\frac{n_y\pi}{a}y\right)\sin\left(\frac{n_z\pi}{a}z\right).\]
Since each wavefunction resembles a one-dimensional oscillator we can simply take the normalization constants from that system for each term:
\[A_x A_y A_z = \sqrt{\frac{2}{a}}\sqrt{\frac{2}{a}}\sqrt{\frac{2}{a}} = \left(\frac{2}{a}\right)^{3/2}.\]
Thus,
\[\psi_{ii}(x,y,z) = \left(\frac{2}{a}\right)^{3/2}\sin\left(\frac{n_x\pi}{a}x\right)\sin\left(\frac{n_y\pi}{a}y\right)\sin\left(\frac{n_z\pi}{a}z\right).\]
Finally, we recognize that since each term was equal to some constant, so too was the right hand side of our equation. Expressed with some rearrangement and substitution we find
\[E=\frac{1}{2m}\left(\frac{\hbar\pi}{a}\right)^2\left(n_x^2 + n_y^2 + n_z^2 \right),\ n_x,n_y,n_z\in\mathbb{N}.\]

\pagebreak
\item Call the distinct energies \(E_1,E_2,E_3\), etc., in order of increasing energy. Note that the subscripted integer is not a quantum number here; instead, these number simply represent the lowest energy, \(E_1\), second lowest energy, \(E_2\), etc. Determine what \(E_1,E_2,E_3,E_4,E_5\), and \(E_6\) are in terms of \(\hbar,m,\pi\), and \(a\). Determine the degeneracy of each of these energies.

Since \(\hbar,m,\pi\), and \(a\) are constant we see that the energy level is only changed by the three quantum numbers \(n_x,n_y\), and \(n_z\). Specifically, the energy is determined by the sum of the quantum numbers and the number of ways to sum the quantum numbers equivalently determines the degeneracy.

\(E_1\) is the energy for which the quantum numbers sum to 3,
\[E_1 = 3 \left(\frac{\hbar\pi}{a}\right)^2\frac{1}{2m}.\]
There is only one configuration so \(E_1\) has no degeneracy.

\(E_2\) is the energy for which the quantum numbers sum to 6,
\[E_2 = 6 \left(\frac{\hbar\pi}{a}\right)^2\frac{1}{2m}. \]
There are three configurations, any combination of two quantum numbers being one and one quantum number being 2, so this state has three-fold degeneracy.

\(E_3\) is the energy for which the quantum numbers sum to 9,
\[E_3 = 9\left(\frac{\hbar\pi}{a}\right)^2\frac{1}{2m}.\]
This state has three-fold degeneracy.

\(E_4\) is the energy for which the quantum numbers sum to 11,
\[E_4 = 11 \left(\frac{\hbar\pi}{a}\right)^2\frac{1}{2m}.\]
This state has three-fold degeneracy, every combination of two quantum numbers being 1 and one quantum number being 3.

\(E_5\) is the energy for which the quantum numbers sum to 12,
\[E_5 = 12 \left(\frac{\hbar\pi}{a}\right)^2\frac{1}{2m}.\]
This state has no degeneracy, all quantum numbers must be 2.

\(E_6\) is the energy for which the quantum numbers sum to 14,
\[E_6 = 14 \left(\frac{\hbar\pi}{a}\right)^2\frac{1}{2m}.\]
This state has six-fold degeneracy because we allow every combination of one quantum number being 1, one quantum number being 2, and one quantum number being 3.

\item What is the degeneracy of \(E_{14}\), and what is unusual about this case?

The degeneracy of \(E_{14}\) is four. The 14th highest energy level is achieved for all states being equal to 3,
\[3^2 + 3^2 + 3^2 = 27.\]
This would normally result in a degeneracy of one; however, the combination of one quantum number being 5 and the other two being 1 also sums to 27,
\[5^2 + 1^2 + 1^2 = 27.\]
This introduces three more possible configurations of quantum numbers. This is unusual because any other time we require that all states be the same, such as in \(E_1\), we find no degeneracy.

\end{enumerate}
\pagebreak
\item Construct the spherical harmonics \(Y_0^0(\theta,\phi)\) and \(Y_2^1(\theta,\phi)\) using the formula for spherical harmonics,
\[Y_\ell^m(\theta,\phi) = \epsilon\sqrt{\frac{(2\ell+1)}{4\pi}\frac{(\ell-|m|)!}{(\ell+|m|)!}}\exp(im\phi)P_\ell^m\left(\cos(\theta)\right),\]
where the constant \(\epsilon\) is defined to be
\[\epsilon = \begin{cases}
(-1)^m, & m > 0, \\ 1, & m \le 0,
\end{cases}\]
the Rodrigues formula for the Legendre polynomial,
\[P_\ell^m(x) \equiv \left(1-x^2\right)^{|m|/2}\diff[|m|]{}{x}P_\ell(x).\]
Check that solutions for \(Y_0^0(\theta,\phi)\) and \(Y_2^1(\theta,\phi)\) are normalized and orthogonal.

\begin{align*}
Y_0^0(\theta, \phi) &= \epsilon\sqrt{\frac{(0+1)}{4\pi}\frac{(0-|0|)!}{(0+|0|)!}}\exp(0)P_0^0\left(\cos(\theta)\right)
\\&= \sqrt{\frac{1}{4\pi}}\left((1-(\cos\theta)^2)^0\diff[0]{}{x}P_0(\cos(\theta))\right)
\\&= \sqrt{\frac{1}{4\pi}}.
\\&\\
Y_2^1(\theta, \phi) &= \epsilon \sqrt{\frac{(2(2)+1)}{4\pi}\frac{(2-1)!}{(2+1)!}}\exp(i\phi)P_2^1\left(\cos(\theta)\right)
\\&= -\sqrt{\frac{5}{4\pi}\frac{1}{6}}\exp(i\phi)
\left( \left(1-(\cos\theta)^2\right)^{1/2} \diff{}{\cos\theta}P_2(\cos\theta) \right)
\\&= -\sqrt{\frac{5}{24\pi}}\exp(i\phi) \sqrt{1-(\cos(\theta))^2} \diff{}{\cos\theta}\frac{1}{2}\left(3(\cos\theta)^2-1\right)
\\&= -\sqrt{\frac{5}{24\pi}}\exp(i\phi)\sin(\theta)3\cos(\theta)
\\&= -3\sqrt{\frac{5}{24\pi}}\exp(i\phi)\sin(\theta)\cos(\theta)
\\&= -\sqrt{\frac{15}{8\pi}}\exp(i\phi)\sin(\theta)\cos(\theta).
\end{align*}

Checking for normalization:
\begin{align*}
\bint{0}{2\pi}{\bint{0}{\pi}{|Y_0^0(\theta,\phi)|^2\sin(\theta)}{\theta}}{\phi} &= \frac{1}{4\pi}\bint{0}{2\pi}{\bint{0}{\pi}{\sin(\theta)}{\theta}}{\phi}
\\&= \frac{1}{4\pi} 2 (2\pi)
\\&= 1.
\\&\\
\bint{0}{2\pi}{\bint{0}{\pi}{|Y_2^1(\theta,\phi)|^2\sin(\theta)}{\theta}}{\phi} &= \frac{15}{8\pi} \bint{0}{\pi}{\sin(\theta)^2\cos(\theta)^2\sin(\theta)}{\theta}\bint{0}{2\pi}{e^{i\phi}e^{-i\phi}}{\phi}
\\&= \frac{15}{4}\bint{0}{\pi}{\sin(\theta)^2(1-\sin(\theta)^2)\sin(\theta)}{\theta}
\\&= \frac{15}{4}\left(\bint{0}{\pi}{\sin(\theta)^3}{\theta} - \bint{0}{\pi}{\sin(\theta)^5}{\theta}\right)
\\&= \frac{15}{4}\left(\frac{4}{3} - \frac{16}{15}\right)
\\&= 5 - 4
\\&= 1.
\end{align*}

Checking for orthogonality:
\begin{align*}
\bint{0}{2\pi}{\bint{0}{\pi}{Y_0^{0*}Y_2^1\sin(\theta)}{\theta}}{\phi} &= -\sqrt{\frac{1}{4\pi}}\sqrt{\frac{15}{8\pi}}\bint{0}{\pi}{\sin(\theta)^2\cos(\theta)}{\theta}\bint{0}{2\pi}{\exp(i\phi)}{\phi}
\\&= 0.
\end{align*}
We need only observe that the \(\phi\) integral integrates the complex exponential \(\exp(i\phi)\) over a full circle of arc is is therefore 0.

\end{enumerate}
\end{document}
