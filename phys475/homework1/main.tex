% document style header
\documentclass[a4paper, 12pt]{config/homework}

% import default packages
\usepackage{config/defpackages}
% import custom math commands
\usepackage{config/domath}

% end preamble
\begin{document}

% document title
\noindent
\begin{tabularx}{\textwidth}{>{\centering\arraybackslash}X>{\centering\arraybackslash}X>{\centering\arraybackslash}X}
Calvin Sprouse & PHYS475 Homework 1 & 2024 April 03\\
\midrule
\end{tabularx}

% homework problems begin
\begin{enumerate}
\item Use separation of variables in cartesian coordinates to solve the infinite cubical well. The potential of the infinite cubical well is
\[V(x,y,z)=\begin{cases}
0, & \text{if}\ x,y,z\ \text{are all between 0 and}\ a, \\
\infty, & \text{otherwise}.
\end{cases}\]
\begin{enumerate}[label=(\alph*.)]
\item Find the stationary states and their corresponding energies. Apply boundary conditions at the boundaries of the well. Note that solutions should depend on three distinct quantum numbers.



\item Call the distinct energies \(E_1,E_2,E_3\), etc., in order of increasing energy. Note that the subscripted integer is not a quantum number here; instead, these number simply represent the lowest energy, \(E_1\), second lowest energy, \(E_2\), etc. Determine what \(E_1,E_2,E_3,E_4,E_5\), and \(E_6\) are in terms of \(\hbar,m,\pi\), and \(a\). Determine the degeneracy of each of these energies.



\item What is the degeneracy of \(E_{14}\), and what is unusual about this case?



\end{enumerate}
\pagebreak
\item Construct the spherical harmonics \(Y_0^0(\theta,\phi)\) and \(Y_2^1(\theta,\phi)\) using the formula for spherical harmonics,
\[Y_\ell^m(\theta,\phi) = \epsilon\sqrt{\frac{(2\ell+1)}{4\pi}\frac{(\ell-|m|)!}{(\ell+|m|)!}}\exp(im\phi)P_\ell^m\left(\cos(\theta)\right),\]
where the constant \(\epsilon\) is defined to be
\[\epsilon = \begin{cases}
(-1)^m, & m \ge 0, \\ 1, & m \le 0,
\end{cases}\]
the Rodrigues formula for the Legendre polynomial,
\[P_\ell^m(x) \equiv \left(1-x^2\right)^{|m|/2}\diff[|m|]{}{x}P_\ell(x).\]
Check that solutions for \(Y_0^0(\theta,\phi)\) and \(Y_2^1(\theta,\phi)\) are normalized and orthogonal.



\end{enumerate}
\end{document}
