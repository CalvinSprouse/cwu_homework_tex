% document style header
\documentclass[a4paper, 12pt]{config/homework}

% import default packages
\usepackage{config/packages}
\usepackage{config/commands}

% end preamble
\begin{document}

% document title
\noindent
\hfill Calvin Sprouse \hfill PHYS 475 Homework 2 \hfill 2024 April 10 \hfill
\bigskip

% homework problems begin
\begin{enumerate}
\item An electron is in the spin state
\[\chi = A \matr{3i \\ 4}.\]
\begin{enumerate}
\item Normalize \(\chi\) to determine A.
\bigskip \\
We begin by recognizing that
\[\chi = A \matr{3i \\ 4} = A\left(3i\bra*{\uparrow} + 4\ket*{\downarrow}\right).\]
Then,
\[ 1 = \left| \chi \right|^2
= A^*A \left((3i)(-3i)\braket*{\uparrow}{\downarrow} + (4)(4)\braket*{\downarrow}{\downarrow} + 0 + 0\right)
= A^2 \left(9 + 16\right), \]
which is satisfied for \(A = 1/5\).

\bigskip
\item Determine \(\expval{S_x}\), \(\expval{S_y}\), \(\expval{S_z}\).
\bigskip \\
From Griffiths Eq.\ 4.145 and Eq.\ 4.147,
\[
\widehat{S_x} = \frac{\hbar}{2}\matr{0 & 1 \\ 1 & 0}, \quad
\widehat{S_y} = \frac{\hbar}{2}\matr{0 & -i \\ i & 0}, \quad
\widehat{S_z} = \frac{\hbar}{2}\matr{1 & 0 \\ 0 & -1}.
\]
Then, in general,
\[
\expval{S_a}
= \bra*{\chi}\,\widehat{S_a}\,\ket*{\chi}
= \frac{1}{25}\matr{-3i & 4} \widehat{S_a} \matr{3i \\ 4},
\]
where \(a\) represents \(x\), \(y\), or \(z\).
Thus,
\begin{align*}
\expval{S_x} &= \frac{1}{25}\frac{\hbar}{2}\matr{-3i & 4}\matr{0 & 1 \\ 1 & 0}\matr{3i \\ 4} = 0,
\\ \expval{S_y} &= \frac{1}{25}\frac{\hbar}{2}\matr{-3i & 4}\matr{0 & -i \\ i & 0}\matr{3i \\ 4} = -\frac{24}{50}\hbar,
\\ \expval{S_z} &= \frac{1}{25}\frac{\hbar}{2}\matr{-3i & 4}\matr{1 & 0 \\ 0 & -1}\matr{3i \\ 4} = -\frac{7}{50}\hbar.
\end{align*}

\bigskip
\item Determine the uncertainties \(\sigma_{S_x}\), \(\sigma_{S_y}\), and \(\sigma_{S_x}\).
\bigskip
In general,
\[\sigma_a = \sqrt{\expval{a^2} - \expval{a}^2},\]
where \(a\) represents some observable. Our square spin operators are then
\[
\widehat{S_x^2} = \frac{\hbar^2}{4}\matr{1 & 0 \\ 0 & 1}, \quad
\widehat{S_y^2} = \frac{\hbar^2}{4}\matr{1 & 0 \\ 0 & 1}, \quad
\widehat{S_z^2} = \frac{\hbar^2}{4}\matr{1 & 0 \\ 0 & 1}.
\]
Since the operators are equal, their expectation values on \(\chi\) will be equal. We also notice that the matrix for each operator is the identity matrix. Thus,
\[\expval{S_x^2} = \expval{S_y^2} = \expval{S_z^2} = \frac{\hbar^2}{4}\bra*{\chi}\widehat{S^2_x}\ket*{\chi} = \frac{\hbar^2}{4}\braket*{\chi}{\chi} = \frac{\hbar^2}{4}.\]
Therefore,
\begin{align*}
\sigma_{S_x} &= \sqrt{\expval{S_x^2} - \expval{S_x}^2} = \sqrt{\frac{\hbar^2}{4}-0} = \frac{\hbar}{2}, \\
\sigma_{S_y} &= \sqrt{\expval{S_y^2} - \expval{S_y}^2} = \sqrt{\frac{\hbar^2}{4} - \left(-\frac{12}{25}\hbar\right)^2} = \frac{7}{50}\hbar \\
\sigma_{S_z} &= \sqrt{\expval{S_z^2} - \expval{S_y}^2} = \sqrt{\frac{\hbar^2}{4} - \left(-\frac{7}{50}\hbar\right)^2} = \frac{12}{25}\hbar.
\end{align*}

\bigskip
\item Confirm that the results are consistent with all three uncertainty principles; that is,
\[\sigma_{S_x}\sigma_{S_y} \ge \frac{\hbar}{2}\left|\expval{S_z}\right|,\]
and the cyclic permutations.
\bigskip
Notice from part (c),
\[
\sigma_{S_x} = \frac{\hbar}{2}, \quad
\sigma_{S_y} = \frac{7}{50}\hbar = \left|\expval{S_z}\right|, \quad
\sigma_{S_z} = \frac{12}{50}\hbar = \left|\expval{S_y}\right|.
\]
Then,
\begin{align*}
\sigma_{S_x}\sigma_{S_y} \ge \frac{\hbar}{2}\left|\expval{S_z}\right|
&\Rightarrow \frac{\hbar}{2}\left|\expval{S_z}\right| \ge \frac{\hbar}{2}\left|\expval{S_z}\right|, \\
\sigma_{S_y}\sigma{S_z} \ge \frac{\hbar}{2}\left|\expval{S_x}\right|
&\Rightarrow \left(\frac{7}{50}\hbar\right)\left(\frac{12}{50}\hbar\right) \ge 0, \\
\sigma_{S_z}\sigma{S_x} \ge \frac{\hbar}{2}\left|\expval{S_y}\right|
&\Rightarrow \left|\expval{S_y}\right|\frac{\hbar}{2} \ge \frac{\hbar}{2}\left|\expval{S_y}\right|.
\end{align*}

\end{enumerate}
\pagebreak
\item For a generalized spinor,
\[\chi = a\chi_{+,z} + b\chi_{-,z} = a\bra*{\uparrow} + b\ket*{\downarrow},\]
where \(\chi_{+,z}\) is spin up and \(\chi_{-,z}\) is spin down, compute \(\expval{S_x}\), \(\expval{S_y}\), \(\expval{S_z}\), \(\expval{S_x^2}\), \(\expval{S_y^2}\), \(\expval{S_z^2}\). Check that
\[\expval{S_x^2} + \expval{S_y^2} + \expval{S_z^2} = \expval{S^2}.\]
\bigskip
From Griffiths Eq.\ 4.145 and Eq.\ 4.147,
\[
\widehat{S_x} = \frac{\hbar}{2}\matr{0 & 1 \\ 1 & 0}, \quad
\widehat{S_y} = \frac{\hbar}{2}\matr{0 & -i \\ i & 0}, \quad
\widehat{S_z} = \frac{\hbar}{2}\matr{1 & 0 \\ 0 & -1}.
\]
Then,
\[
\widehat{S_x^2} = \widehat{S_y^2} = \widehat{S_z^2} = \frac{\hbar^2}{4}\matr{1 & 0 \\ 0 & 1}.
\]
Consequently,
\[
\widehat{S^2} = \widehat{S_x^2} + \widehat{S_y^2} + \widehat{S_z^2} = \hbar^2\frac{3}{4}\matr{1 & 0 \\ 0 & 1}.
\]
Then,
\begin{align*}
   \expval{S_x} &= \bra*{\chi}\,\widehat{S_x}\,\ket*{\chi} = \frac{\hbar}{2}\matr{a^* & b^*}\matr{0 & 1 \\ 1 & 0}\matr{a \\ b} = \frac{\hbar}{2}\left(a^*b + b^*a\right),
\\ \expval{S_y} &= \bra*{\chi}\,\widehat{S_y}\,\ket*{\chi} = \frac{\hbar}{2}\matr{a^* & b^*}\matr{0 & -i \\ i & 0}\matr{a \\ b} = i\frac{\hbar}{2}\left(ab^* - a^*b\right),
\\ \expval{S_z} &= \bra*{\chi}\,\widehat{S_z}\,\ket*{\chi} = \frac{\hbar}{2}\matr{a^* & b^*}\matr{1 & 0 \\ 0 & -1}\matr{a \\ b} = \frac{\hbar}{2}\left(a^*a + b^*b\right).
\end{align*}
Then,
\[\expval{S_x^2} = \expval{S_y^2} = \expval{S_z^2} = \frac{\hbar^2}{4}.\]
Finally,
\[\expval{S^2} = \expval{S_x^2} + \expval{S_y^2} + \expval{S_z^2} = \hbar^2\frac{3}{4}.\]


\end{enumerate}
\end{document}
