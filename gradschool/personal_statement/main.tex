% document style header
\documentclass[a4paper, 12pt]{article}

% import packages
\usepackage[margin=2.25cm]{geometry}

% end preamble
\begin{document}

% TODO:
% Dr. Craig: implement the small changes recomended
% fix 180 TA to LA (and briefly VERY BRIEFLY explain LA)
% fix astronomy TA to Labratory Experience in Teaching class"
% Dr. White: tell the school what I will do for them
% run through grammarly

My joy of discovery began very early in life with robotics and aspirations of exploring Mars. Starting in first grade I joined a competitive robotics team. The next year, after demonstrating a knack for engineering, programming, and leadership, I was elected team captain. In middle school, I started teaching after school robotics classes. By high school, with the help of my robotics team, I developed and implemented after school robotics classes for my school district enabling every K-12 student to have access to robotics, for free. In my senior year, the culmination of 12 years leading and teaching robotics, my team and I reached over 500 students through volunteer outreach. All the while staying competitive in the engineering challenges presented to us.

I had my first taste of physics research at Central Washington University (CWU) when I took observational astronomy in Spring 2022. For the final project, I wrote Python scripts for the CWU 0.6 meter research telescope to autonomously observe eclipsing binary system 44 Boo for over 6 hours. The data, also processed in Python, was compared against published data to demonstrate the efficacy of my methods.

I broadened my research experience that summer at the CWU Computational Biophysics Lab where I was selected be a research assistant on an NSF grant. I began by creating visualizations in Python of the agent-based MatLab simulations used to study microtubule dynamics in axons of neurons. These visualizations would become essential tools to debug and communicate results to experimental collaborators around the country and international conferences. Upon returning to classes in the Fall I presented my work at PhysCon 2022 and, by nomination, at the Murdock College Science Research Conference (MCSRC). During the summer I was able to learn enough MatLab to work on the simulations in the Fall as well. I presented my contributions to understanding the effect of our simulation parameters on the self-organization of microtubules at the 2023 Biophysical Society Meeting in San Diego. Standing with my team in front of our poster, I remember feeling an incredible joy. It was as clear to me as it is now, I am meant to do research.

My passion for astronomy, however, demanded that I continue research on the telescope. I spent the next two quarters coding data reduction pipelines and  analysis scripts for a new spectrograph. Then I updated my previous work in automated observations of eclipsing binary 44 Boo to create a flexible piece of software for autonomously observing a variety of targets. As a test, I sampled a dozen eclipsing binary systems to characterize our telescopes precision and have since presented my results at two conferences, including, by nomination, at the 2023 MCSRC.

The year I ventured into the world of research I also returned to my love of teaching. With my past experience in robotics I was hired for two tutoring positions, math and physics. As a math tutor I would help students by drop-in and appointment with math from algebra to calculus and beyond in nearly any subject. I found myself learning with the student as I figured out new ways to teach concepts and dissected problems I had never seen before in subjects I never studied. As a physics tutor I would host drop-in tutoring hours tailored for introductory physics courses and would meet regularly with professors to discuss struggles and upcoming class content. Tutoring inspired me to do better in my classes and I eventually opened my sessions to upper-division courses as well. In returning to teaching my career became obvious, I realized I had to be a professor. Nowhere else could I meld my passions for research and teaching so perfectly with my love of physics.

With my goals set I realized I needed to take my tutoring a step further, I had to become a TA. That spring, when observational astronomy was offered again, I applied to TA for the course. I was accepted and told to pick a topic to teach throughout the quarter. Inspired by my experience in the class I decided I would teach naked eye astronomy: the identification of constellations and celestial coordinate systems using only the night sky. I prepared curriculum and gave lessons throughout the quarter using the night sky when I was able, and a planetarium otherwise. My conviction that I needed to become a professor only grew stronger in watching my students learn throughout the quarter. Some even approached me with questions about getting involved in undergraduate research and are now leading projects of their own. While the astronomy class had to end I knew my TA experience didn't. I signed up to TA for the introductory physics classes where I would spend considerable time in the classroom helping students with labs. I took up grading for the classes and watched my physics tutoring sessions grow from a couple of students a week to over twenty. I became a nexus of physics tutoring in the department. My passion and experience even lead me to be chosen as lead tutor at the math center where my responsibilities expanded to include training new tutors and analyzing data to request funds and better serve students.

My research interests became focused on the fundamental, at the logical conclusion of a series of one simple question: `why?' I looked outwards for the summer leading into my Senior year, and was selected for an REU position at the University of Wisconsin-Madison to study neutrino physics with IceCube. My project for the summer: a particularly high energy neutrino seen in IceCube not long before my arrival that needed analysis. This muon-neutrino had enough energy to catapult a muon beyond the confines of the detector. To determine the energy involved in this interaction I turned to Monte Carlo simulations of muons in ice. Leveraging my extensive experience in Python as both a powerful tool for simulations and data science I arrived at my conclusion: this neutrino is the highest energy neutrino ever observed. My conviction to pursue this career only grew, and once again my research interest focused. Neutrinos were my window into incredible physics. During my time at IceCube I attended every talk I could on neutrino oscillations, sterile neutrinos, I was fascinated by all the wonderful ways neutrinos made us question our models. What started as an exercise pushing the limits of my computational skills became my introduction to the world of theoretical physics. My work earned me a presentation at the 2024 American Astronomical Society meeting, another opportunity to peer into the frontiers of physics.

I returned from the summer inspired to take my physics further. With my Biophysics research I am investigating novel models of axon growth, a question inspired by observations made using my visualization software followed by a deep dive into literature. I'll be presenting my ongoing work at the 2024 Biophysical Society Meeting in Pennsylvania. I even joined a project to design and build an apparatus to measure the magnetocaloric effect. This arduino controlled measurement system will allow future CWU students to investigate renewable refrigerants in our solid state physics lab. I am also taking electives classes in probability and statistics after learning about the statistical models used to test beyond standard model physics of neutrinos. As I find myself drawn further into theoretical physics my need for advanced mathematics becomes clear, thus I am also taking sets and logic and plan to take courses in group theory during my final year here at CWU.

That brings me to the present. At the logical conclusion of a series of one simple question, `why', I find my answer: theoretical physics. From quantum gravity, particle physics, to theories in nuclear physics and cosmology I know this is what I want to study. At Boulder, where the High Energy Theory Group probes quantum gravity, cosmology, supersymmetry, beyond standard model physics, and so much more, I will be fully immersed in a wonderful world of graduate studies. I know one day I will become a great professor, make contributions to our understanding of the universe, and inspire future students to Be Boulder!

\end{document}
