% document style header
\documentclass[a4paper, 12pt]{article}

% import packages
\usepackage[margin=3cm]{geometry}

% end preamble
\begin{document}
All work and no play makes Calvin a dully boy. But all research and no work is what I want, I dont even need play, just research! I think Physics is really is really cool and I love questions! Me like why! Me study rock star and that which lie betwixt through number speak.

I have had a love of discovery since I was young. Starting in first grade I joined competitive FIRST robotics teams and quickly became team captain. Then in middle school I started teaching after school robotics classes with my Dad and fell in love with teaching. I brought that joy to my high school robotics team and together we would reach over 500 students a year in robotics classes. To this day, when I visit my home high school robotics team, I recognize the students from my time teaching their after school classes.

It was no surprise when I became a Physics student and Astronomy minor at Central Washington University (CWU). As soon as I was able to I took a class in astronomy research. I have always had a fascination for the stars and was finally able to look at them. I was hooked, and did my class research project on eclipsing binaries. Here I was once again able to use my Python skills to automate observations on a telescope. I didn't stop at the class though. I Applied these scripts and observations to characterize our research telescopes accuracy as a means to determine if we can see exoplanets. I have been able to present this research at two conferences. I was even able to work on creating data analysis scripts for an under construction spectrograph. To extend my use of computer science even further I have joined another student in design and build an apparatus to test the magnetocaloric effect from the ground up.

My early astronomy experience got me a job as a summer research student for my colleges Computational Biophysics Lab. Here I was once again able to join my love of discovery and previously obtained computer science skills to create visualizations in Python. These helped us improve our MatLab simulations and eventually became a part of our presentations. I then joined the team throughout the school year and worked on simulations of motor based interactions in axons, with potential applications in treating Alzheimer's. I was able to present my work at 6 conferences including PhysCon 2022, the Biophysical Society Meeting (BPS) 2023, and soon at the BPS 2024 meeting. I have learned so much just from presentation alone. Beyond the stimulating conversations at posters, and exposure to presentations, I have had significant practice in communicating science.

At every step of the way my desire to pursue knowledge has been motivated by a desire to spread it. I experience the same joy in teaching that I do in research. It was no surprise then, when I got a job as a Math Tutor my Junior year. Not only did I notice my own math skills developing, I noticed the skills of my students growing as I gained quite a few regular drop-ins. Very soon it wasn't just for math, I began tutoring physics as a PALs tutor. Beyond providing general tutoring services PALs tutors worked with professors directly to tailor tutoring for specific classes. There was only one tutoring position left, to be a TA. I had my first TA position as TA for the observational astronomy class I took the year prior. Here I developed lesson plans to teach students about observing the night sky with just their eyes. On the roof of the physics building my students learned how to identify constellations and apply celestial coordinates to the night sky. Now I am a TA for the introductory physics series, and spend considerable time in the classroom working with students. I have become a nexus point for tutoring physics at central, occupying all positions possible and reaching students at every point in their major.

As I went further in physics I discovered my love of theory in electromagnetic theory. I became fascinated by gravity, no surprise because of my love of cosmology. When my astronomy club organized a visit to nearby Laser Interferometer Gravitational-wave Observatory (LIGO), I knew I wanted to take my knowledge further. I pursued summer REU programs and accepted one at the University of Wisconsin Madison with the IceCube Neutrino Observatory. Here I used Python to calculate the energy of a particular high energy neutrino that IceCube spotted shortly before my arrival. That neutrino, as it turns out, is the highest energy neutrino ever seen, and work continues to find its astrophysical origins. I will be presenting this work at the American Astronomical Society January 2024 Meeting.

A thrilling conclusion on why I want a PhD, my love of research, my love of teaching, I can bring it all together! This is the job I want, nay the job I need. Then check that it implements my professors feedback. Do I talk enough about research. Am I clear in what I did while also getting into who I am and that I love this. Do "I" shine through. Am I clear that research is what I need to do. I should elaborate on my plan, that I want to teach something cool. That I want to learn theory. I should elaborate a bit on my math and what I love about it and how it connects to why I want to pursue theoretical physics.

\end{document}
