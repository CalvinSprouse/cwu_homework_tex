% document style header
\documentclass[a4paper, 12pt]{../../config/homework}

% import default packages
\usepackage{../../config/defpackages}

% import custom math commands
\usepackage{../../config/domath}

% end preamble
\begin{document}

% document title
\noindent
\begin{tabularx}{\textwidth}{>{\centering\arraybackslash}X>{\centering\arraybackslash}X>{\centering\arraybackslash}X}
\toprule
Calvin Sprouse & Personal Statement & 2023 November 29\\
\midrule
\end{tabularx}

% homework problems begin
Below is my personal statement draft. I found a lot of trouble fitting everything into one cohesive statement. What I have below is too long for many universitys but is a collection of all my thoughts.

Areas that I especially want feedback to focus on:
\begin{itemize}
    \item Where to cut down on content to meet potential limits,
    \item Where the flashy language heavily inhibits the communication of my ideas (in general I want the language to feel like it makes my statement into a story, not my statement into a bunch of words that say nothing),
    \item Where the language changes in a jolt, I copied some stuff nearly directly from older statements so there are times where it might feel like the literary equivalent to patchwork,
\end{itemize}

Some things I wanted to include but couldnt find room, advice here would also be appreciated:
\begin{itemize}
    \item My many presentations and the experience presenting research,
    \item My work on the spectrograph and electrocaloric aparatus',
    \item My passions outside of physics that make me me, climbing, building things, photography,
    \item My significant involvement in outreach with the department (I have a history of volunteer work outside of college too that I struggled to incorporate so this would fit here as well).
\end{itemize}

Any and all thoughts are appreciated! Thank you for your patience as I \textit{slowly} worked on this, writing about myself is a dreadful task such that everytime I sat down to do it I convinced myself to work on homework, sometimes weeks in advance, then leading me to be exhausted and feel that I had done enough for the day. I also fully recognize this is a lot to ask on short notice, I am going to continue editing it and will be sending my draft to many people beyond just my writers for a wide array of feedback.

% begin statement
\pagebreak
I am inherently captivated by the foundational 'whys' that drive the universe, an unyielding pursuit ignited by the allure of theoretical physics. This fascination begun amid the elegance of electromagnetism, where mathematical revelations unfolded within my electromagnetic theory class. Observing the pervasive influence of electromagnetic forces shaping our world and the enigmatic divergence of gravity stirred a deep curiosity. This inspired me to delve deeper into mathematics. I have branched beyond my physics curriculum to take mathematics classes. I am inexorably drawn to math, convinced it has the answers I need. This draw propels my aspiration toward theoretical physics, seeking to unravel the enigmas of particle physics and gravity, realms teeming with profound complexities and uncharted territories.

My journey began amidst the whirring mechanics of robotics, where I found not just a fascination for engineering but also a calling for imparting knowledge. Starting with 1st grade, I joined a FIRST Robotics team. The next year, I was the team captain. 10 years later, totaling 12 years of robotics experience, I developed and implemented an after-school volunteer service for my school district. Every student K-12 had access to an after-school robotics class. Whenever I visit my hometown of North Bend, Washington, and visit my old robotics team to mentor them, I am greeted by the aged faces of kids I previously taught in my after-school programs. On top of the after-school programs, I would run summer camps in robotics and programming with my Dad. There, in my first foray into teaching, I found another passion. All the joy of discovery that I remember experiencing when I was learning I could now impart onto others, and their joy became mine as I learned from teaching.

My academic quest surged forward, through high school I pursued AP classes in Physics and Computer Science. At home I would solve problems with code. Coding was endless expanse I could explore at will, at my fingers, and all I needed was the internet. It was and still remains a passion of mine and some of the scripts I wrote I still use today. I was then, of course, overjoyed when I began computational research in the Summer of my Sophomore year. As a Summer Research Assistant in Computational Biophysics I pursued a deeper understanding of the mechanics taking place within axons. I crafted visualizations that spoke volumes, presenting our endeavors at esteemed conferences and igniting dialogues in the realm of scientific inquiry. It was incredible, it was my first time experiencing the joy of research and that would ignite the flame that burns stronger today. I still work on this project to this day. It has been my longest standing research project. I work with an incredible team to further develop these simulations and extent them to a wide range of questions. I am currently gearing up to present, for the second time, at the Biophysical Society Meeting in February 2024.

My odyssey led me to the University of Wisconsin-Madison on a summer REU program, where the enigmatic world of neutrinos beckoned. Engaging with the IceCube neutrino detector unveiled a new world of physics. Tasked with calculating the energy of a momentous neutrino interaction, I navigated through the labyrinth of high-energy neutrinos, their elusive nature demanding innovative approaches. I coded Monte Carlo simulations to elucidate these cosmic wanderers. In my calculations I found the highest-energy neutrino observed to date. A tremendously exciting result that I am exctied to present at the American Astronomical Society in January 2024. Throughout this research I took every availble oppurtunity to listen to the theorists who wove tales of sterile neutrinos. My passion for particle physics erupted.

Amidst this scientific mosaic, my role as the president of the CWU Astronomy Club emboldens my commitment to STEM dissemination within my community. Steering the club, I orchestrate celestial journeys in our planetarium, weaving tales of the cosmos for eager audiences, infusing lectures with a theatrical allure. The opportunity to share the wonders of space, from the intricacies of celestial bodies to the boundless tapestry of galaxies, fills me with a sense of profound honor and humility. It's akin to conducting symphonies of cosmic marvels, inviting fellow enthusiasts on voyages that traverse the realms of our night sky and extend to the cosmic horizon. In my time as president I have been able to send my members to Oregon to witness the eclipse. While I was unable to join them due to my workload I was nontheless overjoyed to hear about their time, see their pictures, and listen to their stories. I reserve a special joy in that I know I was able to make that happen.

At every step of the way my desire to pursue knowledge is motivated by a desire to spread it. I feel that there is nothing to be gained in knowledge alone; knowing the secrets of the universe is not enough; they must be taught. Teaching, for me, is not a mere occupation but a profound calling, a sanctum where my ardor for physics converges with my zeal for nurturing minds. From my roles as a math and physics tutor to the realm of teaching, each engagement has been transformative. I have had the humbling experience of crafting lesson plans, practice exams, and lectures as a TA both in observational astronomy classes and for introductory physics. While every interaction with my students is special some stand out. I recall a particular regular to my sessions who upon leaving put a sticky note at my desk. It read, “Thank you so much Calvin! Why is it best to teach physics on the edge of a cliff? Because that’s where students have the most potential!.” It was silly, it was short, it was a sticky note. But in that moment it was the most powerful force in the unvierse. To this day that sticky note is in between my name cards of my name tag. Every time I see it I am reminded why I love teaching. I believe that one day I will get to teach students about something as incredible as quantum gravity. But more than anything I want to teach students something that excites them, something that makes them excited to learn.

In this moment, my trajectory aligns with <insert school here>, where theoretical physics, particle physics, and gravity converge. The culmination of my experiences reinforces a resounding truth: a PhD isn’t merely an aspiration but an imperative compass guiding me toward a future where my passion resonates with purpose, delving deeper into the universe—an odyssey empowered by a PhD.
\end{document}
