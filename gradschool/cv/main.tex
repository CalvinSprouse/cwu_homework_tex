% Document class and font size
\documentclass[a4paper,9pt]{extarticle}

% Packages
\usepackage[utf8]{inputenc}
\usepackage{geometry}
\usepackage{titlesec}
\usepackage{enumitem}
\usepackage{hyperref}
\usepackage{tabularx}
\usepackage{fancyhdr}

% configure page with geometry
\geometry{letterpaper, margin=0.75in}

% Formatting
% Removes item separation
\setlist{noitemsep}
% Section title format
\titleformat{\section}{\large\bfseries}{\thesection}{1em}{}[\titlerule]
% Section title spacing
\titlespacing*{\section}{0pt}{\baselineskip}{\baselineskip}

% Begin document
\begin{document}

% Disable page numbers
\pagestyle{fancy}
\renewcommand{\headrulewidth}{0pt}
\fancyhead{}
\fancyhead[L]{\textit{Calvin Sprouse}}
\fancyhead[R]{\textit{December 2023}}
% Remove header from the first page
\thispagestyle{empty}

% Header
\begin{flushleft}
% Name
\textbf{\LARGE Calvin Sprouse}\\[2pt]
Full Time Student and Part Time Tutor in Physics and Math, Central Washington University (CWU), Ellensburg, Washington
% Contact info
\\ \href{mailto:sprousecal@gmail.com}{sprousecal@gmail.com} | \href{tel:4252602368}{(425) 260-2368} | \href{https://www.linkedin.com/in/calvin-sprouse}{www.linkedin.com/in/calvin-sprouse}
% for webpage add
% | \href{https://JohnDoe/en/staff/single/59}{Organization page}
% to above line
\end{flushleft}

% Education Section
\section*{RESEARCH INTERESTS}
\noindent
Theoretical Physics, Quantum Gravity, High Energy Physics, Astroparticle Physics, Particle Physics
% Education Section
\section*{EDUCATION}
\noindent
% University name and location
\textbf{Central Washington University}, Ellensburg, Washington \hfill September 2020 | June 2024\\
% Degree and GPA
B.S. Physics \hfill Cumulative GPA: 3.73/4.00 \\
Minors in Astronomy and Mathematics
% Additional info
% Thesis Title: XXXXXXXXXXXXXXXXXXXXXXXXXXXXXXXX

% \noindent\\
% % University name and location
% \textbf{\#University Name},  \#City, \#Country \hfill \#Month \#Year | \#Month \#Year\\
% % Degree and GPA
% Bachelor of Science: \#Major field
% % Additional info

% Experience Section
\section*{ACADEMIC EXPERIENCE}
\noindent
% Company name and location
\textbf{CWU Tutoring Center} \hfill Ellensburg, Washington\\
% Position and duration
\textit{Lead Tutor} \hfill September 2022 | Present
\begin{itemize}
    \item Work with students by drop-in or appointment on math issues in any STEM course.
    \item Reinforce study habits with a focus on math concepts.
    \item Develop and give trainings to fellow tutors according to CRLA standards.
    \item Preform office tasks associated with running a tutoring center.
    % Job responsibilities and achievements
\end{itemize}

\noindent
% Company name and location
\textbf{CWU Tutoring Center} \hfill Ellensburg, Washington\\
% Position and duration
\textit{Peer Assisted Learning Physics Tutor} \hfill September 2022 | Present
\begin{itemize}
    \item Provide individual and group help to students in any physics course.
    \item Reinforce study habits with a focus on physics concepts.
    \item Create session plans targeted to introductory physics students.
    \item Regularly interact with professors of introductory classes to advertise and discuss upcoming coursework.
    % Job responsibilities and achievements
\end{itemize}

\noindent
% Company name and location
\textbf{CWU Department of Physics} \hfill Ellensburg, Washington\\
% Position and duration
\textit{TA for PHYS 303 Observational Astronomy} \hfill June 2023
\begin{itemize}
    \item Developed astronomy focused lesson plans for a quarter long course.
    \item Taught lessons using a planetarium.
    \item Taught students how to identify constellations and use celestial coordinate systems using only their eyes.
    \item Evaluated students throughout the quarter.
    \item Led observational sessions and mentored students doing research on the CWU research telescope.
    % Job responsibilities and achievements
\end{itemize}

\noindent
% Company name and location
\textbf{CWU Department of Physics} \hfill Ellensburg, Washington\\
% Position and duration
\textit{TA for PHYS 181 Introductory Physics with Lab} \hfill September 2023 | Present
\begin{itemize}
    \item Worked in the classroom to assist with labs.
    \item Graded student assignments and provided feedback.
    \item Provided explanations to students in class on lecture material.
    \item Read and discussed physics teaching literature as part of an accompanying class.
    % Job responsibilities and achievements
\end{itemize}

\noindent
% Company name and location
\textbf{Northwest Earth Space Science Pathways Student Assistant} \hfill Ellensburg, Washington\\
% Position and duration
\textit{Peer Assisted Learning Physics Tutor} \hfill September 2022 | June 2023
\begin{itemize}
    \item Helped create robotics curriculum for K-12 student teams.
    \item Hosted week-long summer camps for students to learn robotics.
    \item Taught K-12 students engineering and programming.
    % Job responsibilities and achievements
\end{itemize}

\noindent
% Company name and location
\textbf{After School Robotics Instructor} \hfill North Bend, Washington\\
% Position and duration
\textit{Peer Assisted Learning Physics Tutor} \hfill 2013 | 2020
\begin{itemize}
    \item Taught K-12 students skills in teamwork, leadership, engineering, programming, and project management.
    \item Mentored competitive FIRST student teams.
    \item Coordinate with other student teachers to provide classes to an entire school district.
    % Job responsibilities and achievements
\end{itemize}

\pagebreak
% Projects Section
\section*{PROJECTS}
\noindent
% Project name and location
\textbf{Computational Models of Neuronal Cytoskeleton} \hfill CWU - NSF\\
\textit{Undergraduate Student Researcher}
% Project link and duration
\hfill June 2022 | Present
\begin{itemize}
    \item Develop computational models on the mechanics of the neuronal Cytoskeleton in MatLab.
    \item Developed data visualizations in Python.
    \item Presented work at 6 research conferences including PhysCon 2022, the Biophysical Society Meeting 2023, and soon 2024.
    \item Made posters and presentations for conferences.
    \item Wrote user guides and kept a lab notebook documenting procedures.
\end{itemize}

\noindent
% Project name and location
\textbf{Construction of a new Spectrograph for the CWU 0.6m Research Telescope} \hfill CWU - NSF\\
\textit{Undergraduate Student Researcher}
% Project link and duration
\hfill September 2022 | June 2023
\begin{itemize}
    \item Programmed data reduction notebooks in Python for an under construction spectrograph.
    \item Researched data analysis methods for using spectrograph data.
\end{itemize}

\noindent
% Project name and location
\textbf{Characterization of CWU 0.6m Telescope Precision} \hfill CWU\\
\textit{Undergraduate Student Researcher}
% Project link and duration
\hfill March 2022 | June 2023
\begin{itemize}
    \item Created and executed methods to reliably measure variable stars.
    \item Wrote Python scripts to automatically control a research telescope.
    \item Wrote Python scripts to preform data analysis on hundreds of observations.
    \item Presented this research at two research conferences.
\end{itemize}

\noindent
% Project name and location
\textbf{Development of a Magneto Caloric Testing Apparatus} \hfill CWU\\
\textit{Undergraduate Student Researcher}
% Project link and duration
\hfill January 2023 | Present
\begin{itemize}
    \item Developed software in Python for an Arduino controlled measurement device.
    \item Research sampling material for device calibration.
\end{itemize}

\noindent
\textbf{Calculating the Energy of IceCube Neutrino 190331A}\\
\textit{University of Wisconsin Madison Astrophysics REU Summer 2023}
\begin{itemize}
    \item Developed Monte Carlo simulations for calculating neutrino energies.
    \item Worked with IceCube data.
    \item Learned to work within IceCube data analysis protocols.
    \item Learned to work with large research teams and follow data handling procedures.
\end{itemize}

\noindent
\textbf{CWU Learning Commons Math Center Data Analysis}\\
\textit{Lead Tutor}
\begin{itemize}
    \item Handled sensitive student data according to FERPA guidelines.
    \item Wrote Python scripts to process student data from a variety of formats.
    \item Used data collected in the math tutoring center to answer questions and improve tutoring.
\end{itemize}

\section*{PUBLICATIONS}
\noindent
\textbf{Biophysical Society Meeting 2023 Poster Abstract}
% Project name and location
\begin{itemize}
    \item Craig et. al. (2023). Self-organization of the microtubule cytoskeleton in developing axons. Biophysical Journal Abstract, \href{doi.org/10.1016/j.bpj.2022.11.1506}{doi.org/10.1016/j.bpj.2022.11.1506}
    \item Sprouse et. al. (2024). The Highest-Energy Astrophysical Muon-Neutrino: 190331A. American Astronomical Society 243rd Meeting.
\end{itemize}

\section*{SELECTED COURSES}
\begin{tabularx}{1\textwidth}{>{\raggedright\arraybackslash}X >{\raggedright\arraybackslash}X }
    % \textbf{Physics Courses}&   \textbf{Math Courses}\\
    \vspace{-0.2in}
    \begin{itemize}
        \item PHYS 292 Exploring Physics Teaching
        \item MATH 314 Probability and Statistics
    \end{itemize}
    & \vspace{-0.2in} \begin{itemize}
        \item MATH 260 Sets and Logic
    \end{itemize}
\end{tabularx}

\vspace{-0.2in}
\section*{AWARDS}
\textbf{Sigma Pi Sigma Department Nomination} \hfill CWU Deptartment of Physics\\
Nominated by the Physics department for academic excellence and volunteer service. \hfill June 2023
% \textbf{College of the Sciences Deans List}  \hfill CWU\\
% Awarded a deans list recognition for maintaining a GPA above 3.5/4.0.\\

% \section*{OTHER EXPERIENCES}
% \noindent
% \textbf{\# you can here experiences those are less related to you applying position}  \hfill \#City, \#Country\\ % Company name and location
% \textit{\# Your Position} \hfill \#Month \#Year | \#Month \#Year % Position and duration
% \begin{itemize}
%     \item \# Description of your job duty and responsibilities.
%     \item \# Description of your job duty and responsibilities.
%     \item \# Description of your job duty and responsibilities.
%     \item \# Description of your job duty and responsibilities.
% \end{itemize}

% \section*{ENGLISH \& GRE TESTS}
% \begin{tabularx}{1\textwidth}{>{\raggedright\arraybackslash}X >{\raggedright\arraybackslash}X }
% \textbf{IELTS (Academic): 7.5} (overall score)&   \textbf{GRE General Test:}\\ \\
% Listening: XX | Reading: XX & Quant: XXX | Verbal: XXX\\
% Speaking: XX   | Writing: XX& Analytical writing: X\\
% Test date: \#Month \#Year &Test date: \#Month \#Year
% \end{tabularx}

% Skills Section
\section*{SKILLS}
\begin{itemize}
    \item \textbf{Computer Science:} Python scripting, simulations, and data science; MatLab; Mathematica; Git; GitHub; \LaTeX
\end{itemize}

% \section*{REFERENCES}
% \textbf{Prof. John Doe}\\
% \textit{\#Position, \# Department, \# Organization name or University name, \# City, \# Country}\\
% E-mail: \href{mailto:johndoe@XXX.XX}{johndoe@XXX.XX}\\
% Scholar Profiles: \href{related link}{University of XXX - Personal Page} | \href{related link}{Google Scholar} | \href{https://www.linkedin.com/in/xxxxxxxx}{LinkedIn}\\ \\
% \textbf{Prof. John Doe}\\
% \textit{\#Position, \# Department, \# Organization name or University name, \# City, \# Country}\\
% E-mail: \href{mailto:johndoe@XXX.XX}{johndoe@XXX.XX}\\
% Scholar Profiles: \href{related link}{University of XXX - Personal Page} | \href{related link}{Google Scholar} | \href{https://www.linkedin.com/in/xxxxxxxx}{LinkedIn}\\ \\
% \textbf{Prof. John Doe}\\
% \textit{\#Position, \# Department, \# Organization name or University name, \# City, \# Country}\\
% E-mail: \href{mailto:johndoe@XXX.XX}{johndoe@XXX.XX}\\
% Scholar Profiles: \href{related link}{University of XXX - Personal Page} | \href{related link}{Google Scholar} | \href{https://www.linkedin.com/in/xxxxxxxx}{LinkedIn}\\ \\
% End document
\end{document}
