% LaTeX Document made from HomeworkTemplate
% Last Updated: 2023 november 08

% document style header
\documentclass[a4paper, 12pt]{../../config/homework}

% import default packages
\usepackage{../../config/defpackages}

% import fun symbol packages (and coffee)
% \usepackage{.config/sympack}

% import custom math commands
\usepackage{../../config/domath}

% end preamble
\begin{document}

% document title
\noindent
\begin{tabularx}{\textwidth}{>{\centering\arraybackslash}X>{\centering\arraybackslash}X>{\centering\arraybackslash}X}
Calvin Sprouse & Book Problems 8 & Due 2023 Dec 01 1600\\
\midrule
\end{tabularx}

% homework problems begin
\section*{\S 2.6 problems.}
% S2.6: 2,4,6,9,11,16(a,b)
\begin{enumerate}
\item[2.] Prove Theorem 1: For all integers, $a, b, c, x$ and $y$, if $a|b$ and $b|c,$ then $a|(bx+cy)$.
\[\forall a, b, c, x, y\in\ints\left(\left(a|b\land a|c\right) \rightarrow a|(bx+cy)\right).\]
\begin{proof}
Let $a, b, c, x,$ and $y$ be given as arbitrary integers. Assume that $a|b$ and $a|c$. By the definition of divides there exist $w,u\in\ints$ such that
\begin{align*}
aw &= b,\\
au &= c.
\end{align*}
Let $k\in\ints$ be defined as
\[k=wx+uy.\]
Then
\begin{align*}
    bx+cy &= awx + auy,
    \\&= a(wx+uy),
    \\&= ak.
\end{align*}
Thus, by definition, $a|(bx+cy)$. Therefore, for $a, b, c, x, y\in\ints$, if $a|b$ and $b|c$, then $a|(bx+cy)$.
\end{proof}

\item[4.] Let $m,n\in\ints, n\neq0$. Prove that if $n^2x^2-2mnx+m^2=n^2$, then $x$ is irrational.
\[\forall m,n\in\ints\left(\left((n\neq0) \land (n^2x^2-2mnx+m^2=n^2)\right) \rightarrow \left(\forall a\in\ints,b\in\ints^+\,x\neq\frac{a}{b}\right)\right).\]
I think I stumped myself by trying to write the symbolic version.

\pagebreak
\item[6.] Use the contrapositive to prove the following statement. For all $x\in\reals^+$ if $x$ is irrational, then $\sqrt{x}$ is irrational. You will need to use the following consequence of the Closure Properties for the Rational Numbers: If $x$ is rational, then $x^2$ is rational.
\[\forall\,x\in\reals^+\left((x\nirat) \rightarrow (\sqrt{x} \nirat)\right).\]
\begin{proof}
Let $x\in\reals^+$ be given. We proceed with a proof by the contrapositive:
\[(\sqrt{x}\irat) \rightarrow (x \irat).\]
Assume $\sqrt{x}$ is rational, that is, there exist $a,b\in\ints$ such that
\[\sqrt{x} = \frac{a}{b}.\]
Then
\begin{align*}
\sqrt{x} &= \frac{a}{b},\\
x &= \frac{a^2}{b^2}.
\end{align*}
By the Closure Properties of the Rational Numbers $a^2/b^2$ is rational, that is, there exist $c,d\in\ints$ such that
\[\frac{a^2}{b^2} = \frac{c}{d}.\]
This is a contradiction, I think (check this later, it just feeeeeels like one).
Thus our assumption must be wrong; $\sqrt{x}$ is irrational. Therefore if $x$ is irrational then $\sqrt{x}$ is irrational.
\end{proof}

\pagebreak
\item[9.] Prove or disprove the following statement. For all $x,y\in\reals$, if $x$ and $y$ are irrational, then $xy$ is irrational.
\[\forall x,y\in\reals\left(\left(\left(x\ \text{is irrational}\right)\land\left(y\ \text{is irrational}\right)\right)\rightarrow\left(xy\ \text{is irrational}\right)\right).\]
\textit{Counterexample.} Suppose $x=\sqrt{2}$ and $y=\sqrt{2}$. As proven in class, $\sqrt{2}$ is irrational. Then
\begin{align*}
xy &= \sqrt{2}\cdot\sqrt{2},\\
&= \sqrt{2\cdot 2},\\
&= \sqrt{4},\\
&= 2.
\end{align*}
Thus for two particular irrational numbers, $x$ and $y$, their product, $xy$ is rational. This disproves the statement
\[\forall x,y\in\reals\left(\left(\left(x\ \text{is irrational}\right)\land\left(y\ \text{is irrational}\right)\right)\rightarrow\left(xy\ \text{is irrational}\right)\right).\]

\vspace{2\singlelineheight}
\item[11.] Prove, for all integers $x$ and $y$, $14x+36y\neq 51.$
\[\forall x,y\in\ints\ 14x+36y\neq51.\]
\begin{proof}
We proceed with a proof by contradiction:
\[\exists\,x,y\in\ints\,14x+36y=51.\]
If there are two integers, $x$ and $y$ such that $14x+36y=51$ then they can be found by re-arrangement. That is
\begin{align*}
14x + 36y &= 51,\\
14x &= 51 - 36y,\\
x &= \frac{51-36y}{14},\\
&= \frac{51}{14} - \frac{36}{14}y.
\end{align*}
This is a contradiction since $x,y\in\ints$ but integers are not closed under division. Thus,
\[\forall x,y\in\ints\ 14x+36y\neq51.\]
\end{proof}

\pagebreak
\item[16]
\begin{enumerate}
\item[a)] Use the contrapositive to prove, for all $x\in\ints$, that if $3|x^2$, then $3|x$. There will be two cases, namely $x\mod3=1$ and $x\mod3=2$.
\begin{proof}
Let \(x\in\ints\) be given. We use the contrapositive, that is, if \(3\ndiv x\) then \(3 \ndiv x^2\). We proceed with a proof by cases:
\begin{enumerate}[label=\textit{(\roman*)}]
\item \(x\mod3=0\),
\item \(x\mod3=1\),
\item \(x\mod3=2\).
\end{enumerate}
\textit{Case (i).} Suppose \(x\mod3=0\). Then \(x=3q+1\) for some \(q\in\ints\). Clearly, \(x|3\) by the same \(q\). Thus we have vacuous truth.
\\\textit{Case (ii).} Suppose \(x\mod3=1\). Then \(x=3q+1\) for some \(q\in\ints\). Clearly, \(x\ndiv 3\). Then
\begin{align*}
x^2 &= (3q+1)^2,\\
&= 9q^2 + 6q + 2,\\
&= 3q(3q+2) + 2.
\end{align*}
That is, $x^2\mod3=2$. Thus \(x^2 \ndiv 3\).
\\\textit{Case (iii).} Suppose \(x\mod3=2\). Then \(x=3q+2\) for some \(q\in\ints\). Clearly \(x\ndiv3\). Then
\begin{align*}
x^2 &= (3q+2)^2,\\
&= 9q^2 + 12q + 4,\\
&= 3(3q^2 + 4q) + 4.
\end{align*}
That is, \(x^2 \mod 3 = 4\). Thus \(x^2 \ndiv 3\).
Therefore, for all \(x\in\ints\) if \(3\ndiv x\) then \(3 \ndiv x^2\). By the contrapositive, for all \(x\in\ints\) if \(3|x\) then \(3|x^2\).
\end{proof}
\item[b)] Use part a) of this exercise to prove that the square root of $3$, $\sqrt{3}$, is irrational.
\end{enumerate}
\end{enumerate}

\pagebreak
\section*{\S 2.7 problems.}
% S2.7: 2,3,4
\begin{enumerate}
\item[2.] Prove $\log_{32}16\nirat$.

\item[3.] Prove $\log_{10}7\nirat$.

\item[4.] Prove $\log_4{5}\nirat$.
\end{enumerate}

\pagebreak
\section*{\S 3.2 problems.}
% S3.2: 2,3,5,9,10(d,k),14,24
\begin{enumerate}
\item[2.] Let $A,B,$ and $C$ be sets, $x$ an object, and $p$ and $q$ statements. For each expression given below, determine whether it makes sense (yes) or it does not make sense (no). If your answer is yes, state whether the expression is a statement or a set. If the answer is no, briefly explain why.
\begin{enumerate}[label=\alph*)]
\item $x\in A$.
\\Yes. This is a statement.
\item $p\in\ints$.
\\No. A statement does not belong to the set of integers.
\item $A\in q$.
\\No. A set does not belong to a statement.
\item $\neg A$.
\\No. There is no meaning to the negation of a set.
\item $x\in A \land B$.
\\No. While $x\in A$ is a statement that makes sense $B$ is not a statement.
\item $x\in A \lor q$.
\\Yes. This is a statement.
\item $(A \cup B) \cap C$.
\\Yes. This is a set.
\item $(A\cup B) \subseteq C$.
\\Yes. This is a statement.
\item $(x\in A)^c$.
\\No. $x\in A$ is a statement and there is no meaning to the complement of a statement.
\item $x\in A^c$.
\\Yes. This is a statement.
\item $\neg (x\in A) \cap B$.
\\No. $\neg (x\in A)$ is a statement and there is no meaning to the intersect of a statement and a set.
\item $\neg (x \in (A \cap B))$.
\\Yes. This is a statement.
\end{enumerate}

\pagebreak
\item[3.] Let $S = \left\{x\in\ints:\exists\,k\in\ints, x=2k-1\right\}$ and $T=\left\{x\in\ints:x\ \text{is odd}\right\}.$ Prove that $S=T$.
\begin{proof}
Let $x\in\ints$ be given. We proceed with a proof by cases
\begin{enumerate}[label=(\roman*).]
\item $x\in S$,
\item $x\in T$.
\end{enumerate}
\textit{Case (i).} Suppose $x\in S$, that is, $x=2k-1$ for some $k\in\ints$.
\\\textit{Case (ii).} Suppose $x\in T$, that is, x is odd. Then, by definition, there exists $q\in\ints$ such that $x=2q + 1$.
\end{proof}

\pagebreak
\item[5.] Let $S=\left\{x\in\ints:x\mod12=8\right\}$ and $T=\left\{x\in\ints:4|x\right\}.$ Prove that $S\subseteq T$, but $T\nsubseteq S.$

\item[9.] Let $S=\left\{x\in\ints:\exists\,r,s\in\ints,x=9r+6s\right\}$ and $T=\left\{x\in\ints:3|x\right\}.$
\begin{enumerate}[label=\alph*)]
\item Prove that $S\subseteq T$.
\item Prove that $T \subseteq S$.
% hint: 9n + 6(-n) = 3n.
\end{enumerate}

\item[10.] Let $A,B,$ and $C$ be subsets of a universal set $\mathcal{U}$. Prove each of the following set theory theorems using a sequence of logically equivalent compound forms.
\begin{enumerate}
\item[d.] $A \cap \left(B \cup C\right) = \left(A \cap B\right)\cup\left(A\cap C\right).$
\item[k.] $A\cap A^c=\emptyset.$
\end{enumerate}

\item[14.] Prove that, if $A\cap B^c=\emptyset$, then $A\subseteq B$.

\item[24.] Find the mistake in the \say{proof} of the following \say{proposition.} Is this \say{proposition} true? If not, find a counterexample.
\\\say{\textbf{Proposition.}} Let $A,B,$ and $C$ be sets and suppose that $A\subseteq (B \cup C)$. Then $A\subseteq B$ or $A\subseteq C$.
\\\say{\textit{proof.}} Let $x$ be any object and suppose that $x\in A$. Then $x\in (B\cup C)$ since $A\subseteq (B \cup C).$ Thus, by the definition of union, $x\in B$ or $x\in C$. Therefore, for all objects $x$, if $x\in A$, then $x\in B$ or, for all objects $x$, if $x\in A$, then $x\in C.$
\\Hence, by the definition of subset, $A\subseteq B$ or $A\subseteq C$.
\end{enumerate}
\end{document}
