% document style header
\documentclass[a4paper, 12pt]{config/homework}

% import default packages
\usepackage{config/defpackages}

% import custom math commands
\usepackage{config/domath}

% end preamble
\begin{document}

% document title
\noindent
\begin{tabularx}{\textwidth}{>{\centering\arraybackslash}X>{\centering\arraybackslash}X>{\centering\arraybackslash}X}
Calvin Sprouse & PHYS 451 HW 7 & Due 2023 Dec 03\\
\midrule
\end{tabularx}

% homework problems begin
\begin{enumerate}
\item Consider the exploded spacecraft from one of the best episodes of television ever produced. Initially, the spacecraft was traveling at 5 m/s to the east at a height of 1.96 km. When the spacecraft was 100 m west of the center of town it experienced a rapid unplanned dissasembly (RUD). Suppose none of the pieces acuire appreciable vertical velocities after the RUD. After the dust settles the following three pieces are located:
\begin{table}[h]
    \centering
    \begin{tabular}{cccc}
    Piece & Mass [kg] & Distance from town center [km] & Angle from north [degrees] \\ \midrule
    1 & \(300 \unit{kg}\) & \(6 \unit{\kilo\meter}\) & \(0 \unit{\degree}\) \\
    2 & \(1000 \unit{kg}\) & \(1.5 \unit{\kilo\meter}\) & \(126 \unit{\degree}\) \\
    3 & \(400 \unit{kg}\) & \(4 \unit{\kilo\meter}\) & \(205 \unit{\degree}\)
    \end{tabular}
    \end{table}
\begin{enumerate}
\item Compare the linear momentum of each piece after the explosion to the craft prior to the explosion.
\item Determine the center of mass of the pieces after the explosion, and compare it to where it \textit{should} be.
\end{enumerate}

\end{enumerate}
\end{document}
