% document style header
\documentclass[a4paper, 12pt]{config/homework}

% import default packages
\usepackage{config/defpackages}
% import custom math commands
\usepackage{config/domath}
\usepackage{geometry}

% end preamble
\begin{document}

% document title
\noindent
\begin{tabularx}{\textwidth}{>{\centering\arraybackslash}X>{\centering\arraybackslash}X>{\centering\arraybackslash}X}
Calvin Sprouse & Topology Problem Set 2 & 2024 April 09\\
\midrule
\end{tabularx}

% homework problems begin
% 1a, 2a, 7, 5(i), 5(iii)
\vspace{\baselineskip}
\begin{enumerate}
\item In the context of exercise 5 (iii) on page 31:
\begin{enumerate}[label=\alph*.]
\item Give two examples of open sets that are not \(\reals\) or \(\emptyset\). Use at least one complete sentence to explain why the given sets are open.

\(\tau_3\) consists of \(\reals\), \(\emptyset\), and every interval \([n,\infty)\), for \(n\in\reals^+\).

The sets \([1,\infty)\) and \([2,\infty)\) are open to \(\tau\_3\). Either sets compliment is an interval from \(-\infty\) to the start-point exclusive. Since the compliment of the sets is not in \(\tau_3\), the sets are open to \(\tau\).

\item Give two examples of closed sets that are not \(\reals\) or \(\emptyset\). Use at least one complete sentence to explain why the given sets are closed.

The sets \((-\infty,1)\) and \((-\infty,2)\) are closed to \(\tau_3\). These are, in fact, the compliments to the sets defined above. Since the compliment of either set is in \(\tau_3\), these sets are closed to \(\tau_3\).

\end{enumerate}
\item In the context of exercise 6 (ii) on page 31:
\begin{enumerate}[label=\alph*.]
\item Give two examples of open sets that are not \(\mathbb{N}\) or \(\emptyset\). Use at least one complete sentence to explain why the given sets are open.

\(\tau_2\) consists of \(\mathbb{N}\), \(\emptyset\), and every set \(\{n, n+1, \dots\}\), for \(n\in\ints^+\). This is called the final segment topology.

The sets \(\{2, 3, 4, \dots\}\) and \(\{3, 4, 5, \dots\}\) are in \(\tau_2\) and are thus open sets.

\item Give two examples of closed sets that are not \(\mathbb{N}\) or \(\emptyset\). Use at least one complete sentence to explain why the given sets are closed.

The sets \(\{1\}\) and \(\{1, 2\}\) are closed sets to \(\tau_2\). These sets are the compliments to the sets define above over the positive integers, \(\mathbb{N}\), and are thus closed sets to \(\tau_2\).

\end{enumerate}
\pagebreak
\item Exercise 1.2: \#2 (page 36). Let \((X,\tau)\) be a topological space with the property that every subset is closed. Prove that this is a discrete space.

\begin{proof}

\end{proof}

\end{enumerate}
\end{document}
