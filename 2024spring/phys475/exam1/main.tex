% document style header
\documentclass[a4paper, 12pt]{config/homework}

% import default packages
\usepackage{config/packages}
\usepackage{config/commands}

% end preamble
\begin{document}

% document title
\noindent
\hfill Calvin Sprouse \hfill PHYS 475 Midterm Exam \#1 \hfill 2024 April 25 \hfill
\bigskip

% homework problems begin
\bigskip\noindent
1.\ A non-relativistic particle with mass \(m\) moves in a three-dimensional potential, \(V(r)\), which is spherically-symmetric and vanishes as \(r\to\infty\). At a certain time, this particle is found in the state
\[\psi(r,\theta,\phi) = Cr^{\sqrt{3}}\exp\left[-\alpha r\right]\cos\left[\theta\right],\]
where \(C\) and \(\alpha\) are constants. We have ignored spin.
\begin{enumerate}[label=(\alph*)]
\item What is the orbital angular momentum of this state; that is, what are the quantum numbers \(l\) and \(m_l\)?
\item What is the energy, \(E\), of this state? The radial equation, \(u=rR\), may be helpful here. Recall, \(V(r)\to0\) as \(r\to\infty\).
\item Now that the energy, \(E\), is known from part (b), what is the potential, \(V(r)\)?
\end{enumerate}
\bigskip
\begin{enumerate}[label=(\alph*)]
\item Since the potential, \(V(r)\), is spherically-symmetric, the wavefunction, \(\psi\), may be separated into a radial and angular function:
\[\psi(r,\theta,\phi)=R_{n,\ell}(r)Y_\ell^{m_\ell}(\theta,\phi).\]
Let \(C=C_R C_Y\), where \(C_R\) is a constant associated with the radial equation and \(C_Y\) is a constant associated with the angular equation. The information about orbital angular momentum will come from the radial part of the equation, which will be a spherical harmonic of the form
\[Y_\ell^{m_\ell} = C_Y \cos\left(\theta\right).\]
Griffiths Table 4.3 lists normalized spherical harmonics for particular values of \(\ell\) and \(m_\ell\). The only spherical harmonic of a this form is given by \(\ell=1\) and \(m_\ell=0\); that is,
\[Y_1^0=\sqrt{\frac{3}{4\pi}} \cos\left(\theta\right).\]
Thus, \(\ell=1\) and \(m_\ell=0\).

\pagebreak
\item By Griffiths Eq.\ 4.37,
\[-\frac{\hbar^2}{2m}\diff[2]{u}{r}+\left(V+\frac{\hbar^2}{2m}\frac{\ell\left(\ell+1\right)}{r^2}\right)u=Eu,\]
where \(u=rR\). Substituting \(\ell=1\) from part (a) yields
\[-\frac{\hbar^2}{2m}\diff[2]{u}{r}+\left(V+\frac{\hbar^2}{mr^2}\right)u=Eu.\]
See below for the evaluation of the second derivative of \(u\) with respect to \(r\). The solution is given in terms of \(u\). Then, multiplication of both sides by the inverse of \(u\) results in
\[E = -\frac{\hbar^2}{2m}\left(\alpha^2 - 2\alpha r^{-1} \left(1+\sqrt{3}\right) + r^{-2}\left(3+\sqrt{3}\right)\right) + V + \frac{\hbar^2}{mr^2}.\]
Since the energy, \(E\), is constant we may take any \(r\) to evaluate \(E\). Let \(r\to\infty\). Then, \(V\to0\). Therefore,
\[E = -\frac{\hbar^2 \alpha^2}{2m}.\]


\bigskip
\item To find the potential, we rearrange the expression from part (b) before letting \(r\to\infty\):
\[V = E - \frac{\hbar^2}{mr^2} + \frac{\hbar^2}{2m}\left(\alpha^2 - 2\alpha r^{-1}\left(1+\sqrt{3}\right) r^{-2}\left(3+\sqrt{3}\right)\right).\]
Then, we substitute \(E\) from part (b):
\[V = \frac{\hbar^2}{2m}\left(-\alpha^2 - 2r^{-2} + r^{-2}\left(3 + \sqrt{3}\right) - 2\alpha r^{-1}\left(1 + \sqrt{3}\right) + \alpha^2\right).\]
Simplifying yields the potential energy, \(V\), as a function of \(r\):
\[V = \frac{\hbar^2}{2m}\left(1+\sqrt{3}\right)\left(r^{-2}-2\alpha r^{-1}\right).\]

\end{enumerate}
\pagebreak\noindent
The aforementioned derivative, where \(u=rR\), given in terms of \(u\):
\begin{align*}
\diff[2]{u}{r} &= \diff{}{r}\diff{}{r}\left[C_R rr^{\sqrt{3}}e^{-\alpha r}\right]
\\&= C_R \diff{}{r}\diff{}{r}\left[r^{\sqrt{3}+1}e^{-\alpha r}\right]
\\&= C_R \diff{}{r}\left[r^{\sqrt{3}+1}\diff{}{r}\left[e^{-\alpha r}\right] + e^{-\alpha r}\diff{}{r}\left[r^{\sqrt{3}+1}\right]\right]
\\&= C_R \diff{}{r}\left[rr^{\sqrt{3}(-\alpha)}e^{-\alpha r} + e^{-\alpha r}\left(\sqrt{3}+1\right)r^{\sqrt{3}}\right]
\\&= \diff{}{r}\left[\left(\sqrt{3}+1-\alpha r\right)R\right]
\\&= \left(\sqrt{3}+1-\alpha r\right)\diff{R}{r} + R\diff{}{r}\left[\sqrt{3}+1-\alpha r\right]
\\&= \left(\sqrt{3}+1-\alpha r\right)\diff{R}{r} + (-\alpha)R
\\&= \left(\sqrt{3}+1-\alpha r\right)C_R \left(r^{\sqrt{3}}\diff{}{r}\left[e^{-\alpha r}\right] + e^{-\alpha r}\diff{}{r}\left[r^{\sqrt{3}}\right]\right) -\alpha R
\\&= \left(\sqrt{3}+1-\alpha r\right)C_R \left(r^{\sqrt{3}}(-\alpha)e^{-\alpha r} + e^{-\alpha r}\sqrt{3}r^{\sqrt{3}-1}\right)-\alpha R
\\&= \left( \left(\sqrt{3}+1-\alpha r\right)\left(-\alpha + r^{-1}\sqrt{3}\right) -\alpha\right)R
\\&= \left(r\alpha^2 -2\alpha\left(1 + \sqrt{3}\right) + r^{-1}\left(3 + \sqrt{3}\right)\right)R
\\&= \left(\alpha^2 - 2\alpha r^{-1}\left(1+\sqrt{3}\right) + r^{-2}\left(3+\sqrt{3}\right)\right)u.
\end{align*}
\end{document}
