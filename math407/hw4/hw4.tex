% document style header
\documentclass[a4paper, 12pt]{config/homework}

% import default packages
\usepackage{config/defpackages}
% import custom math commands
\usepackage{config/domath}

% end preamble
\begin{document}

% document title
\noindent
\begin{tabularx}{\textwidth}{>{\centering\arraybackslash}X>{\centering\arraybackslash}X>{\centering\arraybackslash}X}
Calvin Sprouse & MATH407 HW4 & 2024 January 29\\
\midrule
\end{tabularx}

% homework problems begin
\begin{enumerate}
\item A decision maker is described by the utility function \(u(w)=w^{1/3}\). She is given the choice between two random amounts \(X_1\) and \(X_2\), in exchange for her entire present wealth \(w_0\). Suppose that
\[X_1 = \begin{cases}
\phantom{0}8 & \text{with probability}\ 0.5 \\
27 & \text{with probability}\ 0.5
\end{cases}\] and
\[X_2 = \begin{cases}
\phantom{0}1 & \text{with probability}\ 0.6 \\
64 & \text{with probability}\ 0.4
\end{cases}\]
\begin{enumerate}[label=(\alph*)]
\item Show that she prefers \(X_1\) to \(X_2\).


\item Determine for what values of \(w_0\) she should decline the offer.


\item Give an example of a utility function in which she would prefer \(X_2\) to \(X_1\).


\end{enumerate}
\item Recall that the iso-elastic property says that for any \(k > 0\), \(u(kw) = f(k)u(w) + g(k)\) for some \(f(k)\) and \(g(k)\).
\begin{enumerate}[label=(\alph*)]
\item Identify the functions \(f(k)\) and \(g(k)\) in the case of \(u(w)=\ln(w)\).


\item Identify the functions \(f(k)\) and \(g(k)\) in the case of \(u(w)=\frac{w^\lambda - 1}{\lambda}\).


\end{enumerate}
\item Recall that the Arrow-Pratt absolute risk aversion function is given by
\[A(w) = -\frac{\diff[2]{u(w)}{w}}{\diff{u(w)}{w}}.\]
\begin{enumerate}[label=(\alph*)]
\item Compute \(A(w)\) in the case of \(u(w)=\ln(w)\). Is \(A(w)\) non-increasing?
\begin{align*}
A(w) &= -\frac{\diff[2]{u(w)}{w}}{\diff{u(w)}{w}}
\\&= -\frac{\diff[2]{\ln(w)}{w}}{\diff{\ln(w)}{w}}
\\&=
\end{align*}

\item Compute \(A(w)\) in the case of \(u(w)=\frac{w^{\lambda} - 1}{\lambda}\). Is \(A(w)\) non-increasing.


\end{enumerate}
\end{enumerate}
\end{document}
