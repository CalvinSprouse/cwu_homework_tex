% document style header
\documentclass[a4paper, 12pt]{config/homework}

% import default packages
\usepackage{config/defpackages}

% import custom math commands
\usepackage{config/domath}

% end preamble
\begin{document}

% document title
\noindent
\begin{tabularx}{\textwidth}{>{\centering\arraybackslash}X>{\centering\arraybackslash}X>{\centering\arraybackslash}X}
Calvin Sprouse & MATH407 HW2 & Due 2024 Jan 19\\
\midrule
\end{tabularx}

% homework problems begin
\begin{enumerate}
\item Show that the converse of p4 is not true. That is, provide a counterexample of an asymmetric binary relation \(R\) with elements \(x\) and \(y\) such that \(\neg xRy\) does not imply \(yRx\).

The binary relation \(R\) is asymmetric if \(xRy \Rightarrow yRx\) for all \(x,y\).

\textit{Counterexample.} Let \(R\) be the binary relation ``is married to''. Suppose \(\neg xRy\); that is, \(x\) is not married to \(y\). It is well known that this does not imply \(y\) is married to \(x\). Thus, \(\neg xRy\) does not imply \(yRx\).

\item Consider the set of all triples where each component is a real number; that is, \(\reals^3\). Let \(x=(x_1,x_2,x_3)\) and \(y=(y_1,y_2,y_3)\) and define the weak preference \(x \preccurlyeq^* y\) if \(x_i \le y_i\) for at least two out of the three components, \(i=1,2,3\).
\begin{enumerate}[label= (\alph*)]
\item Show \(\preccurlyeq^*\) is connected.

The binary relation \(R\) is connected if \(xRy\) or \(yRx\) for all \(x,y\).
\begin{proof}
Let the binary relation \(\preccurlyeq^*\) be defined for \(x,y\in \reals^3\) such that \(x \preccurlyeq^* y\) if \(x_i \le y_i\) for at least two out of the three components \(i = 1, 2, 3\). The binary relation \(\le \) on the real numbers is connected. Thus, \(x_i \le y_i\) or \(y_i \le x_i\) for \(i= 1, 2, 3\). We now proceed with two cases:
\begin{enumerate}[label= \roman*.]
\item Suppose \(x \preccurlyeq^* y\). Clearly, \(x \preccurlyeq^* y\) or \(y \preccurlyeq^* x\) is true.
\item Suppose \(\neg (x \preccurlyeq^* y)\). That is, \(x_i \le y_i\) is not true for two out of the three components. Since \(\le \) is connected, the components for which \(x_i \le y_i\) is not true instead satisfy \(y_i \le x_i\). Thus, for two out of the three components \(y_i \le x_i\). Therefore, \(y \preccurlyeq^* x\).
\end{enumerate}
The binary relation \(\preccurlyeq^*\) is connected.
\end{proof}

\item Show \(\preccurlyeq^*\) is not transitive by providing a counterexample.

The binary relation \(R\) is transitive if \(( xRy \land yRz) \Rightarrow xRz\) for all \(x,y,z\).

\textit{Counterexample.} Let \(x=(1,2,3)\), \(y=(2,2,1)\), and \(z=(2,0,2)\). It is clear that \( x \preccurlyeq^* y\) and \( y \preccurlyeq^* z\). Furthermore, it is clear that \(\neg x \preccurlyeq^* z\). Thus, \(\preccurlyeq^*\) is not transitive.

\pagebreak
\item Define the strict preference \(x \prec^* y\) by \(x \preccurlyeq^* y\) but not \(y \preccurlyeq^* x\).
\begin{enumerate}[label= \roman*.]
\item Explain why it is equivalent to say \(x \prec^* y\) if \(x_i < y_i\) for at least two out of the three components, \(i=1,2,3\).

If two of the three components satisfy \(x_i \le y_i\) then those same two may satisfy \(y_i \le x_i\) if the two components are equal. The strict preference \(\prec^*\) is stronger than \(\preccurlyeq^*\) in the same way that \(<\) is stronger than \(\le \). Thus, if we say \(x \prec^* y\) then \(x \preccurlyeq^* y\) is implied.

\item Prove or give a counterexample. \(\prec^*\) is asymmetric.

The binary relation \(R\) is asymmetric if \(xRy \Rightarrow \neg yRx\) for all \(x,y\).

\begin{proof}
Let \(\prec^*\) be the binary relation defined as \(x \preccurlyeq^* y \land \neg y \preccurlyeq^* x\). We proceed with a proof by contradiction. Suppose \(x \prec^* y\); that is, \(x_i < y_i\) for at least two \(i\) where \(i = 1,2,3\). Further suppose \(y \prec^* x\); that is, \(y_i < x_i\) for at least two \(i\). Thus two elements of \(x_i < y_i\) and two elements of \(y_i < x_i\). Since there are three elements in \(x\) and \(y\) each there must be at least one element pair, \(x_j\) and \(y_j\), that satisfies \(x_j < y_j\) and \(y_j < x_j\). This is, clearly, a contradiction. There are no real numbers \(x_j\) and \(y_j\) that satisfy this condition. Thus, \(\neg y \prec^* x\). Therefore, \(x \prec^* y \Rightarrow \neg y \prec^* x\). That is, \(\prec^*\) is asymmetric.
\end{proof}

\item Prove or give a counterexample. \(\prec^*\) is negatively transitive.

The binary relation \(R\) is negatively transitive if \(\neg xRy \land \neg yRz \Rightarrow \neg xRz\) for all \(x,y\).

\begin{proof}
\(\neg x \prec^* y\) means \(x_i \ge y_i\) for at least two \(i\). This is equivalently \(y \preccurlyeq^* x\).
Similarly, \(\neg y \prec^* z\) means \(y_i \ge z_i\) for at least two \(i\). This is equivalently \(z \preccurlyeq^* y\).
Thus negative transitivity of the strict preference can be expressed as transitivity of the weak preference. As proven in 2b, \(\preccurlyeq^*\) is not transitive. Thus, \(\prec^*\) is not negatively transitive.
\end{proof}

\end{enumerate}
\end{enumerate}
\end{enumerate}
\end{document}
