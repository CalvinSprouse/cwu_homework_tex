% document style header
\documentclass[a4paper, 12pt]{config/homework}

% import default packages
\usepackage{config/defpackages}

% import custom math commands
\usepackage{config/domath}

% end preamble
\begin{document}

% document title
\noindent
\begin{tabularx}{\textwidth}{>{\centering\arraybackslash}X>{\centering\arraybackslash}X>{\centering\arraybackslash}X}
Calvin Sprouse & MATH407 HW2 & Due 2024 Jan 19\\
\midrule
\end{tabularx}

% homework problems begin
\begin{enumerate}
\item Show that the converse of p4 is not true. That is, prove a counterexample of an asymmetric binary relation \(R\) with elements \(x\) and \(y\) such that \(\neg xRy\) does not imply \(yRx\).

\item Consider the set of all triples where each component is a real number, \(\reals^3\). Let \(x=(x_1,x_2,x_3)\) and \(y=(y_1,y_2,y_3)\) and define the weak preference \(x \preccurlyeq^* y\) if \(x_i \le y_i\) for at least two out of the three components, \(i=1,2,3\).
\begin{enumerate}[label= (\alph*)]
\item Show \(\preccurlyeq^*\) is connected.

\item Show \(\preccurlyeq^*\) is not transitive by providing a counterexample.

\item Define the strict preference \(x \prec^* y\) by \(x \preccurlyeq^* y\) but not \(y \preccurlyeq^* x\).
\begin{enumerate}[label= \roman*.]
\item Explain why it is equivalent to say \(x \prec^* y\) if \(x_i < y_i\) for at least two out of the three components, \(i=1,2,3\).

\item Prove or give a counterexample: \(\prec^*\) is symmetric.

\item Prove or give a counterexample: \(\prec^*\) is negatively transitive.

\end{enumerate}
\end{enumerate}
\end{enumerate}
\end{document}
