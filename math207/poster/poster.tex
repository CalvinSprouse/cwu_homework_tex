\documentclass{beamer}

% import packages
\usepackage[T1]{fontenc}
\usepackage{lmodern}
\usepackage[a0paper]{beamerposter}
\usepackage{graphicx}
\usepackage{booktabs}
\usepackage{tikz}
\usepackage{pgfplots}
\usepackage{anyfontsize}
\usepackage{siunitx}
\usepackage{multicol}
\usepackage{subcaption}

% configure pgfplots
\pgfplotsset{compat=1.14}

% specify theme and color theme
\usetheme{gemini}
\usecolortheme{earth}

% define poster lengths for columns
% If you have N columns, choose \sepwidth and \colwidth such that
% (N+1)*\sepwidth + N*\colwidth = \paperwidth
\newlength{\sepwidth}
\newlength{\colwidth}
\setlength{\sepwidth}{0.025\paperwidth}
\setlength{\colwidth}{0.3\paperwidth}
% define the separator column command
\newcommand{\separatorcolumn}{\begin{column}{\sepwidth}\end{column}}

% configure the author block
\title{The Knitted Torus}
\author{Calvin Sprouse \and Nicholas Klein}
\institute[CWU]{Department of Mathematics, Central Washington University}

% configure the footer
\footercontent{
MATH 207 Honors Seminar \hfill
2024 May 31 \hfill
\href{mailto:calvin.sprouse@cwu.edu}{calvin.sprouse@cwu.edu}
}

% setup logo
\logoright{\includegraphics[height=7cm]{figures/logos/cwu_logo.png}}


%-----------------------------------------------------------------------------%
% begin poster
\begin{document}

% begin the content framework
\begin{frame}[t]
\begin{columns}[t]
\separatorcolumn%


%-----------------------------------------------------------------------------%
% first column of content
\begin{column}{\colwidth}

%------------------------------------------------
% introduction/abstract block
\begin{block}{Introduction}

A torus is a doughnut. A torus can be knitted. A knitted torus is a doughnut made of yarn. The knitted torus is a mathematical object that can be studied in the context of topology. The knitted torus is a torus that has been knitted. The knitted torus is a torus that has been knitted in a particular way. The knitted torus is a torus that has been knitted in a particular way that is interesting to study. The knitted torus is a torus that has been knitted in a particular way that is interesting to study in the context of topology.

\end{block}

\begin{block}{Background}

What is a torus? A torus is a doughnut. A torus is a doughnut that has been mathematically defined. A torus is a doughnut that has been mathematically defined as a surface of revolution. A torus is a doughnut that has been mathematically defined as a surface of revolution about an axis. A torus is a doughnut that has been mathematically defined as a surface of revolution about an axis that is not the axis of revolution. A torus is a doughnut that has been mathematically defined as a surface of revolution about an axis that is not the axis of revolution that is interesting to study.

\end{block}

\end{column}
\separatorcolumn%


%-----------------------------------------------------------------------------%
% second column of content
\begin{column}{\colwidth}

%------------------------------------------------
% content block
\begin{block}{Model}

A torus can be modeled as a surface of revolution. A torus can be modeled as a surface of revolution about an axis. A torus can be modeled as a surface of revolution about an axis that is not the axis of revolution. A torus can be modeled as a surface of revolution about an axis that is not the axis of revolution that is interesting to study. A torus can be modeled as a surface of revolution about an axis that is not the axis of revolution that is interesting to study in the context of topology.

\end{block}

%------------------------------------------------
% content block
\begin{block}{Results}

A torus can be knitted. A torus can be knitted in a particular way. A torus can be knitted in a particular way that is interesting to study. A torus can be knitted in a particular way that is interesting to study in the context of topology. A torus can be knitted in a particular way that is interesting to study in the context of topology in the context of topology.

\end{block}

\end{column}
\separatorcolumn%


%-----------------------------------------------------------------------------%
% third column of content
\begin{column}{\colwidth}

%------------------------------------------------
% content block
\begin{block}{Results}

Results are pending.

\end{block}

%------------------------------------------------
% conclusion block
\begin{block}{Conclusion}

Conclusion is pending.

\end{block}

%------------------------------------------------
% reference block
\begin{block}{References}
\begin{multicols}{2}
\fontsize{16pt}{12pt}\selectfont
\nocite*
\bibliographystyle{plain}
\bibliography{poster_refs.bib}
\end{multicols}
\end{block}

\end{column}
\separatorcolumn%

\end{columns}
\end{frame}
\end{document}
