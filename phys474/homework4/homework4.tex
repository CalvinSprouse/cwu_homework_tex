% document style header
\documentclass[a4paper, 12pt]{config/homework}

% import default packages
\usepackage{config/defpackages}
% import custom math commands
\usepackage{config/domath}

% end preamble
\begin{document}

% document title
\noindent
\begin{tabularx}{\textwidth}{>{\centering\arraybackslash}X>{\centering\arraybackslash}X>{\centering\arraybackslash}X}
Calvin Sprouse & PHYS474 HW4 & 2024 January 24\\
\midrule
\end{tabularx}

% homework problems begin
\begin{enumerate}
\item A particle is trapped in a harmonic oscillator potential. We know that at \(t=0\), the particle can be represented by the wavefunction
\[\Psi(x,0) = A\left(2\psi_0(x) + 5\psi_2(x)\right),\]
where \(\psi_0\) and \(\psi_2\) are the stationary-state solutions for \(n=0\) and \(n=0\), respectively.
\begin{enumerate}[label=(\alph*)]
\item Normalize \(\Psi(x,0)\).
\[1 = \braket*{\Psi(x,0)}{\Psi{x,0}}
= A^2 \left( 4 \braket*{\psi_0}{\psi_0} + 25 \braket*{\psi_2}{\psi_2} + 20\braket*{\psi_0}{\psi_2} \right)
= 29 A^2.\]
% \begin{align*}
% 1 &= \braket*{\Psi(x,0)}{\Psi{x,0}}
% \\&= A^2 \left( 4 \braket*{\psi_0}{\psi_0} + 25 \braket*{\psi_2}{\psi_2} + 20\braket*{\psi_0}{\psi_2} \right)
% \\&= 29 A^2.
% \end{align*}
Thus,
\[A = \sqrt{\frac{1}{29}}\]

\item Construct \(\Psi(x,t)\) and then determine \(\left|\Psi(x,t)\right|^2\). Will \(\expval{x}\) depend on time?

The complete wavefunction is given by
\[\Psi(x,t) = \sqrt{\frac{1}{29}} \left( 2 \Psi_0 + 5 \Psi_2 \right),\]
where \(\Psi_n\) represents \(\psi_n(x)\phi_n(t)\). Thus,
\begin{align*}
\left| \Psi(x,t) \right|^2 &= \frac{1}{29}\left(2\Psi_0^* + 5\Psi_2^*\right)\left(2\Psi_0 + 5\Psi_2\right)
\\&= \frac{1}{29}\left(4\left|\Psi_0\right|^2 + 2\Psi_0^*\Psi_2 + 5\Psi_2^*\Psi_0 + 25\left|\Psi_2\right|^2\right)
\\&= \frac{1}{29}\left(4\left|\psi_0\right|^2
+ 20\psi_0\psi_2\left(e^{-2i\omega t} + e^{2i\omega t}\right)
+ 25\left|\psi_2\right|^2\right)
\\&= \frac{1}{29}\left(
4\left|\psi_0\right|^2
+ 10 \psi_0 \psi_2 \cosh(2i\omega t)
+ 25 \left|\psi_2\right|^2
\right)
\\&= \frac{1}{29}\left(
4\left|\psi_0\right|^2
+ 10 \psi_0 \psi_2 \cos(2\omega t)
+ 25 \left|\psi_2\right|^2
\right)
% \\&= \frac{1}{29}\left(
% 4\left|\psi_0\right|^2
% + \frac{10}{\sqrt{2}}\psi_0\widehat{a_+}^2\psi_0\cos(2\omega t)
% + 25 \left|\psi_0\right|^2
% \right)
\end{align*}
\(\expval{x}\) will depend on time!
\end{enumerate}

\pagebreak
\item Consider the stationary states of the harmonic oscillator. As straightforwardly as possible, compute the following quantities for the \(n\)th stationary state \(\psi_n(x)\).
\begin{enumerate}[label=(\alph*)]
\item \(\expval{x}\).

Given
\[\widehat{x} = \sqrt{\frac{\hbar}{2m\omega}}\left(\widehat{a_+} + \widehat{a_-}\right).\]
Then,
\[\expval{x}
= \bra*{n}\widehat{x}\ket*{n}
= \sqrt{\frac{\hbar}{2m\omega}}\left(\bra*{n}\widehat{a_+}\ket*{n} + \bra*{n}\widehat{a_-}\ket*{n}\right)
= 0.\]

\item \(\expval{x^2}\).

Given
\begin{align*}
\widehat{x^2} &= \widehat{x}^{\,2}
\\&= \frac{\hbar}{2m\omega} \left(\widehat{a_+}^{\,2} + \widehat{a_-}^{\,2} + \widehat{a_-}\widehat{a_+} + \widehat{a_+}\widehat{a_-}\right)
\\&= \frac{\hbar}{2m\omega} \left(\widehat{a_+}^{\,2} + \widehat{a_-}^{\,2} + \frac{1}{\hbar \omega}\left(\widehat{H}+\frac{1}{2}\right) + \frac{1}{\hbar \omega}\left(\widehat{H}-\frac{1}{2}\right)\right)
\\&= \frac{\hbar}{2m\omega} \left(\widehat{a_+}^{\,2} + \widehat{a_-}^{\,2} + \frac{2}{\hbar \omega}\widehat{H}\right).
\end{align*}

Then,
\[\expval{x^2} = \bra*{n}\widehat{x^2}\ket*{n} = \frac{\hbar}{2m\omega} \left( \bra*{n}\widehat{a_+}^{\,2}\ket*{n} + \bra*{n}\widehat{a_-}^{\,2}\ket*{n} + \frac{2}{\hbar \omega}\bra*{n}\widehat{H}\ket*{n}\right) = \frac{E_n}{m\omega^2}.\]

\item \(\expval{p}\).
\[\expval{p} = m\diff{}{t}\expval{x} = m\diff{}{t}(0) = 0.\]

\item \(\expval{p^2}\)

Given
\[\expval{H} = \frac{\expval{p^2}}{2m} = E_n.\]

Then,
\[\expval{p^2} = 2m\expval{H} = 2m E_n.\]

\item \(\expval{T}\)

Given
\[\expval{H} = \expval{T} + \expval{V} = \expval{T} + \frac{1}{2}m\omega^2\expval{x^2}.\]
Then,
\[\expval{T} = \expval{H} - \expval{V} = E_n - \frac{1}{2}m\omega^2\frac{E_n}{m\omega^2} = \frac{1}{2}E_n.\]

\item Is the Heisenberg uncertainty principle satisfied for all values of \(n\)?

The uncertainty of \(x\), \(\sigma_x\), is given by
\[\sigma_x = \sqrt{\expval{x^2} - \expval{x}^2} = \sqrt{\frac{E_n}{m\omega^2}}.\]
The uncertainty of \(p\), \(\sigma_p\), is given by
\[\sigma_p = \sqrt{\expval{p^2} - \expval{p}^2} = \sqrt{2mE_n}.\]
The product of uncertainties is given by
\[\sigma_x\sigma_p = \sqrt{\frac{E_n}{m\omega^2}}\sqrt{2mE_n} = \frac{E_n}{\omega}\sqrt{2} = \hbar\left(n+\frac{1}{2}\right)\sqrt{2}.\]
The Heisenberg uncertainty principle requires that
\[\sigma_x\sigma_p \ge \frac{\hbar}{2}.\]
Where a factor of \(\hbar/2\) appears on both sides thus simplifying to
\[\left(2n + 1\right)\sqrt{2} \ge 1.\]
The left side increases with \(n\) so we check only the smallest \(n\), that is \(n=0\).
\[\sqrt{2} \ge 1,\]
is indeed true. Whence, the uncertainty principle is satisfied for all \(n\).

\end{enumerate}

\pagebreak
\item A particle in a harmonic oscillator potential is described by the normalized wavefunction
\[\ket*{\Psi(x,0)} = \frac{1}{\sqrt{5}}\ket*{1} + \frac{2}{\sqrt{5}}\ket*{2},\]
where \(\ket*{n}\) represents the \(n\)th stationary state.
\begin{enumerate}[label=(\alph*)]
\item What is \(\ket*{\Psi(x,t)}\)?
\item What is the expectation value for energy?
\item What is \(\langle x(t) \rangle\)?
\end{enumerate}

\end{enumerate}
\end{document}
