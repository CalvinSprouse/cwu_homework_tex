% document style header
\documentclass[a4paper, 12pt]{config/homework}

% import default packages
\usepackage{config/defpackages}
% import custom math commands
\usepackage{config/domath}

% end preamble
\begin{document}

% document title
\noindent
\begin{tabularx}{\textwidth}{>{\centering\arraybackslash}X>{\centering\arraybackslash}X>{\centering\arraybackslash}X}
Calvin Sprouse & PHYS474 HW4 & 2024 January 24\\
\midrule
\end{tabularx}

% homework problems begin
\begin{enumerate}
\item A particle is trapped in a harmonic oscillator potential. We know that at \(t=0\), the particle can be represented by the wavefunction
\[\Psi(x,0) = A\left(2\psi_0(x) + 5\psi_2(x)\right),\]
where \(\psi_0\) and \(\psi_2\) are the stationary-state solutions for \(n=0\) and \(n=0\), respectively.
\begin{enumerate}[label=(\alph*)]
\item Normalize \(\Psi(x,0)\).
\begin{align*}
1 &= \braket*{\Psi(x,0)}{\Psi{x,0}}
\\&= A^2 \left( 4 \braket*{\psi_0}{\psi_0} + 25 \braket*{\psi_2}{\psi_2} + 20\braket*{\psi_0}{\psi_2} \right)
\\&= 29 A^2.
\end{align*}
Thus, \(A=\sqrt{1/29}\).

\item Construct \(\Psi(x,t)\) and then determine \(\left|\Psi(x,t)\right|^2\). Will \(\expval{x}\) depend on time?

The complete wavefunction is given by
\[\Psi(x,t) = \sqrt{\frac{1}{29}} \left( 2 \Psi_0 + 5 \Psi_2 \right),\]
where \(\Psi_n\) represents \(\psi_n(x)\phi_n(t)\). Thus,
\begin{align*}
\left| \Psi(x,t) \right|^2 &= \frac{1}{29}\left(2\Psi_0^* + 5\Psi_2^*\right)\left(2\Psi_0 + 5\Psi_2\right)
\\&= \frac{1}{29}\left(4\left|\Psi_0\right|^2 + 2\Psi_0^*\Psi_2 + 5\Psi_2^*\Psi_0 + 25\left|\Psi_2\right|^2\right)
\\&= \frac{1}{29}\left(4\left|\psi_0\right|^2
+ 20\psi_0\psi_2\left(e^{-2i\omega t} + e^{2i\omega t}\right)
+ 25\left|\psi_2\right|^2\right)
\\&= \frac{1}{29}\left(
4\left|\psi_0\right|^2
+ 10 \psi_0 \psi_2 \cosh(2i\omega t)
+ 25 \left|\psi_2\right|^2
\right)
\end{align*}

\end{enumerate}

\pagebreak
\item Consider the stationary states of the harmonic oscillator. As straightforwardly as possible, compute the following quantities for the \(n\)th stationary state \(\psi_n(x)\).
\begin{enumerate}[label=(\alph*)]
\item \(\expval{x}\)
\item \(\expval{x^2}\)
\item \(\expval{p}\)
\item \(\expval{p^2}\)
\item \(\expval{T}\)
\item Is the Heisenberg uncertainty principle satisfied for all values of \(n\)?
\end{enumerate}

\pagebreak
\item A particle in a harmonic oscillator potential is described by the normalized wavefunction
\[\ket*{\Psi(x,0)} = \frac{1}{\sqrt{5}}\ket*{1} + \frac{2}{\sqrt{5}}\ket*{2},\]
where \(\ket*{n}\) represents the \(n\)th stationary state.
\begin{enumerate}[label=(\alph*)]
\item What is \(\ket*{\Psi(x,t)}\)?
\item What is the expectation value for energy?
\item What is \(\langle x(t) \rangle\)?
\end{enumerate}

\end{enumerate}
\end{document}
