% document style header
\documentclass[a4paper, 12pt]{config/homework}

% import default packages
\usepackage{config/defpackages}
% import custom math commands
\usepackage{config/domath}

% end preamble
\begin{document}

% document title
\noindent
\begin{tabularx}{\textwidth}{>{\centering\arraybackslash}X>{\centering\arraybackslash}X>{\centering\arraybackslash}X}
Calvin Sprouse & PHYS474 Homework 7 & 2024 February 26\\
\midrule
\end{tabularx}

% homework problems begin
\begin{enumerate}
\item Determine the eigenfunctions and eigenvalues of the operator
\[\hat{Q} = \diff[2]{}{\phi}\]
where \(\phi\) is the angle in polar coordinates. Due to the rotational symmetry of the problem, your eigenfunctions, \(f\), should satisfy the boundary condition
\[f(\phi) = f(\phi + 2\pi).\]
Is the spectrum of eigenvalues degenerate or non-degenerate?



\pagebreak
\item Suppose that \(f(x)\) and \(g(x)\) are both eigenfunctions of an operator \(\hat{Q}\). The spectrum is degenerate such that both \(f(x)\) and \(g(x)\) have the same eigenvalue, \(q\).
\begin{enumerate}[label=(\alph*.)]
\item  Prove that any linear combination of \(f(x)\) and \(g(x)\) is also an eigenfunction of \(Q\). What is its eigenvalue?



\item An anti-hermitian operator obeys the following condition:
\[\hat{Q}^\dagger = -\hat{Q}.\]
Show that the expectation value of an anti-hermitian operator is imaginary.



\item Show that the commutator of two hermitian operators is anti-hermitian.



\item Show that the commutator of two anti-hermitian operators is also anti-hermitian.



\end{enumerate}
\pagebreak
\item We have two operators \(\hat{A}\) and \(\hat{B}\) each with two eigenstates. The eigenstates and corresponding eigenvalues are characterized by the equations
\begin{align*}
\hat{A}\psi_1 &= a_1 \psi_1, \\
\hat{A}\psi_2 &= a_2 \psi_2, \\
\hat{B}\phi_1 &= b_1 \phi_1, \\
\hat{B}\phi_2 &= b_2 \phi_2.
\end{align*}
Suppose we know that the eigenstates for each operator are related by
\begin{align*}
\psi_1 &= \frac{1}{5}\left(3\phi_1 + 4\phi_2\right), \\
\psi_2 &= \frac{1}{5}\left(4\phi_1 - 3\phi_2\right).
\end{align*}
\begin{enumerate}[label=(\alph*.)]
\item If observable \(A\) is measured and we obtain a value of \(a_1\), what is the state of the system in the instant after the measurement was made?



\item If \(B\) is now measured following the measurement in part (a.), what are the possible results and what are their associated probabilities?



\item If we measure \(A\) again immediately following the measurement of \(B\) in part (b.), what is the probability of obtaining \(a_1\)? This is tricky because we do not know what value of \(B\) we obtained in part (b.).



\end{enumerate}
\end{enumerate}
\end{document}
