% document style header
\documentclass[a4paper, 12pt]{config/homework}

% import default packages
\usepackage{config/defpackages}
% import custom math commands
\usepackage{config/domath}

% end preamble
\begin{document}

% document title
\noindent
\begin{tabularx}{\textwidth}{>{\centering\arraybackslash}X>{\centering\arraybackslash}X>{\centering\arraybackslash}X}
Calvin Sprouse & PHYS474 Homework 7 & 2024 February 28\\
\midrule
\end{tabularx}

% homework problems begin
\begin{enumerate}
\item Determine the eigenfunctions and eigenvalues of the operator
\[\hat{Q} = \diff[2]{}{\phi}\]
where \(\phi \) is the angle in polar coordinates. Due to the rotational symmetry of the problem, your eigenfunctions, \(f\), should satisfy the boundary condition
\[f(\phi) = f(\phi + 2\pi).\]

The eigenvalue equation
\[\hat{Q}f = qf,\]
where \(q\) is the eigenvalue, has the general solution
\[f=Ae^{q^b \phi}.\]
Then,
\[\diff[2]{f}{\phi} = \left(q^b\right)^2 Ae^{q^b \phi} = \left(q^b\right)^2 f,\]
where \(\left(q^b\right)^2\) is the eigenvalue yielding the condition
\[\left(q^b\right)^2 = q.\]
Thus, \(b=1/2\). Thusly the eigenfunction \(f\) is given by
\[f = A\exp\left(\sqrt{q}\phi\right).\]
We apply the boundary condition by letting \(\phi=0\). Thus, having neglected the amplitudes which are constant,
\[1 = \exp\left(\sqrt{q}2\pi\right).\]
This is satisfied for
\[\sqrt{q}2\pi = in 2\pi.\]
Thus,
\[q = -n^2.\]

Is the spectrum of eigenvalues degenerate or non-degenerate?

The spectrum of eigenvalues, \(n\in\ints \), is degenerate for \(n\ne 0\).

\pagebreak
\item Suppose that \(f(x)\) and \(g(x)\) are both eigenfunctions of an operator \(\hat{Q}\). The spectrum is degenerate such that both \(f(x)\) and \(g(x)\) have the same eigenvalue, \(q\).
\begin{enumerate}[label=(\alph*.)]
\item  Prove that any linear combination of \(f(x)\) and \(g(x)\) is also an eigenfunction of \(Q\). What is its eigenvalue?

\begin{proof}
Let \(f(x)\) and \(g(x)\) be defined as eigenfunctions of an operator \(\hat{Q}\) such that
\[\hat{Q}f=qf, \quad \hat{Q}g = qg.\]
Furthermore let \(a,b\in\reals \) be given. Then, let \(j(x)\) be a function defined as a linear combination of \(f\) and \(g\) such that
\[j(x) = af(x) + bg(x).\]
Then,
\begin{align*}
\hat{Q}j &= \hat{Q}\left[af + bg\right] \\
&= \hat{Q}af + \hat{Q}bg \\
&= a\hat{Q}f + b\hat{Q}g \\
&= aqf + bqg \\
&= q\left(af + bg\right) \\
&= qj.
\end{align*}
That is, a linear combination of eigenfunctions of \(\hat{Q}\), each with eigenvalue \(q\), is also an eigenfunction of \(\hat{Q}\) with eigenvalue \(q\).
\end{proof}

\item An anti-hermitian operator obeys the following condition:
\[\hat{Q}^\dagger = -\hat{Q}.\]
Show that the expectation value of an anti-hermitian operator is imaginary.

\begin{align*}
\langle Q \rangle &= \bra*{f}\hat{Q}\ket*{f}
\\&= \langle \hat{Q}^\dagger f | f \rangle
\\&= - \langle \hat{Q} f | f \rangle
\\&= - \left( \langle f | \hat{Q} f \rangle \right)^*
\\&= - \langle Q \rangle^*.
\end{align*}

\item Show that the commutator of two hermitian operators is anti-hermitian.

Let \(\hat{A}\) and \(\hat{B}\) be hermitian operators.
% Given \[\left(\hat{A}\hat{B}\right)^\dagger = \hat{B}^\dagger \hat{A}^\dagger,\]
The commutator of \(\hat{A}\) and \(\hat{B}\) is given by
\begin{align*}
\left[\hat{A}, \hat{B}\right] &= \hat{A}\hat{B} - \hat{B}\hat{A}
\\&= \hat{A}^\dagger \hat{B}^\dagger - \hat{B}^\dagger \hat{A}^\dagger
\\&= \left(\hat{B}\hat{A}\right)^\dagger - \left(\hat{A}\hat{B}\right)^\dagger
\\&= -\left(\hat{A}\hat{B} - \hat{B}\hat{A}\right)^\dagger
\\&= -\left[\hat{A}, \hat{B}\right]^\dagger.
\end{align*}

\item Show that the commutator of two anti-hermitian operators is also anti-hermitian.

Let \(\hat{A}\) and \(\hat{B}\) be anti-hermitian operators.
% Given \[\left(\hat{A}\hat{B}\right)^\dagger = \hat{B}^\dagger \hat{A}^\dagger,\]
The commutator of \(\hat{A}\) and \(\hat{B}\) is given by
\begin{align*}
\left[\hat{A}, \hat{B}\right] &= \hat{A}\hat{B} - \hat{B}\hat{A}
\\&= \hat{A}^\dagger \hat{B}^\dagger - \hat{B}^\dagger \hat{A}^\dagger
\\&= \left(\hat{B}\hat{A}\right)^\dagger - \left(\hat{A}\hat{B}\right)^\dagger
\\&= -\left(\hat{A}\hat{B} - \hat{B}\hat{A}\right)^\dagger
\\&= -\left[\hat{A}, \hat{B}\right]^\dagger.
\end{align*}
Because the commutator is the product of two operators if both are anti-hermitian the minus-sign has no effect. Thus this has the same form as in part (c.).

\end{enumerate}
\pagebreak
\item We have two operators \(\hat{A}\) and \(\hat{B}\) each with two eigenstates. The eigenstates and corresponding eigenvalues are characterized by the equations
\begin{align*}
\hat{A}\psi_1 &= a_1 \psi_1, \\
\hat{A}\psi_2 &= a_2 \psi_2, \\
\hat{B}\phi_1 &= b_1 \phi_1, \\
\hat{B}\phi_2 &= b_2 \phi_2.
\end{align*}
Suppose we know that the eigenstates for each operator are related by
\begin{align*}
\psi_1 &= \frac{1}{5}\left(3\phi_1 + 4\phi_2\right), \\
\psi_2 &= \frac{1}{5}\left(4\phi_1 - 3\phi_2\right).
\end{align*}
\begin{enumerate}[label=(\alph*.)]
\item If observable \(A\) is measured and we obtain a value of \(a_1\), what is the state of the system in the instant after the measurement was made?

To measure \(a_1\) the system must be in state \(\psi_1\). Therefore, a measurement of \(a_1\) puts the system in state \(\psi_1\).

\item If \(B\) is now measured following the measurement in part (a.), what are the possible results and what are their associated probabilities?

Before measurement the system is in state \(\psi_1\) which is a linear combination of the states \(\phi_1\) and \(\phi_2\). The probability of measuring each of these states, \(P(\phi_n)\), is given by the square of their amplitudes. Thus,
\begin{align*}
P(\phi_1) &= \left|\frac{3}{5}\right|^2 = \frac{9}{25}, \\
P(\phi_2) &= \left|\frac{4}{5}\right|^2 = \frac{16}{25}.
\end{align*}

\item If we measure \(A\) again immediately following the measurement of \(B\) in part (b.), what is the probability of obtaining \(a_1\)? This is tricky because we do not know what value of \(B\) we obtained in part (b.).

Expressing \(\phi_1\) and \(\phi_2\) as linear combinations of \(\psi_1\) and \(\psi_2\) we have
\begin{align*}
\phi_1 &= \frac{3}{5}\psi_1 + \frac{4}{5}\psi_2, \\
\phi_2 &= \frac{4}{5}\psi_1 - \frac{3}{5}\psi_2.
\end{align*}
Futhermore, we can define a new quantum state, \(\xi \), based on the unknown outcome of measuring the observable \(B\). This quantum state will be a superposition of outcomes of a measurement of \(B\) with amplitudes given by the root of the probability of measuring that quantum state. That is,
\begin{align*}
\xi &= \frac{3}{5}\phi_1 + \frac{4}{5}\phi_2
\\&= \frac{3}{5}\left(\frac{3}{5}\psi_1 + \frac{4}{5}\psi_2\right) + \frac{4}{5}\left(\frac{4}{5}\psi_1 - \frac{3}{5}\psi_2\right)
\\&= \psi_1.
\end{align*}
This, however, is wrong\footnote{Realization that this was wrong required a lot of ``this cant be right'' type thoughts followed by a probability tree from my prior stats class. I feel like there is some interpretation to be had here given that I returned \(\psi_1\) which was the wavefunction I started with.}. For as cool of a result as this would be it neglects the fact that measurement of \(B\) changes the wavefunction so we cannot construct \(\xi \) in this way. I kept this in the response because I thought it was neat.
We instead must consider the probability of measuring \(\psi_1\) given \(\phi_1\) or \(\phi_2\), \(P\left(\psi_1 \mid \phi_n\right)\). This is simply the product of the probability of making the first measurement, \(\phi_n\), and the probability of making the measurement \(\psi_1\).
\begin{align*}
P\left(\psi_1 \mid \phi_1\right) &= \frac{9}{25}\left|\frac{3}{5}\right|^2 = \frac{81}{625}, \\
P\left(\psi_1 \mid \phi_2\right) &= \frac{16}{25}\left|\frac{4}{5}\right|^2 = \frac{256}{625}.
\end{align*}
We then sum these two probabilities to find the probability of measuring \(\psi_1\) after making some measurement of observable \(B\):
\[P(\psi_1) = \frac{81}{625} + \frac{256}{625} = \frac{337}{625}.\]

\end{enumerate}
\end{enumerate}
\end{document}
