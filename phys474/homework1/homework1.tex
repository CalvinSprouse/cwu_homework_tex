% document style header
\documentclass[a4paper, 12pt]{config/homework}

% import default packages
\usepackage{config/defpackages}

% import custom math commands
\usepackage{config/domath}

% end preamble
\begin{document}

% document title
\noindent
\begin{tabularx}{\textwidth}{>{\centering\arraybackslash}X>{\centering\arraybackslash}X>{\centering\arraybackslash}X}
Calvin Sprouse & PHYS474 Homework 1 & Due 2024 Jan 10\\
\midrule
\end{tabularx}

% homework problems begin
Note that integrals have been evaluated by the provided integral table.
\begin{enumerate}
\item Consider the continuous Gaussian distribution, \(\rho(x) = Ae^{-\lambda (x-a)^2}\), where \(A\), \(a\), and \(\lambda \) are positive, real constants. Note that this is \underline{not} a wavefunction, but rather a distribution.
\begin{enumerate}[label=(\alph*)]
\item Normalize the distribution to determine \(A\).
\begin{align*}
1 &= \bint{-\infty}{\infty}{\rho(x)}{x}
\\&= \bint{-\infty}{\infty}{Ae^{-\lambda (x-a)^2}}{x}
\\&= A \bint{-\infty}{\infty}{e^{-\lambda (x-a)^2}}{x}
\\&= A \bint{-\infty}{\infty}{e^{-(\lambda x^2 -2\lambda ax + \lambda a^2)}}{x}
\\&= A \sqrt{\frac{\pi}{\lambda}} \exp\left({\frac{(-2\lambda a)^2 - 4\lambda^2 a^2}{4\lambda}}\right)
\\&= A \sqrt{\frac{\pi}{\lambda}}.
\end{align*}
Thus, \(A = \sqrt{\lambda / \pi}\).

\item Determine \(\braket{x}\), \(\braket{x^2}\), and \(\sigma \).

The average value of \(x\), or expectation value, is given by
\begin{align*}
\braket{x} &= \bint{-\infty}{\infty}{x \rho(x)}{x}
\\&= \sqrt{\frac{\lambda}{\pi}} \bint{-\infty}{\infty}{x \exp{\left(-\lambda(x-a)^2\right)}}{x}
\\&= \sqrt{\frac{\lambda}{\pi}} a \sqrt{\frac{\pi}{\lambda}}
\\&= a.
\end{align*}

The average of the squares of \(x\) is given by
\begin{align*}
\braket{x^2} &= \bint{-\infty}{\infty}{x^2 \rho(x)}{x}
\\&= \sqrt{\frac{\lambda}{\pi}} \bint{-\infty}{\infty}{x^2 \exp{\left(-\lambda(x-a)^2\right)}}{x}
\\&= \sqrt{\frac{\lambda}{\pi}} \bint{-\infty}{\infty}{x^2 \exp{\left(-\lambda x^2 +2\lambda ax - \lambda a^2\right)}}{x}
\\&= \sqrt{\frac{\lambda}{\pi}} \bint{-\infty}{\infty}{x^2 \exp{\left(-\lambda x^2 + 2\lambda a x\right)}\exp{\left(-\lambda a^2\right)}}{x}
\\&= \sqrt{\frac{\lambda}{\pi}} \exp{\left(-\lambda a^2\right)} \bint{-\infty}{\infty}{x^2 \exp{\left(-\lambda x^2 + 2\lambda a x\right)}}{x}
\\&= \sqrt{\frac{\lambda}{\pi}} \exp{\left(-\lambda a^2\right)} \frac{\sqrt{\pi} (2\lambda + (2\lambda a)^2)}{4 \lambda^{5/2}} \exp{\left(\frac{(2\lambda a)^2}{4\lambda}\right)}
\\&= \frac{2\lambda + (2\lambda a)^2}{4\lambda^2}
\\&= \frac{1 + 2\lambda a^2}{2\lambda}.
\end{align*}

The standard deviation, \(\sigma \), of \(\rho \) is given by
\begin{align*}
\sigma &= \sqrt{\braket{x^2} - \braket{x}^2}
\\&= \sqrt{\frac{1 + 2\lambda a^2}{2\lambda} - a^2}
\\&= \frac{1}{\sqrt{2\lambda}}.
\end{align*}
\end{enumerate}

\pagebreak
\item At time \(t=\qty{0}{\second}\), an electron is represented by the wave function,
\[\Psi(x, 0) = \begin{cases}
    A\frac{x}{a}, & 0 \le x \le a \\
    A \frac{(b-x)}{(b-a)}, & a \le x \le b \\
    0, & \text{otherwise}
\end{cases}\]
where \(A\), \(a\), and \(b\) are constants.
\begin{enumerate}[label=(\alph*)]
\item Normalize \(\Psi \).
\begin{align*}
1 &= \bint{-\infty}{\infty}{\left| \Psi \right|^2}{x}
\\&= \bint{-\infty}{0}{(0)^2}{x} + \bint{0}{a}{\left(A \frac{x}{a}\right)^2}{x} + \bint{a}{b}{\left(A\frac{(b-x)}{(b-a)}\right)^2}{x} + \bint{b}{\infty}{(0)^2}{x}
\\&= 0 + \frac{A^2}{a^2}\bint{0}{a}{x^2}{x} + \frac{A^2}{(b-a)^2}\bint{a}{b}{(b-x)^2}{x} + 0
\\&= \frac{A^2}{a^2} \left[\frac{1}{3}x^3\right]_0^a + \frac{A^2}{(b-a)^2}\left[-\frac{a^3}{3} + a^2b - ab^2 + \frac{b^3}{3}\right]_a^b
\\&= \frac{A^2}{a^2} \frac{a^3}{3} + \frac{A^2}{(b-a)^2}\frac{(b-a)^3}{3}
\\&= A^2 \left(\frac{a}{3} + \frac{b-a}{3}\right)
\\&= A^2 \frac{b}{3}.
\end{align*}
Thus, \(A = \sqrt{3/b}\).

\item Sketch \(\Psi(x,0)\) as a function of \(x\).

\item Where is the electron most likely to be found at \(t=\qty{0}{\second}\)?

\item What is the probability the electron will be found in the region \(x \le a\)? Check your result in the limiting case where \(b=a\) and \(b=2a\).

\item Determine \(\braket{x}\).
\begin{align*}
\braket{x} &= \bint{-\infty}{\infty}{x\left| \Psi \right|^2}{x}
\\&= \bint{0}{a}{x\left(\sqrt{\frac{3}{b}} \frac{x}{a}\right)^2}{x} + \bint{a}{b}{x\left(\sqrt{\frac{3}{b}}\frac{(b-x)}{(b-a)}\right)^2}{x}
\\&= \frac{3}{ba^2}\bint{0}{a}{x^3}{x} + \frac{3}{b}\frac{1}{(b-a)^2}\bint{a}{b}{(b^2 x - 2bx^2 + x^3)}{x}
\\&= \frac{3}{ba^2}\left[\frac{1}{4}x^4\right]_0^a + \frac{3}{b}\frac{1}{(b-a)^2} \left[\frac{b^2}{2}x^2 - \frac{2b}{3}x^3 + \frac{1}{4}x^4\right]_a^b
\\&= \frac{3}{4}\frac{a^2}{b} + 3 \frac{1}{b(b-a)^2}\left(\frac{b^4}{2} - \frac{2b^4}{3} + \frac{b^4}{4} - \frac{a^2 b^2}{2} + \frac{2ba^3}{3} - \frac{a^4}{4}\right)
\\&=
\end{align*}

\end{enumerate}
\end{enumerate}
\end{document}
