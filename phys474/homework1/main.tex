% document style header
\documentclass[a4paper, 12pt]{config/homework}

% import default packages
\usepackage{config/defpackages}

% import custom math commands
\usepackage{config/domath}

% end preamble
\begin{document}

% document title
\noindent
\begin{tabularx}{\textwidth}{>{\centering\arraybackslash}X>{\centering\arraybackslash}X>{\centering\arraybackslash}X}
Calvin Sprouse & PHYS474 Homework 1 & Due 2024 Jan 10\\
\midrule
\end{tabularx}

% homework problems begin
\begin{enumerate}
\item Consider the continuous Gaussian distribution, \(\rho(x) = Ae^{-\lambda (x-a)^2}\), where \(A\), \(a\), and \(\lambda \) are positive, real constants. Note that this is \underline{not} a wavefunction, but rather a distribution.
\begin{enumerate}[label=(\alph*)]
\item Normalize the distribution to determine \(A\).
\begin{align*}
1 &= \bint{-\infty}{\infty}{\rho(x)}{x}
\\&= \bint{-\infty}{\infty}{Ae^{-\lambda (x-a)^2}}{x}
\\&= A \bint{-\infty}{\infty}{e^{-\lambda (x-a)^2}}{x}
\\&= A \bint{-\infty}{\infty}{e^{-(\lambda x^2 -2\lambda ax + \lambda a^2)}}{x}
\\&= A \sqrt{\frac{\pi}{\lambda}} \exp\left({\frac{(-2\lambda a)^2 - 4\lambda^2}{4\lambda}}\right)
\\&= A \sqrt{\frac{\pi}{\lambda}} \exp\left(\lambda a^2 - \lambda\right)
\\\sqrt{\frac{\lambda}{\pi}} \exp\left(-\lambda a^2 + \lambda\right) &= A
\end{align*}

\item Determine \(\braket{x}\), \(\braket{x^2}\), and \(\sigma\).
\end{enumerate}

\pagebreak
\item At time \(t=\qty{0}{\second}\), an electron is represented by the wave function,
\[\Psi(x, 0) = \begin{cases}
    A\frac{x}{a}, & 0 \le x \le a \\
    A \frac{(b-x)}{(b-a)}, & a \le x \le b \\
    0, & \text{otherwise}
\end{cases}\]
where \(A\), \(a\), and \(b\) are constants.
\begin{enumerate}[label=(\alph*)]
\item Normalize \(\Psi\).
\item Sketch \(\Psi(x,0)\) as a function of \(x\).
\item Where is the electron most likely to be found at \(t=\qty{0}{\second}\)?
\item What is the probability the electron will be found in the region \(x \le a\)? Check your result in the limiting case where \(b=a\) and \(b=2a\).
\item Determine \(\braket{x}\).
\end{enumerate}
\end{enumerate}
\end{document}
