% document style header
\documentclass[a4paper, 12pt]{config/homework}

% import default packages
\usepackage{config/defpackages}
% import custom math commands
\usepackage{config/domath}

% end preamble
\begin{document}

% document title
\noindent
\begin{tabularx}{\textwidth}{>{\centering\arraybackslash}X>{\centering\arraybackslash}X>{\centering\arraybackslash}X}
Calvin Sprouse & PHYS474 Exam 1 & 2024 February 01\\
\midrule
\end{tabularx}

% homework problems begin
\begin{enumerate}
\item You trap an electron in an infinite-square well with size \(a\). At time \(t=0\), we know that the electron is in the following state:
\[\Psi(x,0) = \begin{cases}
    bx^2, & 0 \le x \le a \\
    \phantom{b^2}0, & \text{otherwise}
\end{cases}\]
\begin{enumerate}[label=(\alph*)]
\item (\textit{5 points}) Assuming it is a real constant, determine \(b\) by normalizing \(\Psi(x,0)\).

\begin{align*}
1 &= \bint{-\infty}{\infty}{\left|\Psi(x,0)\right|^2}{x}
\\&= \bint{-\infty}{\infty}{\Psi^*(x,0)\Psi(x,0)}{x}
\\&= \bint{-\infty}{0}{0}{x} + \bint{0}{a}{(bx^2)(bx^2)}{x} + \bint{-\infty}{0}{0}{x}
\\&= b^2 \bint{0}{a}{x^4}{x}
\\&= b^2 {\left[ \frac{1}{5}x^5 \right]}_0^a
\\&= b^2 \frac{a^5}{5}.
\end{align*}
Solving for \(b\) we find
\[b = \sqrt{\frac{5}{a^5}}.\]

\pagebreak
\item (\textit{8 points}) Construct the function \(\Psi(x,t)\) for this electron (\textit{i.e.}, including time-dependence).

We will express \(\Psi \) as a linear combination of stationary states \(\psi_n(x)\) with associated time-dependence \(\phi_n(t)\):
\[\Psi(x,t)
= \sum_{n=1}^{\infty}{c_n\psi_n(x)\phi_n(t)}
= \sum_{n=1}^{\infty}{c_n \sqrt{\frac{2}{a}}}\sin\left(\frac{n\pi}{a}x\right)\exp\left(-\frac{i E_n}{\hbar}t\right).\]
The coefficients \(c_n\) can be found by applying Fourier's trick:
\begin{align*}
c_n &= \bint{-\infty}{\infty}{\psi_n\Psi(x,0)}{x}
\\&= \bint{-\infty}{0}{\psi_n\times 0}{x} + \bint{0}{a}{\psi_n\Psi(x,0)}{x} + \bint{a}{\infty}{\psi_n\times 0}{x}
\\&= \bint{0}{a}{\sqrt{\frac{2}{a}}\sin\left(\frac{n\pi}{a}x\right)\Psi(x,0)}{x}
\\&= b\sqrt{\frac{2}{a}} \bint{0}{a}{x^2\sin\left(\frac{n\pi}{a}x\right)}{x}
\\&= \sqrt{\frac{10}{a^6}} {\left[\frac{2x}{\left(\frac{n\pi}{a}\right)^2} \sin\left(\frac{n\pi}{a}x\right) + \left(\frac{2}{\left(\frac{n\pi}{a}\right)^3} - \frac{x^2}{\left(\frac{n\pi}{a}\right)}\right)\cos\left(\frac{n\pi}{a}x\right)\right]}_0^a
\\&= \sqrt{\frac{10}{a^6}} \left[
    \frac{2a}{\left(\frac{n\pi}{a}\right)^2} \sin\left(n\pi\right)
    + \left(\frac{2}{\left(\frac{n\pi}{a}\right)^3} - \frac{a^2}{\left(\frac{n\pi}{a}\right)}\right) \cos\left(n\pi\right)
    \right.
\\&\phantom{= \sqrt{\frac{10}{a^6}}} \left.
    - 0 - \left(\frac{2}{\left(\frac{n\pi}{a}\right)^3}\right) \cos\left(0\right)
    \right]
\\&= \sqrt{\frac{10}{a^6}}\left(
    0 + \left(\frac{2}{\left(\frac{n\pi}{a}\right)^3} - \frac{a^2}{\left(\frac{n\pi}{a}\right)}\right) \cos\left(n\pi\right)
    - 0 - \left(\frac{2}{\left(\frac{n\pi}{a}\right)^3}\right)
\right)
\\&= \sqrt{\frac{10}{a^6}} \left(
    \left(\frac{2}{\left(\frac{n\pi}{a}\right)^3} - \frac{a^2}{\left(\frac{n\pi}{a}\right)}\right) \cos\left(n\pi\right)
    - \frac{2}{\left(\frac{n\pi}{a}\right)^3}
\right)
\\&= \begin{cases}
    -\sqrt{\frac{10}{a^6}}\frac{a^3}{n\pi}, & n\ \text{is even}\ \\
    -\sqrt{\frac{10}{a^6}}\left(\frac{4a^3}{(n\pi)^3} - \frac{a^3}{n\pi}\right), & n\ \text{is odd}
\end{cases}.
\end{align*}

\pagebreak
\item (\textit{5 points}) If you measure the electron's energy, the probability of obtaining \(E_n\) can be denoted \(P(E_n)\). Determine \(P(E_n)\) for \(n=\{1,2,3,4,5\} \).

The probability of measuring the energy \(E_n\) is given by the probability of finding the electron in stationary state \(n\); this is given by
\[P(E_n) = \left|c_n\right|^2.\]
Thus,
\begin{align*}
P(E_1) &= \left|c_1\right|^2 = ,
\\P(E_3) &= \left|c_3\right|^2 = ,
\\P(E_2) &= \left|c_2\right|^2 = ,
\\P(E_4) &= \left|c_4\right|^2 = ,
\\P(E_5) &= \left|c_5\right|^2 = .
\end{align*}

\item (\textit{2 points}) What is the probability of obtaining an energy larger than \(E_5\) in an energy measurement on this electron?

The probability of obtaining energy, \(H\), greater than \(E_5\) can be denoted
\[P(H > E_5) = \sum_{n=6}^{\infty}{P(E_n)}.\]
Since there is a probability 1 of obtaining an energy we can instead write
\[P(H > E_5) = 1 - P(H < 5) = 1 - \sum_{n=1}^{5}{P(E_n)}.\]
The probability of obtaining energy \(P(E_n)\) for \(n=1,2,3,4,5\) was found above. Thus,
\begin{align*}
P(H > E_5) &= 1 - \sum_{n=1}^{5}{P(E_n)}
\\&= 1 - \left(P(E_1) + P(E_2) + P(E_3) + P(E_4) + P(E_5)\right)
\\&= 1 - \left(  \right)
\\&= .
\end{align*}

\end{enumerate}
\end{enumerate}
\end{document}
