% document style header
\documentclass[a4paper, 12pt]{config/homework}

% import default packages
\usepackage{config/defpackages}
% import custom math commands
\usepackage{config/domath}

% end preamble
\begin{document}

% document title
\noindent
\begin{tabularx}{\textwidth}{>{\centering\arraybackslash}X>{\centering\arraybackslash}X>{\centering\arraybackslash}X}
Calvin Sprouse & PHYS474 Exam 2 & 2024 February 29\\
\midrule
\end{tabularx}

% homework problems begin
\vspace{\baselineskip}\noindent
An electron is sent from \(x\to-\infty \) towards a potential barrier,
\[V(x) = \begin{cases}
0, & x < 0, \\ V_0, & x > 0.
\end{cases}\]
The electron's energy is \(E>V_0\).
\begin{enumerate}[label=(\alph*.)]
\item (\textit{5 points}) Solve the time-independent Schr{\"o}dinger equation for \(\psi_\text{I}(x)\) and \(\psi_\text{II}(x)\), which are solutions for \(x<0\) and \(x>0\) respectively. Like we did in class for finite square-wells, combine the collection of constants, \(\hbar,m,V_0\), and \(E\), into real quantities \(k,\ell\in\reals \).



\item (\textit{5 points}) Apply boundary conditions at \(x=0\) and solve for the reflection and transmission coefficients \(R\) and \(T\). Remember that these coefficients are defined
\[R \equiv \frac{|B|^2}{|A|^2}, \quad T\equiv\frac{|F|^2}{|A|^2},\]
where \(A\) is the incident amplitude, \(B\) is the reflected amplitude, and \(F\) is the transmitted amplitude.



\item (\textit{2 points}) Using your results from part (b.), calculate \(R+T\). What do you expect \(R+T\) should equal? Do you have any ideas for why \(R+T\) gives an unexpected answer in this case?

Given that \(R\) and \(T\) are interpreted as probabilities, and that a particle can only be reflected or transmitted at a barrier, we should expect
\(R + T = 1. \)

\item (\textit{5 points}) Calculate a quantity called the probability current,
\[j(x) \equiv \frac{i\hbar}{2m}\left[\psi\diffp{\psi^*}{x} - \psi^*\diffp{\psi}{x}\right],\]
on both sides of the barrier. Let \(j_\text{I}(x)\) be the probability current for \(x<0\) and \(j_\text{II}(x)\) be the probability current for \(x>0\). Evaluate them at \(x=0\); that is, \(j_\text{I}(0)\) and \(j_\text{II}(0)\). Don't forget that \(A,B\), and \(F\) could be complex.



\item (\textit{3 points}) It must be true that \(j_\text{I}(0)=j_\text{II}(0)\). Construct an equation using this conservation rule and your answers from part (d.). Divide this equation by \(|A|^2\) and rearrange it so that you determine what linear combination of \(R\) and \(T\) sums to 1.



\end{enumerate}
\end{document}
