% document style header
\documentclass[a4paper, 12pt]{config/homework}

% import default packages
\usepackage{config/defpackages}
% import custom math commands
\usepackage{config/domath}

% end preamble
\begin{document}

% document title
\noindent
\begin{tabularx}{\textwidth}{>{\centering\arraybackslash}X>{\centering\arraybackslash}X>{\centering\arraybackslash}X}
Calvin Sprouse & PHYS 474 HW5 & 2024 February 07\\
\midrule
\end{tabularx}

% homework problems begin
\begin{enumerate}
\item Consider a free particle that is described at \(t=\qty{0}{\second}\) by the wavefunction, \(\Psi(x,0)=Ae^{-a|x|}\) where both \(A\) and \(a\) are positive, real constants.
\begin{enumerate}
\item Normalize \(\Psi(x,0)\).

We begin by expressing \(\Psi \) as a piecewise function:
\[\Psi(x,0) = \begin{cases}
Ae^{-ax}, & x > 0 \\ Ae^{ax}, & x < 0
\end{cases}\]
Now, \(\Psi \) can be normalized by the typical method:
\begin{align*}
1 &= \bint{-\infty}{\infty}{\left|\Psi\right|^2}{x}
\\&= A^2\bint{-\infty}{0}{e^{2ax}} + A^2\bint{0}{\infty}{e^{-2ax}}{x}
\\&= A^2\left( \left[\frac{1}{2a}e^{2ax}\right]_{-\infty}^0 + \left[-\frac{1}{2a}e^{-2ax}\right]_0^\infty \right)
\\&= A^2\left( \frac{1}{2a} - 0 + 0 + \frac{1}{2a} \right)
\\&= A^2 \frac{1}{a}.
\end{align*}
Thus, \(A=\sqrt{a}\).

\item Determine \(\phi(k)\).

We begin with the typical \(\phi(k)\) and exploit Euler's formula:
\[\phi(k)
= \frac{1}{\sqrt{2\pi}} \bint{-\infty}{\infty}{\Psi(x,0)e^{-ikx}}{x}
= \sqrt{\frac{a}{2\pi}} \bint{-\infty}{\infty}{e^{-a|x|}\left(\cos(kx)-i\sin(kx)\right)}{x}.\]
\(\Psi \) has not been expressed in piecewise form because we can simplify here. \(\Psi \) is even, \(\cos \) is even, and \(\sin \) is odd. Our limits of integration are symmetric so the \(\sin \) term integrates to 0. We are left with only even functions over symmetric limits so we can further simplify to
\[\phi(k) = \sqrt{\frac{2a}{\pi}}\bint{0}{\infty}{e^{-ax}\cos(kx)}{x}.\]
The \(\cos \) function can be expressed in terms of imaginary exponentials:
\[\phi(k) = \frac{1}{2}\sqrt{\frac{2a}{\pi}}\bint{0}{\infty}{e^{-ax}\left(e^{ikx} + e^{-ikx}\right)}{x}.\]
These represent waves moving in the \(+x\) and \(-x\) direction. Then,
\[\phi(k) = \sqrt{\frac{a}{2\pi}}\left(\bint{0}{\infty}{e^{ikx-ax}}{x} + \bint{0}{\infty}{e^{-ikx - ax}}{x}\right).\]
These are deceptively simple integrals and evaluate to
\begin{align*}
\phi(k) &= \sqrt{\frac{a}{2\pi}}\left[
    e^{-ax}\left(
    \frac{e^{ikx}}{ik-a}
    + \frac{e^{-ikx}}{ik+a}
    \right)
\right]_0^\infty.
\\&= \sqrt{\frac{a}{2\pi}}\left(0 - \frac{1}{ik-a} - \frac{1}{ik+a} \right)
\\&= -\sqrt{\frac{a}{2\pi}}\left(\frac{(ik+a) - (ik-a)}{(ik-a)(ik+1)}\right)
\\&= -\sqrt{\frac{a}{2\pi}}\frac{2a}{(ik-a)(ik+a)}
\\&= \frac{1}{k^2 + a^2} \sqrt{\frac{2a^3}{\pi}}.
\end{align*}

\item Construct \(\Psi(x,t)\) in the form of an integral.

The full wavefunction is then given by
\begin{align*}
\Psi(x,t) &= \frac{1}{\sqrt{2\pi}}\bint{-\infty}{\infty}{\phi(k)e^{i(kx- \frac{\hbar k^2}{2m}t)}}{k}
\\&= \frac{a^{3/2}}{\pi} \bint{-\infty}{\infty}{\frac{\exp\left[i\left(kx - \frac{\hbar k^2}{2m}t\right)\right]}{k^2 + a^2}}{k}.
\end{align*}

\item Evaluate the integral for \(\Psi(x,t)\) in the limiting cases of a very large \(a\) and a very small \(a\).

\end{enumerate}
\pagebreak
\item As we discussed in class, the time-independent Schr{\"o}dinger equation for the free particle has solutions that look like \(Ae^{ikx} + Be^{-ikx}\) or like \(C\cos(kx) + D\sin(kx)\). Show that these are equivalent solutions. Determine what the constants \(C\) and \(D\) are as a function of \(A\) and \(B\) and vice versa.

We begin with the Euler formula:
\begin{align*}
Ae^{ikx} + Be^{-ikx} &= A\left(\cos(kx) + i \sin(kx)\right) + B\left(\cos(-kx) + i\sin(-kx)\right)
\\&= A\cos(kx) + B\cos(kx) + Ai\sin(kx) - Bi\sin(kx)
\\&= (A+B)\cos(kx) + (A-B)i\sin(kx).
\end{align*}
Thus,
\[C = A+B,\quad D=i(A-B).\]
Some simple arrangement yields
\[A = \frac{C-D}{2}, \quad B = \frac{C+D}{2}.\]

\item Consider a bead with mass \(m\) that slides frictionlessly around a circular wire ring with circumference \(L\). We can think about this problem like a free particle assuming boundary condition of the form \(\psi(x+L) = \psi(x)\). Determine the normalized stationary states and their corresponding energies. You should find two distinct solutions for each energy (so there is a two-fold degeneracy in this system). These two states represent clockwise and counter-clockwise rotation.

We begin with the TISE with \(V(x)=0\), \(k^2=2mE/\hbar^2\), and some typical re-arrangement has been done:
\[\diff[2]{\psi}{x} = -k^2\psi.\]
This differential equation has general solutions of the form
\[\psi(x) = Ae^{ikx} + Be^{-ikx}.\]
We apply the boundary condition \(\psi(x+L) = \psi(x)\):
\[A\exp\left(ik(x+L)\right) + B\exp\left(-ik(x+L)\right) = A\exp\left(ikx\right) + B\exp\left(-ikx\right).\]
This is arranged to the (hopefully useful) form
\[A\exp(ikx)\left(\exp(ikL)-1\right) = B\exp(-ikx)\left(1-\exp(-ikL)\right).\]
To find the coefficients \(A\) and \(B\) we exploit the fact that this is valid for all \(x\). Let \(x=0\); then,
\[A\left(\exp(ikL) - 1\right) = B\left(1-\exp(-ikL)\right),\]
which on its own does not tell us much. Since we are trapped on a circular ring a displacement \(L\) corresponds to \(2\pi\) radians. Thus, \(kL=2\pi\). As cool as this is to notice substituting \(x=L\) and applying the relation \(kL=2\pi\) yields the extremely insightful expression \(0=0\). We instead let \(x=L/2\); then,
\[Ae^{i\pi}\left(e^{i\pi}-1\right) = Be^{-i\pi}\left(1 - e^{-i\pi}\right) \Rightarrow A = -B.\]
Then,
\begin{align*}
0 &= A\left(\exp(ikL) - 1\right) + B\left(\exp(-ikL)-1\right)
\\&= A\left(\exp(ikL) - 1\right) -A\left(\exp(-ikL)-1\right)
\\&= A\left(e^{ikL}-e^{-ikL}\right)
\\&= i2A\sin(kL).
\end{align*}
Iff \(A\ne 0\), \(kL=n\pi\) where \(n\in\ints^+\). From the earlier definition of \(k\) we find
\[E = \frac{k^2\hbar^2}{2m} = \frac{n^2 \pi^2 \hbar^2}{2mL^2}.\]

\pagebreak
\item Consider a particle interacting with a potential energy given by
\[V(x)\begin{cases}
\infty, & x < 0 \\ -\alpha\delta(x-b), & x>0
\end{cases}\]
\begin{enumerate}
\item Determine the bound-state solutions (assume \(E<0\)) to the time-independent Schr{\"o}dinger equation in three different regions of \(x\): \(x<0\), \(0 \le x \le b\), and \(x > b\). Define \(K^2=-2mE/\hbar^2\) in your solutions.

In region 1 the potential is infinite so the particle cannot exist there, thus
\[\psi_1(x)=0.\]
In region 2 following the example in \S2.6
\[\psi_{2}(x) = B\exp(-Kx).\]
Similarly, in region 3
\[\psi_{3}(x) = F\exp(-Kx).\]

\item Demand continuity of the wavefunctions at \(x=0\) and \(x=a\) to reduce the number of unknown constant parameters.

The first continuity dictates that \(\psi_1(0) = \psi_2(0)\):
\[0 = B\exp(-K\times0) = B.\]
Thus, \(B=0\).
The second continuity dictates that \(\psi_2(a)=\psi_3(a)\):
\[0 = F\exp(-K a).\]
Thus, \(F=0\) or \(K=\infty\) or \(a=\infty\). The only physically reasonable solution is that \(F=0\). Unfortunately we are forced to conclude that no particle can exist in this type of potential, but that doesnt make a lot of sense.

I appreciate the extension but I've been at this for awhile so I'm going to take this as my dropped homework and come to some office hours.

\item Show that the discontinuity in the derivative of the wavefunctions at \(x=a\) is given by
\[\left.\diff{\psi}{x}\right|_{a+\epsilon} - \left.\diff{\psi}{x}\right|_{a-\epsilon} = - \frac{2m\alpha}{\hbar^2}\psi(a).\]


\item Using the boundary condition from part (c), show that
\[\frac{K\hbar^2}{m\alpha} = 1 - e^{-2Ka}.\]
This transcendental equation relates \(K\) with \(\alpha \). How many different \(K\) values will solve the equation? You might try to graph the left-hand and right-hand equations vs.\ \( K \) to see where they intersect.



\end{enumerate}
\end{enumerate}
\end{document}
