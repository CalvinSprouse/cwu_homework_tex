% document style header
\documentclass[a4paper, 12pt]{config/homework}

% import default packages
\usepackage{config/defpackages}
% import custom math commands
\usepackage{config/domath}

% end preamble
\begin{document}

% document title
\noindent
\begin{tabularx}{\textwidth}{>{\centering\arraybackslash}X>{\centering\arraybackslash}X>{\centering\arraybackslash}X}
Calvin Sprouse & PHYS474 Homework 6 & 2024 February 22\\
\midrule
\end{tabularx}

% homework problems begin
\begin{enumerate}
\item Consider a particle interacting with the potential barrier:
\[V(x) = \begin{cases}
0, & x < 0, \\ V_0, & x > 0.
\end{cases}\]
What kinds of wavefunction solutions are possible?
% I am looking for a qualitative discussion rather than calculations

Free particle wavefunctions are possible solutions to this potential.
In the region \(x<0\) we have the typical free particle condition \(V(x)=0\).
At the barrier there exist reflected and transmitted components of the free particle solution.
In the region \(x>0\) free particle solutions can exist only if the particle energy is greater than the potential.

\vspace{\baselineskip}
\item We calculated the even bound-state wavefunctions for the finite-square well in class. Now, preform this calculation for the odd wavefunction solutions. By determining the wavefunction and apply appropriate boundary conditions, calculate the transcendental equation that will allow you to determine the bound-state energies. Use \(Z = \ell a\), and \(Z_0 = \frac{a}{\hbar}\sqrt{2mV_0}\) to express your transcendental equation more compactly.
Plot the left- and right-hand sides of the equation as a function of \(Z\) for \(Z_0=2\). How many bound sates with odd wavefunctions are there for \(Z_0=2\)?



\pagebreak
\item Consider a particle interacting with the following potential energy landscape:
\[V(x) = \begin{cases}
\infty, & x < 0, \\ -V_0, & 0 < x < a, \\ 0, & x > a.
\end{cases}\]
\begin{enumerate}[label=(\alph*.)]
\item Determine the bound-state wavefunctions in three regions: (i.)~\(x < 0\), (ii.) \(0 < x < a\), and (iii.)~\(x > a\).
Note that this potential energy is not symmetric, so you cannot assume solutions will be alternatingly even and odd; instead, you must use the most general solution for the middle region's wavefunction.



\item Apply appropriate boundary conditions at \(x=0\) and \(x=a\). Combine the results to determine the transcendental equation that governs the bound-state energies. Does it look familiar?



\end{enumerate}
\end{enumerate}
\end{document}
