% document style header
\documentclass[a4paper, 12pt]{config/homework}

% import default packages
\usepackage{config/defpackages}

% import custom math commands
\usepackage{config/domath}

% end preamble
\begin{document}

% document title
\noindent
\begin{tabularx}{\textwidth}{>{\centering\arraybackslash}X>{\centering\arraybackslash}X>{\centering\arraybackslash}X}
Calvin Sprouse & PHYS474 Homework 2 & Due 2024 Jan 17\\
\midrule
\end{tabularx}

% homework problems begin
\begin{enumerate}
\item Let the wavefunction \(\Psi(x,t)\) be a solution to the time-dependent Schr{\"o}dinger equation when the potential energy is given by \(V(x)\). What is the solution to the Shr{\"o}dinger equation if we now consider a potential of \(V(x) + V_0\) where \(V_0\) is a real positive constant.

\pagebreak
\item A particle is observed in a quantum state described by the wavefunction
\[\Psi(x,t) = A\exp\left(-a\left(\frac{mx^2}{\hbar}+it\right)\right),\]
where \(A\) and \(a\) are real positive constants.
\begin{enumerate}[label=(\alph*)]
\item Normalize \(\Psi \).

As shown in Griffiths, if \(\Psi\) is normalized at any time \(t\) it is normalized at all times \(t\). We will normalize \(\Psi\) at \(t=0\).
\begin{align*}
1 &= \bint{-\infty}{\infty}{\Psi(x,t=0)}{x}
\\&= \bint{-\infty}{\infty}{A\exp\left(\frac{-amx^2}{\hbar} - (ai\times0)\right)}{x}
\\&= A \bint{-\infty}{\infty}{\exp\left(\frac{-amx^2}{\hbar}\right)}{x}
\\&= A \sqrt{\frac{\pi}{\left(\frac{am}{\hbar}\right)}}
\\&= A \sqrt{\frac{\pi\hbar}{am}}.
\end{align*}
Thus, \(A = \sqrt{am / (\pi\hbar)}\).

\pagebreak
\item What is the potential \(V(x)\) that this particle finds itself within?

\(\Psi\) is, by definition, a solution to the Schr{\"o}dinger equation. Thus,
\[i\hbar \diffp{\Psi}{t} = -\frac{\hbar^2}{2m} \diffp[2]{\Psi}{x} + V\Psi.\]
Arranging for the potential energy function \(V\) we get
\[V = \frac{1}{\Psi} \left(i\hbar \diffp{\Psi}{t} + \frac{\hbar^2}{2m} \diffp[2]{\Psi}{x}\right).\]
To make these partial derivatives less threatening, we can begin by writing \(\Psi\) in the form \(\Psi = A \psi(x) \phi(t)\).
The \(A\) component is simply \(A\), as solved for above during normalization.
The exponential term in \(\Psi\) can be separated into an \(x\)-dependent and \(t\)-dependent component;
\[\exp\left(\frac{-amx^2}{\hbar} - ait\right) = \exp\left(-\frac{amx^2}{\hbar}\right)\exp\left(-ait\right),\]
which become \(\psi\) and \(\phi\) respectively. We know
\[\phi(t) = \exp\left(-\frac{iEt}{\hbar}\right) = \exp\left(-ait\right),\]
thus \(a = E / \hbar\). We can now write \(V\) in a more approachable way with ordinary derivatives:
\[V = \frac{1}{A\psi\phi} \left(i\hbar A\psi \diff{\phi}{t} + \frac{\hbar^2}{2m} A\phi \diff[2]{\psi}{x}\right).\]
The ordinary derivatives are
\[\diff{\phi}{t} = \diff{}{t}\left[\exp(-ait)\right] = -ai \exp(-ait) = -ai \phi,\]
and
\[\diff[2]{\psi}{x} = \diff[2]{}{x}\left[\exp\left(-\frac{amx^2}{\hbar}\right)\right] = -\frac{2amx}{\hbar} \exp\left(-\frac{amx^2}{\hbar}\right) = -\frac{2amx}{\hbar} \psi.\]
Thus the potential energy function \(V\) is given by
\begin{align*}
V &= \frac{1}{A\psi\phi} \left(i\hbar A\psi \diff{\phi}{t} + \frac{\hbar^2}{2m} A\phi \diff[2]{\psi}{x}\right)
\\&= \frac{1}{A\psi\phi} \left(-i\hbar A\psi  \frac{E}{\hbar} i \phi + \frac{\hbar^2}{2m} A\phi \frac{-2 E mx}{\hbar^2} \psi\right)
\\&= E - Ex
\\&= E (1-x).
\end{align*}

\pagebreak
\item Determine the expectation values \(\braket{x}\), \(\braket{x^2}\), \(\braket{p}\), \(\braket{p^2}\).
\begin{enumerate}[label=\roman*.]
\item
\begin{align*}
\braket{x} &= \bint{-\infty}{\infty}{x \left|\psi\right|}{x}
\end{align*}
\end{enumerate}

\item Determine the standard deviations for position, \(\sigma_x\), and momentum, \(\sigma_p\).
\item Are your values for \(\sigma_x\) and \(\sigma_p\) consistent with the uncertainty principle?
\end{enumerate}

\pagebreak
\item An electron is trapped in a harmonic quadratic potential. Suppose the expectation value for its position is given by \(\braket{x} = \frac{a}{2}\sin(\omega t)\). Here, \(a\) is a real constant with units of length and \(\omega \) is an angular frequency. What, if anything, can be concluded about the electron's momentum?
\end{enumerate}
\end{document}
