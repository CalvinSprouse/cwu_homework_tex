% document style header
\documentclass[a4paper, 12pt]{config/homework}

% import default packages
\usepackage{config/defpackages}

% import custom math commands
\usepackage{config/domath}

% end preamble
\begin{document}

% document title
\noindent
\begin{tabularx}{\textwidth}{>{\centering\arraybackslash}X>{\centering\arraybackslash}X>{\centering\arraybackslash}X}
Calvin Sprouse & PHYS474 Homework 2 & Due 2024 Jan 17\\
\midrule
\end{tabularx}

% homework problems begin
\begin{enumerate}
\item Let the wavefunction \(\Psi(x,t)\) be a solution to the time-dependent Schr{\"o}dinger equation when the potential energy is given by \(V(x)\). What is the solution to the Shr{\"o}dinger equation if we now consider a potential of \(V(x) + V_0\) where \(V_0\) is a real positive constant.

\pagebreak
\item A particle is observed in a quantum state described by the wavefunction
\[\Psi(x,t) = A\exp\left(-a\left(\frac{mx^2}{\hbar}+it\right)\right),\]
where \(A\) and \(a\) are real positive constants.
\begin{enumerate}[label=(\alph*)]
\item Normalize \(\Psi \).
\item What is the potential \(V(x)\) that this particle finds itself within?
\item Determine the expectation values \(\braket{x}\), \(\braket{x^2}\), \(\braket{p}\), \(\braket{p^2}\).
\item Determine the standard deviations for position, \(\sigma_x\), and momentum, \(\sigma_p\).
\item Are your values for \(\sigma_x\) and \(\sigma_p\) consistent with the uncertainty principle?
\end{enumerate}

\pagebreak
\item An electron is trapped in a harmonic quadratic potential. Suppose the expectation value for its position is given by \(\braket{x} = \frac{a}{2}\sin(\omega t)\). Here, \(a\) is a real constant with units of length and \(\omega \) is an angular frequency. What, if anything, can be concluded about the electron's momentum?
\end{enumerate}
\end{document}
