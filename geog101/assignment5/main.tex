% LaTeX for GEOG101 Assignment 1
% declare document
\documentclass[a4paper, 12pt]{article}

% include the titling package
\usepackage{titling}
\usepackage{datetime}
\usepackage{authblk}
\usepackage{geometry}
\usepackage{setspace}
\usepackage[bottom]{footmisc}

% reduce page margins
\geometry{left=3cm, right=3cm}

% title information
\title{Europe is important (I guess)}
\author{Calvin Sprouse}
\affil{Geography 101 World Regional Geography}
\date{2024 March 13}

% configure the title block
\setlength{\droptitle}{-7em}
\pretitle{\begin{flushleft}\LARGE}
\posttitle{\par\end{flushleft}}
\preauthor{\begin{flushleft}\large}
\postauthor{\par\end{flushleft}}
\predate{\begin{flushleft}\large}
\postdate{\par\end{flushleft}}

% set spacing
\doublespacing%

% the prompt
% Answer the following question in essay form.  This is not a research paper.  The best answers will be those which combine class lecture notes and the textbook and other readings made available on CANVAS for the course.
% Do the best that you can without plagiarizing others and careful with grammar and spelling.  There are no tricks here. The point is to determine levels of comprehension of readings and lectures (1500 words, double spaced, is probably the minimum it would take to answer both questions properly)
% In a nutshell, why did I spend so much lecture time on the evolution of Western Civilization (Europe and European-derived) in this course on “World Regions”?
% In a much larger shell, describe in detail the two major phases of Western imperialism (what were Europeans after, what did they find, when and where?) and how each phase impacted different parts of the non-Western World in different ways (use detailed examples); impacts, indeed, that have shaped most of the contemporary characteristics of each “World Region” to this day, as summarized in your textbook.

\begin{document}
\maketitle

% in a nutshell...
If one were to take a snapshot of the world in the 1400s and pick out the country most likely to take over the world they would likely pick China. If one were to repeat this experiment in the 16th century they might pick Spain or Portugal for theory recent discovery of mountains of silver in the new world. Yet, one would be wrong on both accounts. If one were to take a snapshot of the world today they would find a lack of global Spanish, Portuguese, or Chinese influence. The world today is western, a product of the European Renaissance. That economies increasingly trend toward European capitalist is an unmistakable sign, that there exist wars for democracy throughout the world is another, and that European countries spearhead movements for global alliances is demonstration of the overwhelming impact of the European Renaissance. No other country would go on to leave an impact on the world today quite like Europe would; and since we study world regional geography to understand the world today we must spend considerable time on its shaping forces. 

\end{document}
